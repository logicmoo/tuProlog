%%%%%%%%%%%%%%%%%%%%%%%%%%%%%%%%%%%%%%%%%%%%%%%%%%%%%%%%%%%%%%%%%%%%%%%%%%%%%
%                                                                           %
%  Helper file to generate a single chapter in the tuProlog documentation.  %
%                                                                           %
%   Andrea Omicini, 2/8/2010                                                %
%                                                                           %
%%%%%%%%%%%%%%%%%%%%%%%%%%%%%%%%%%%%%%%%%%%%%%%%%%%%%%%%%%%%%%%%%%%%%%%%%%%%%

\documentclass[11pt]{report}
%%%%%%%%%%%%%%%%%%%%%%%%%%%%%%%%%%%%%%%%%%%%%%%%%%%%%%%%%%%%%%%%%%%%%%%%%
\usepackage[pdftex]{graphicx}
\DeclareGraphicsExtensions{.png, .jpg, .eps, .pdf, .tif}
%%%%%%%%%%%%%%%%%%%%%%%%%%%%%%%%%%%%%%%%%%%%%%%%%%%%%%%%%%%%%%%%%%%%%%%%%
\newcommand\xa[1]{\appendixname~\ref{app:#1}}
\newcommand\labelsec[1]{\label{sec:#1}}
\newcommand\xs[1]{\sectionname~\ref{sec:#1}}
\newcommand\xsp[1]{\sectionname~\ref{sec:#1} \onpagename~\pageref{sec:#1}}
\newcommand\labelssec[1]{\label{ssec:#1}}
\newcommand\xss[1]{\subsectionname~\ref{ssec:#1}}
\newcommand\xssp[1]{\subsectionname~\ref{ssec:#1} \onpagename~\pageref{ssec:#1}}
\newcommand\labelsssec[1]{\label{sssec:#1}}
\newcommand\xsss[1]{\subsectionname~\ref{sssec:#1}}
\newcommand\xsssp[1]{\subsectionname~\ref{sssec:#1} \onpagename~\pageref{sssec:#1}}
\newcommand\labelfig[1]{\label{fig:#1}}
\newcommand\xf[1]{\figurename~\ref{fig:#1}}
\newcommand\xff[2]{\figurenames~\ref{fig:#1}~and~\ref{fig:#2}}
\newcommand\xfp[1]{\figurename~\ref{fig:#1} \onpagename~\pageref{fig:#1}}
\newcommand\labeltab[1]{\label{tb:#1}}
\newcommand\xt[1]{\tablename~\ref{tb:#1}}
\newcommand\xtt[2]{\tablenames~\ref{tb:#1}~and~\ref{ab:#2}}
\newcommand\xtp[1]{\tablename~\ref{tb:#1} \onpagename~\pageref{tb:#1}}
\newcommand\labelenum[1]{\label{enum:#1}}
\newcommand\xen[1]{(\ref{enum:#1})}
\newcommand\xenp[1]{(\ref{enum:#1}) \onpagename~\pageref{enum:#1}}
%******************************************************************************%
\newcommand\bti[1]{\texttt{\textbf{#1}}}
\newcommand\bt[1]{\texttt{#1}}
\newcommand\template[1]{\textit{Template: }\texttt{#1}}
\newcommand\ttit[1]{\texttt{\textit{#1}}}
%******************************************************************************%
\newcommand{\LIA}{\mbox{\textsf{LIA}}}
\newcommand{\aclt}{\mbox{$\mathcal{ACLT}$}}
\newcommand{\respect}{\mbox{\sf{{R}e{S}pec{T}}}}
\newcommand{\luce}{\mbox{\sf{{L}u{C}e}}}
\newcommand{\tucson}{\mbox{\sf{{T}u{CS}o{N}}}}
\newcommand{\alice}{\mbox{\sf{{aliCE}}}}
\newcommand{\tuprolog}{\mbox{{\sf{tu}}Prolog}}
\newcommand{\apice}{\mbox{\sf{{APICe}}}}
%******************************************************************************%
\newcommand\version[1]{\mbox{Document revision: #1}}
\newcommand\approvedby[1]{\mbox{Approved by: #1}}
\newcommand\receivedby[1]{\mbox{Received by: #1}}
\newcommand\creationdate[1]{\mbox{Creation date: #1}}
\newcommand\lastchangesdate[1]{\mbox{Last Changes date: #1}}
\newcommand\noa[2]{\noindent\emph{Note of the author (#1): }#2\\\\}
\newcommand\logo{
    \begin{figure}[tp]
        \begin{center}
            \includegraphics[width=2cm]{common/logo}
        \end{center}
\end{figure}
}
%******************************************************************************%
\newcommand{\classname}[1]{\texttt{#1}}
\newcommand{\varname}[1]{\texttt{#1}}
\newcommand{\predicate}[1]{\texttt{#1}}
\newcommand{\code}[1]{\texttt{#1}}
\newcommand{\keycap}[1]{\textbf{#1}}
\newcommand{\guibutton}[1]{\textsf{#1}}
\newcommand{\userinput}[1]{\texttt{#1}}
\newcommand{\figref}[1]{\figurename~\ref{#1}}
%******************************************************************************%
%\title{{\huge{\bf{\tuprolog{} Guide\\\mbox{ }\\}}}
%        \tuprolog{} version: 1.2.1\\\mbox{ }\\
%        \tuprolog{} IDE version: 1.1.0\\\mbox{ }\\
%{\small{
%    \version{001}\\
%    \creationdate{2002-09-17}\\
%    \lastchangesdate{2004-09-27}\\
%    % \receivedby{aricci}\\
%    % \approvedby{aricci}\\
%    }}
%}
%
%\author{ \mbox{ }\\DEIS, Universit\`{a} di Bologna a Cesena, Italy
%}

%\date{}

\usepackage{supertabular}
\usepackage[bookmarks=true,bookmarksopen=true,bookmarksnumbered]{hyperref}

\begin{document}

%\logo

%\maketitle

%\tableofcontents
\setcounter{chapter}{0}
%=======================================================================
\chapter{What is \tuprolog{}}
\label{what-is}
%=======================================================================

\tuprolog{} is a light-weight Prolog framework for distributed applications and infrastructures.
%
\tuprolog{} is developed and maintained by the \alice{} research group\footnote{\url{http://www.alice.unibo.it}} at the \textsc{Alma Mater Studiorum}---Universit\`{a} di Bologna.
%
It is built as an Open Source software, released under the LGPL license -- thus allowing also for commercial derivative work --, and made available through the pages of the \apice{} web portal\footnote{\url{http://tuprolog.apice.unibo.it}}.

\tuprolog{} is designed to be \emph{minimal}, dynamically \emph{configurable}, \emph{interoperable}, straightforwardly \emph{integrated} with Java and .NET, and easily \emph{deployable}.

First of all, \tuprolog{} is designed with \textit{minimality} in mind.
%
Accordingly, \tuprolog{} core is a tiny Java object that contains only the essential properties of a Prolog engine.
%
Only the required Prolog features -- like, say, ISO compliance, I/O predicates, DCG operators -- are then to be added to or removed from a \tuprolog{} engine according to the contingent application needs.

The obvious counterpart of minimality is \tuprolog{} \textit{configurability}.
%
In fact, a simple yet powerful mechanism based on the notion of \tuprolog{} \textit{library} is provided tha allows required predicates, functors and operators to be loaded and unloaded in a \tuprolog{} engine, both statically and dynamically.
%
Libraries can be either included in the standard \tuprolog{} distribution, or defined \textit{ad hoc} by the \tuprolog{} user / developer. 

A \tuprolog{} library can be built in different ways. 
%
First of all, a \tuprolog{} library could be straightforwardly written in Prolog.
%
On the other hand, a \tuprolog{} library could also be implemented using either Java or any language of .NET framework---depending on the chosen \tuprolog{} implementation.
%
Finally, a \tuprolog{ library could be built by combining Prolog and Java / .NET languages, thus paving the way for multi-language / multi-paradigm integration.
%
Whatever the language(s) used, a \tuprolog{} library can be either used to configure a \tuprolog{} engine when this is started up, or loaded -- and then unloaded -- dynamically at any time during the engine execution.

\tuprolog{} was first implemented upon Java, then ported upon .NET, and is now available on both platforms.
%
\textit{Deployability} of \tuprolog{} owes a lot to Java and .NET.
%
On the Java side, the requirements for \tuprolog{} installation simply amount to the presence of a standard Java VM, and a Java invocation upon a single JAR file is everything needed to start a \tuprolog{} activity.
%
On the .NET side, \ldots ENRICO ENRICO ENRICO ENRICO

\tuprolog{} \textit{integration} with other languages and paradigms is kept as clean as possible, so that the components of a \tuprolog{} application can be developed by choosing at any step the most suitable paradigm---either declarative/logic or imperative/object-oriented.
%
On the Prolog side, thanks to the \texttt{JavaLibrary} library, any Java entity (object, class, package) can be represented as a Prolog term, and exploited from Prolog.
%
So, for instance, Java packages like Swing and {JDBC} can be directly used from within Prolog, straightforwardly enhancing \tuprolog{} with graphics and database access capabilities.
%
In the same way, \texttt{DotNetLibrary} \ldots o come cavolo si chiama ENRICO ENRICO ENRICO ENRICO
%
On the Java side, a \tuprolog{} engine can be invoked and used as a simple Java object, possibly embedded in beans, or exploited in a multi-threaded context, according to the application needs.
%
Also, a multiplicity of different \tuprolog{} engines can be used from a Java program at the same time, each one configured with its own libraries and knowledge base.
%
In the same way, sbrodolata .NET di ENRICO ENRICO ENRICO ENRICO


Interoperability misteriosa: cosa diciamo??? Ancora quanto segue???

Finally, \textit{interoperability} is developed along two main lines:
Internet standard patterns, and coordination models.
%
So, \tuprolog{} supports interaction via TCP/IP and RMI, and can be
also provided as a CORBA service.
%
In addition, \tuprolog{} supports tuple-based coordination under many
forms.
%
First, components of a \tuprolog{} application can be organised around
Java-based tuple spaces, logic tuple spaces, and \respect{} tuple
centres \cite{respect-scico2001}.
%
Then, \tuprolog{} applications can exploit Internet infrastructures
providing tuple-based coordination services, like \luce{}
\cite{luce-aamas2001} and \tucson{} \cite{tucson-aamas99}.



%\bibliography{common/references}
%\bibliographystyle{plain}

\end{document}
