%%%%%%%%%%%%%%%%%%%%%%%%%%%%%%%%%%%%%%%%%%%%%%%%%%%%%%%%%%%%%%%%%%%%%%%%%%%%%
%                                                                           %
%   tuProlog Documentation                                                  %
%                                                                           %
%   A.Ricci, 5/12/00                                                        %
%   E.Denti, 31/05/01                                                       %
%   A.Ricci, 21/06/01                                                       %
%   A.Ricci, 13/09/01                                                       %
%   A.Ricci, 05/07/02                                                       %
%   G.Piancastelli, 27/09/04                                                %
%   G.Piancastelli, 29/10/06                                                %
%   G.Piancastelli, 30/01/07                                                %
%   G.Piancastelli, 19/04/07                                                %
%                                                                           %
%%%%%%%%%%%%%%%%%%%%%%%%%%%%%%%%%%%%%%%%%%%%%%%%%%%%%%%%%%%%%%%%%%%%%%%%%%%%%

\documentclass[11pt]{report}
%\usepackage{upquote}
\usepackage[bookmarks=true,bookmarksopen=true,bookmarksnumbered]{hyperref}
%%%%%%%%%%%%%%%%%%%%%%%%%%%%%%%%%%%%%%%%%%%%%%%%%%%%%%%%%%%%%%%%%%%%%%%%%
 \newif\ifpdf
 \ifx\pdfoutput\undefined
 \pdffalse % we are not running PDFLaTeX
 \else
 \pdfoutput=1 % we are running PDFLaTeX
 \pdftrue
 \fi
%%%%%%%%%%%%%%%
 \ifpdf
 \usepackage[pdftex]{graphicx}
 \else
 \usepackage{graphicx}
 \fi
%%%%%%%%%%%%%%%
 \ifpdf
 \DeclareGraphicsExtensions{.png, .jpg, .eps, .pdf, .tif}
 \else
 \DeclareGraphicsExtensions{.png, .jpg, .eps}
 \fi
%%%%%%%%%%%%%%%
%\usepackage{harvard}
%%%%%%%%%%%%%%%
%\DeclareGraphicsRule{.gif}{bmp}{}{}% declare GIF filename extension
%\DeclareGraphicsRule{.jpg}{bmp}{}{}% declare WMF filename extension
%\DeclareGraphicsRule{.wmf}{bmp}{}{}% declare WMF filename extension
%%%%%%%%%%%%%%%%%%%%%%%%%%%%%%%%%%%%%%%%%%%%%%%%%%%%%%%%%%%%%%%%%%%%%%%%%
% -- reducing margins
 %\setlength{\oddsidemargin}{0pt}
 %\setlength{\evensidemargin}{0pt}
 %\setlength{\textwidth}{16cm}
 %
 %\setlength{\topmargin}{-0.4in}
 %\setlength{\textheight}{9.6in}
%%%%%%%%%%%%%%%%%%%%%%%%%%%%%%%%%%%%%%%%%%%%%%%%%%%%%%%%%%%%%%%%%%%%%%%%%
\newcommand\xa[1]{\appendixname~\ref{app:#1}}
\newcommand\labelsec[1]{\label{sec:#1}}
\newcommand\xs[1]{\sectionname~\ref{sec:#1}}
\newcommand\xsp[1]{\sectionname~\ref{sec:#1} \onpagename~\pageref{sec:#1}}
\newcommand\labelssec[1]{\label{ssec:#1}}
\newcommand\xss[1]{\subsectionname~\ref{ssec:#1}}
\newcommand\xssp[1]{\subsectionname~\ref{ssec:#1} \onpagename~\pageref{ssec:#1}}
\newcommand\labelsssec[1]{\label{sssec:#1}}
\newcommand\xsss[1]{\subsectionname~\ref{sssec:#1}}
\newcommand\xsssp[1]{\subsectionname~\ref{sssec:#1} \onpagename~\pageref{sssec:#1}}
\newcommand\labelfig[1]{\label{fig:#1}}
\newcommand\xf[1]{\figurename~\ref{fig:#1}}
\newcommand\xff[2]{\figurenames~\ref{fig:#1}~and~\ref{fig:#2}}
\newcommand\xfp[1]{\figurename~\ref{fig:#1} \onpagename~\pageref{fig:#1}}
\newcommand\labeltab[1]{\label{tb:#1}}
\newcommand\xt[1]{\tablename~\ref{tb:#1}}
\newcommand\xtt[2]{\tablenames~\ref{tb:#1}~and~\ref{ab:#2}}
\newcommand\xtp[1]{\tablename~\ref{tb:#1} \onpagename~\pageref{tb:#1}}
\newcommand\labelenum[1]{\label{enum:#1}}
\newcommand\xen[1]{(\ref{enum:#1})}
\newcommand\xenp[1]{(\ref{enum:#1}) \onpagename~\pageref{enum:#1}}
%******************************************************************************%
\newcommand\bti[1]{\texttt{\textbf{#1}}}
\newcommand\bt[1]{\texttt{#1}}
\newcommand\template[1]{\textit{Template: }\texttt{#1}}
\newcommand\ttit[1]{\texttt{\textit{#1}}}
%******************************************************************************%
\newcommand{\LIA}{\mbox{\textsf{LIA}}}
\newcommand{\aclt}{\mbox{$\mathcal{ACLT}$}}
\newcommand{\respect}{\mbox{\sf{{R}e{S}pec{T}}}}
\newcommand{\luce}{\mbox{\sf{{L}u{C}e}}}
\newcommand{\tucson}{\mbox{\sf{{T}u{CS}o{N}}}}
\newcommand{\alice}{\mbox{\sf{{aliCE}}}}
\newcommand{\apice}{\mbox{\sf{{APICe}}}}
\newcommand{\tuprolog}{\mbox{{\sf{tu}}Prolog}}
%******************************************************************************%
\newcommand\version[1]{\mbox{Document revision: #1}}
\newcommand\approvedby[1]{\mbox{Approved by: #1}}
\newcommand\receivedby[1]{\mbox{Received by: #1}}
\newcommand\creationdate[1]{\mbox{Creation date: #1}}
\newcommand\lastchangesdate[1]{\mbox{Last Changes date: #1}}
\newcommand\noa[2]{\noindent\emph{Note of the author (#1): }#2\\\\}
\newcommand\logo{
    \begin{figure}[tp]
        \begin{center}
            \includegraphics[width=5cm]{images/logo}
        \end{center}
\end{figure}
}
%******************************************************************************%
\newcommand{\classname}[1]{\texttt{#1}}
\newcommand{\varname}[1]{\texttt{#1}}
\newcommand{\predicate}[1]{\texttt{#1}}
\newcommand{\code}[1]{\texttt{#1}}
\newcommand{\keycap}[1]{\textbf{#1}}
\newcommand{\guibutton}[1]{\textsf{#1}}
\newcommand{\userinput}[1]{\texttt{#1}}
\newcommand{\figref}[1]{\figurename~\ref{#1}}
%******************************************************************************%
\title{{\huge{\bf{\tuprolog{} Guide\\\mbox{ }\\}}}
        \tuprolog{} version: 2.1\\\mbox{ }\\
        \tuprolog{} IDE version: 2.0\\\mbox{ }\\
{\small{
    \lastchangesdate{2007-04-19}\\
    % \receivedby{aricci}\\
    % \approvedby{aricci}\\
    }}
}

\author{ \mbox{ }\\ \textsc{Alma Mater Studiorum}---Universit\`{a} di Bologna a Cesena, Italy
}

\date{}

\begin{document}

\logo

\maketitle

\tableofcontents

%%=======================================================================
\chapter{What is \tuprolog{}}
\label{what-is}
%=======================================================================

\tuprolog{} is a light-weight Prolog framework for distributed applications and infrastructures.
%
\tuprolog{} is developed and maintained by the \alice{} research group\footnote{\url{http://www.alice.unibo.it}} at the \textsc{Alma Mater Studiorum}---Universit\`{a} di Bologna.
%
It is built as an Open Source software, released under the LGPL license -- thus allowing also for commercial derivative work --, and made available through the pages of the \apice{} web portal\footnote{\url{http://tuprolog.apice.unibo.it}}.

\tuprolog{} is designed to be \emph{minimal}, dynamically \emph{configurable}, \emph{interoperable}, straightforwardly \emph{integrated} with Java and .NET, and easily \emph{deployable}.

First of all, \tuprolog{} is designed with \textit{minimality} in mind.
%
Accordingly, \tuprolog{} core is a tiny Java object that contains only the essential properties of a Prolog engine.
%
Only the required Prolog features -- like, say, ISO compliance, I/O predicates, DCG operators -- are then to be added to or removed from a \tuprolog{} engine according to the contingent application needs.

The obvious counterpart of minimality is \tuprolog{} \textit{configurability}.
%
In fact, a simple yet powerful mechanism based on the notion of \tuprolog{} \textit{library} is provided tha allows required predicates, functors and operators to be loaded and unloaded in a \tuprolog{} engine, both statically and dynamically.
%
Libraries can be either included in the standard \tuprolog{} distribution, or defined \textit{ad hoc} by the \tuprolog{} user / developer. 

A \tuprolog{} library can be built in different ways. 
%
First of all, a \tuprolog{} library could be straightforwardly written in Prolog.
%
On the other hand, a \tuprolog{} library could also be implemented using either Java or any language of .NET framework---depending on the chosen \tuprolog{} implementation.
%
Finally, a \tuprolog{ library could be built by combining Prolog and Java / .NET languages, thus paving the way for multi-language / multi-paradigm integration.
%
Whatever the language(s) used, a \tuprolog{} library can be either used to configure a \tuprolog{} engine when this is started up, or loaded -- and then unloaded -- dynamically at any time during the engine execution.

\tuprolog{} was first implemented upon Java, then ported upon .NET, and is now available on both platforms.
%
\textit{Deployability} of \tuprolog{} owes a lot to Java and .NET.
%
On the Java side, the requirements for \tuprolog{} installation simply amount to the presence of a standard Java VM, and a Java invocation upon a single JAR file is everything needed to start a \tuprolog{} activity.
%
On the .NET side, \ldots ENRICO ENRICO ENRICO ENRICO

\tuprolog{} \textit{integration} with other languages and paradigms is kept as clean as possible, so that the components of a \tuprolog{} application can be developed by choosing at any step the most suitable paradigm---either declarative/logic or imperative/object-oriented.
%
On the Prolog side, thanks to the \texttt{JavaLibrary} library, any Java entity (object, class, package) can be represented as a Prolog term, and exploited from Prolog.
%
So, for instance, Java packages like Swing and {JDBC} can be directly used from within Prolog, straightforwardly enhancing \tuprolog{} with graphics and database access capabilities.
%
In the same way, \texttt{DotNetLibrary} \ldots o come cavolo si chiama ENRICO ENRICO ENRICO ENRICO
%
On the Java side, a \tuprolog{} engine can be invoked and used as a simple Java object, possibly embedded in beans, or exploited in a multi-threaded context, according to the application needs.
%
Also, a multiplicity of different \tuprolog{} engines can be used from a Java program at the same time, each one configured with its own libraries and knowledge base.
%
In the same way, sbrodolata .NET di ENRICO ENRICO ENRICO ENRICO


Interoperability misteriosa: cosa diciamo??? Ancora quanto segue???

Finally, \textit{interoperability} is developed along two main lines:
Internet standard patterns, and coordination models.
%
So, \tuprolog{} supports interaction via TCP/IP and RMI, and can be
also provided as a CORBA service.
%
In addition, \tuprolog{} supports tuple-based coordination under many
forms.
%
First, components of a \tuprolog{} application can be organised around
Java-based tuple spaces, logic tuple spaces, and \respect{} tuple
centres \cite{respect-scico2001}.
%
Then, \tuprolog{} applications can exploit Internet infrastructures
providing tuple-based coordination services, like \luce{}
\cite{luce-aamas2001} and \tucson{} \cite{tucson-aamas99}.



%%=====================================================================
\chapter{Installing \tuprolog{}}
\label{installation}
%=====================================================================

Quite obviously, the installation procedure depends on the platform of choice.
For Java, Microsoft .NET and Android, the first step is to manually download the desired distribution (or even just the single binary file) from the \tuprolog{} web site, \texttt{tuprolog.alice.unibo.it}, or directly from the Google code repository, \texttt{tuprolog.googlecode.com}; for Eclipse the procedure is different, since the plug-in installation has to be performed via the Eclipse Plugin Manager.

As a further alternative, users wishing to have a look at \tuprolog{} and trying it without installing anything on their computer can do so by exploiting the `Run via Java Web Start' option, available on the \tuprolog{} web site.

\section{Installation in Java}

The complete Java distribution has the form of a single \texttt{zip} file which contains everything (binaries, sources, documentation, examples, etc.) and unzips into a multi-level directory tree, similar to the following (only first-level sub-dirs are shown):

\begin{Verbatim}[frame=single, framerule=0.5mm, samepage=true, boxwidth=5cm]
    2p
    |---ant
    |---bin
    |---build
    |   |---archives
    |   |---classes
    |   |---release
    |   |---reports
    |   |---tests
    |---doc
    |   |---javadoc
    |---lib
    |---src
    |---test
    |   |---fit
    |   |---unit
    |---tmp
    |---test
\end{Verbatim}

An alternative distribution, without sources, is also available in the \textit{Download} section of the \tuprolog{} repository: obviously, in this case only a subset of the above folders is present (namely, only \texttt{bin}, \texttt{doc}, \texttt{lib} and \texttt{reports}).

If you are only interested in the Java binaries, just look into the \texttt{build/archives} directory, which contains two JAR files:
%
\begin{itemize}
%
\item \texttt{2p.jar}, which contains everything you need to use \tuprolog{},
  such as the core API, the \texttt{Agent} application, libraries, GUI,
  etc.; this is a runnable JAR, that open the \tuprolog{} IDE when double-clicked.
%
\item \texttt{tuprolog.jar}, which contains only the core part of \tuprolog{},
  namely, what you will need to include in a Java application project to be able to access the \tuprolog{} classes, and write multi-paradigm Java/Prolog applications.
\end{itemize}

The other folders contain project-specific files: \texttt{src} contains all the sources, \texttt{doc} all the documentation, \texttt{lib} the libraries used by the \tuprolog{} project, \texttt{test} the sources for the \tuprolog{} test suite (partly as FIT test, partly as JUnit tests), \texttt{ant} some Ant scripts to automate the build of parts of the \tuprolog{} project, etc.


\section{Installation in .NET}

The complete .NET distribution has also the form of a single \texttt{zip} file containing everything; however, due to the automatic generation of \tuprolog{} .NET binaries via IKVM from Java (more on this in Chapter \ref{ch:mpp-in-dotnet}), the unzipped directory tree is simpler, as there are no sources (and therefore no tests, no ant tasks, etc), except for \texttt{OOLibrary} and Conventions, which are NET-specific and therefore written in C\#.
%
So, the resulting tree is similar to the following:
%
\begin{verbatim}
    2p
    |---build
    |   |---examples
    |   |---lib
    |---OOLibrary
    |   |---Conventions
    |   |---Fixtures
    |   |---OOLibrary
\end{verbatim}
%
Here, too, an alternative distribution, without the OOLibrary and conventions sources, is also available in the \textit{Download} section of the \tuprolog{} repository: again, only a subset of the above folders is present in this case.

The .NET binary, \texttt{2p.exe}, can be found in the \texttt{build} folder.


\section{Installation in Android}

The Android distribution has the form of a single \texttt{apk} file, to be installed via install mechanism provided by the Android OS.
So, unless you are interested in the implementation details, there should be no need to download the whole project distribution.
If, however, you like to do so, you will eventually get to a directory tree similar to the following (only the most relevant first-level sub-folders are shown):
%
\begin{verbatim}
    2p
    |---assets
    |---bin
    |   |---classes
    |   |---res
    |---doc
    |---gen
    |---libs
    |---res
    |---screenshots
    |---src
\end{verbatim}
%
The APK binary can be found into the \texttt{bin} folder.

As for the Java case, the other folders contain project-specific files: in particular, \texttt{src} contains the sources, \texttt{res} the Android resources automatically generated during the project build process, \texttt{libs} the libraries used by this project---mainly, the \texttt{tuprolog.jar} file of the corresponding Java version, imported here as an external dependency.


\section{Installation in Eclipse}

The installation procedure is different for the Eclipse platform due to the need to conform to the Eclipse standard procedure for plug-in installation via Plugin Manager.
%Please see the specific section on the \tuprolog{} web site for detailed, screenshot-driven instruction.
%
In order to exploit Eclipse's built-in plugin installation manager, a properly-configured \textit{update site}
must be added to the Eclipse Update Site List first.

To do so:

\begin{enumerate}
  \item open the Eclipse preferences (menu Window $>$ Preferences) and choose the Install/Update item
  and choose the Available Software Sites sub-item. You might want to type ``tuprolog'' in the text field 
  just to check that no other update sites are already defined for it.

  \item now click on the ``Add'' button to add a new software site: in the dialog that appears, type a 
  description in the upper field (e.g. ``tuProlog update site''), and enter the following URL in the lower field:\\
     {\footnotesize{\texttt{http://tuprolog.googlecode.com/svn/2p-plugin/trunk/tuPrologUpdateSite/}}}\\
  The dialog should now look as in Figure \ref{fig:tuPrologPluginInstall-12}.
  Clicking OK, you should now see the new site in the site list.

  \item close this window, and go back to the main Eclipse window. Open the Help menu and choose the 
  \textit{Install new software} item (Figure \ref{fig:tuPrologPluginInstall-34}, top).
  Select the \tuprolog{} software by typing ``tuProlog'' in the filter text field, or by scrolling the site 
  list: after selecting the site, you should see something like the window shown in 
  Figure \ref{fig:tuPrologPluginInstall-4}, bottom.

  \item Now select the \tuprolog{} feature by clicking on the checkbox. If multiple feature versions are 
  proposed (that depends whether you checked the ``show older versions'' option), choose the version you 
  prefer: if unsure, select the most recent.
  Once selected, click Next: installation will take place automatically.
\end{enumerate}

The \tuprolog{} plugin is now installed on your Eclipse system.

\begin{figure}
\centering
  \includegraphics[width=300px]{images/tuPrologPluginInstall-1.png}\\
  \includegraphics[width=300px]{images/tuPrologPluginInstall-2.png}
  \caption{Plugin installation: adding the Update Site, phase 1}\label{fig:tuPrologPluginInstall-12}
\end{figure}

\begin{figure}
\centering
  \includegraphics[width=300px]{images/tuPrologPluginInstall-3.png}\\
  \includegraphics[width=300px]{images/tuPrologPluginInstall-4.png}
  \caption{Plugin installation: adding the Update Site, phase 2}\label{fig:tuPrologPluginInstall-34}
\end{figure}

%%=======================================================================
\chapter{Getting Started}
\label{getting-started}
%=======================================================================

The \tuprolog{} distribution offers some tools either to consult and execute already existing Prolog programs, or to help developing new Prolog theories and interact with a Prolog engine. %
Depending on the use you would like to make of \tuprolog{}, you may want to start exploring the distribution tools along different directions.

%=======================================================================
\section{Prolog Programmer Quick Start}
%=======================================================================

As a Prolog programmer, you would like to start trying \tuprolog{} by running your
already existing Prolog programs. You can execute your programs in the form of source
text files using the \tuprolog{} Agent tool. This tool accepts as arguments the name of
a text file containing a Prolog theory and, optionally, the goal to be solved; then it starts
the demonstration. Once you have properly installed \tuprolog{} in the \emph{dir}
directory, you can use the following template to invoke the Agent tool from the
command line:\\\\
%
\texttt{java -cp \emph{dir}/2p.jar\\
\mbox{~~~~~~~~~}alice.tuprolog.Agent \textit{PrologTextFile}
\{\textit{Goal}\}\\\\}
%
For instance, suppose a text file named \verb|hello.pl| in your current directory contains the following line:
\begin{verbatim}
go :- write('hello, world!'), nl.
\end{verbatim}
In order to execute this Prolog program, you can type at the command prompt:\\\\
%
\texttt{java -cp \emph{dir}/2p.jar alice.tuprolog.Agent hello.pl go.\\\\}
%
Then, the Agent tool tries to prove the goal \texttt{go} with respect to the theory contained in \texttt{hello.pl}. 
%
As a result, the string \texttt{hello, world!} should appear on your standard output.

Also, the goal to be proven can be embedded within the Prolog source by means of the \texttt{solve} directive.
%
For instance, suppose that the text file \texttt{hellogo.pl} in your current directory contains the following lines:
\begin{verbatim}
:- solve(go).
go :- write('hello, world!'), nl.
\end{verbatim}
Then, type:\\\\
%
\texttt{java -cp \emph{dir}/2p.jar alice.tuprolog.Agent hellogo.pl\\\\}
%
Again, this will make \texttt{hello, world!} appear on your standard output.

%=======================================================================
\section{Developer Quick Start}
%=======================================================================

The first thing you may want to do as a developer would probably be to take advantage
of the tools embedded in the Graphical User Interface included in the \tuprolog{}
distribution. The GUI can be obtained by issuing the following command:\\\\
%
\texttt{java -cp \emph{dir}/2p.jar alice.tuprologx.ide.GUILauncher\\\\}
%
The development environment provided by the GUI makes standard Prolog features
easily accessible, such as asking queries, viewing the current solution along
with the related variable substitution, backtracking, and so on. Also, it
enables you to view and edit the current Prolog theory contained in the engine,
and to spy \tuprolog{} at work during goal demonstrations. Finally, it also
offers a facility to dynamically load and unload predicate libraries within the
\tuprolog{} engine.

It is worth remembering that the file \texttt{2p.jar} is an executable Java Archive, so
by invoking the command:\\\\
%
\texttt{java -jar 2p.jar\\\\}
%
in the \emph{dir} directory, or by double-clicking it under most operating systems, the
graphic user interface console is automatically spawned.

You may also want to experience a pure interactive environment on a \tuprolog{}
engine. In this case, you need to get the \tuprolog{} prompt using the command line
shell provided within the distribution. To obtain it, just type:\\\\
%
\texttt{java -cp \emph{dir}/2p.jar alice.tuprologx.ide.CUIConsole\\\\}
%
\noindent which starts a \tuprolog{} interpreter to be used via console, in a sort of
Command Line User Interface mode. To exit the \tuprolog{} console, you have to
issue the standard \texttt{halt.} command.

%%=======================================================================
\chapter{\tuprolog{} Basics}
\label{ch:engine}
%=======================================================================
%
% General stuff about 2P engine features
%
%

\noindent This chapter provides a brief introduction to the basic elements and structure of the \tuprolog{} engine, covering syntax, programming support, and built-in predicates directly provided by the engine.

%---------------------------------------------------------------------
\section{Structure of a \tuprolog{} Engine}
%---------------------------------------------------------------------

\noindent A \tuprolog{} engine has a layered structure, where provided and recognised predicates are organised into three different categories:
%
\begin{description}
\item[built-in predicates] |
Predicates embedded in any \tuprolog{} engine are called built-in predicates.
%
Whatever modification is made to the engine either before or during execution time, it does not affect the number and properties of the built-in predicates.
%
\item[library predicates] |
Predicates loaded in a \tuprolog{} engine by means of a \tuprolog{} library are called library predicates.
%
Since libraries can be loaded and unloaded in \tuprolog{} engines freely at the system start-up, or dynamically at execution time, the set of the library predicates of a \tuprolog{} engine is not fixed, and can change from engine to engine, and in the same engine at different times.
%
\tuprolog{} libraries can be built by mixing Java and Prolog code. Prolog
library predicates can be overridden by Prolog theory predicates. Both Java and
Prolog library predicates cannot be individually retracted: if you want to
remove a single library predicate from the engine, you need to unload the whole
library containing that predicate.
\item[theory predicates] |
Predicates loaded in a \tuprolog{} engine by means of a \tuprolog{} theory are called theory predicates.
%
Since theories can be loaded and unloaded in \tuprolog{} engines freely at the system start-up, or dynamically at execution time, the set of the theory predicates of a \tuprolog{} engine is not fixed, and can change from engine to engine, and in the same engine at different times. 
%
\tuprolog{} theories are simple collections of Prolog clauses.
\end{description}
%
Even though they may seem similar, library and theory predicates are handled differently in a \tuprolog{} engine.

First of all, they are conceptually different. In fact, while theory predicates should be used to axiomatically represent domain knowledge at the time the proof is performed, library predicates should more or less be used to represent what is required (procedural knowledge, utility predicates) in order to actually and effectively perform a (number of) proof(s) in the domain of interest: therefore, library predicates represent more ``stable'' knowledge, which is encapsulated once and for all (at least approximately) within a library container.

Since library and theory predicates are also structurally different, they are handled differently by the engine, and represented differently in the run-time: correspondingly, they have different level of observation when monitoring or debugging a working \tuprolog{} engine.
%
As a consequence, developer tools provided by \tuprolog{} IDE typically show in a separate way the theory axioms or predicates and the loaded libraries or predicates.
%
In addition, the debugging phase typically neglects library predicates (which, as mentioned above, are also conceived as more stable and well-tested), while the effect of the theory predicates is dutifully put in evidence during controlled execution.

%---------------------------------------------------------------------
\section{Prolog syntax}
%---------------------------------------------------------------------

\noindent The term syntax supported by \tuprolog{} engine is basically ISO compliant,\footnote{Currently ISO exceptions, ISO I/O predicates and some ISO directives are not supported.}
and accounts for several elements:
%
\begin{description}
\item[Comments and Whitespaces] -- Whitespaces consist of blanks (including tabs and formfeeds), end-of-line marks, and comments. A whitespace can be put before and after any term, operator, bracket, or argument separator, as long as it does not break up an atom or number or separate a functor from the opening parenthesis that introduces its argument lists.
%
For instance, atom \bt{p(a,b,c)} can be written as \bt{p(\mbox{~a~},\mbox{~b~},\mbox{~c~})}, but not as \bt{\mbox{p~}(a,b,c)}).
%
Two types of comments are supported: one type begins with \bt{/*} and ends with \bt{*/}, the other begins with \bt{\%} and ends at the end of the line.
%
Nested comments are not allowed.
%
\item[Variables] |
A variable name begins with a capital
letter or the underscore mark (\bt{\_}), and consists of letters,
digits, and/or underscores.
%
A single underscore mark denotes an anonymous variable.
%
\item[Atoms] |
There are four types of atoms:
\emph{(i)} a series of letters, digit, and/or underscores, beginning with a lower-case letter; \emph{(ii)} a series of one or more characters from the set \{\texttt{\#}, \texttt{\$}, \texttt{\&}, \texttt{*}, \texttt{+}, \texttt{-}, \texttt{.}, \texttt{/}, \texttt{:}, \texttt{<}, \texttt{=}, \texttt{>}, \texttt{?}, \texttt{@}, \texttt{\textasciicircum}, \texttt{\~}\}, provided it does not begin with \texttt{/*};
\emph{(iii)} The special atoms \texttt{[]} and \texttt{\{\}};
\emph{(iv)} a single-quoted string.
%
\item[Numbers] |
Integers and float are supported.
%
The formats supported for integer numbers are decimal, binary (with \verb|0b|
prefix), octal (with \verb|0o| prefix), and hexadecimal (with \verb|0x|
prefix). The character code format for integer numbers (prefixed by \verb|0'|) is supported only for alphanumeric characters, the white space, and characters in the set \{\texttt{\#}, \texttt{\$}, \texttt{\&}, \texttt{*}, \texttt{+}, \texttt{-}, \texttt{.}, \texttt{/}, \texttt{:}, \texttt{<}, \texttt{=}, \texttt{>}, \texttt{?}, \texttt{@}, \texttt{\textasciicircum}, \texttt{\~}\}.
%
The range of integers is -2147483648 to 2147483647; the range of floats is
-2E+63 to 2E+63-1.
%
Floating point numbers can be expressed also in the exponential format (e.g. \bt{-3.03E-05}, \bt{0.303E+13}).
%
A minus can be written before any number to make it negative (e.g. \bt{-3.03}).
%
Notice that the minus is the sign-part of the number itself; hence \bt{-3.4} is a number, not an expression (by contrast, \bt{- 3.4} is an expression).
%
\item[Strings] |
A series of ASCII characters, embedded in quotes \verb|'| or \verb|"|.
%
Within single quotes, a single quote is written double (e.g, \verb|'don''t forget'|).
%
A backslash at the very end of the line denotes continuation to the next line, so that: \\
\verb|'this is \ |\\
\verb|a single line'|\\
is equivalent to \verb|'this is a single line'| (the line break is ignored).
%
Within a string, the backslash can be used to denote special characters, such as \verb|\n| for a newline,
\verb|\r| for a return without newline,
\verb|\t| for a tab character,
\verb|\\| for a backslash,
\verb|\'| for a single quote,
\verb|\"| for a double quote.
%
\item[Compounds] |
The ordinary way to write a compound is to write the functor (as an atom), an opening parenthesis, without spaces between them, and then a series of terms separated by commas, and a closing parenthesis: \bt{f(a,b,c)}.
%
This notation can be used also for functors that are normally written as operators, e.g. \bt{2+2} = \verb|'+'(2,2)|.
%
Lists are defined as rightward-nested structures using the dot operator \verb|'.'|; so, for example: \\
\bt{[a] =} \verb|'.'(a,[])|\\
\bt{[a,b] =} \verb|'.'(a,'.'(b,[]))|\\
\bt{[a,b|c] =} \verb|'.'(a,'.'(b,c))|\\
%
There can be only one \bt{|} in a list, and no commas after it.
%
Also curly brackets are supported: any term enclosed with \bt{$\{$} and \bt{$\}$} is treated as the argument of the special functor \verb|'{}'|:  \verb|{hotel}| = \verb|'{}'(hotel)|, \bt{$\{$1,2,3$\}$} = \verb|'{}'(1,2,3)|.
%
Curly brackets can be used in the Definite Clause Grammars theory.

\item[Operators] |
Operators are characterised by a name, a specifier, and a priority.
%
An operator name is an atom, which is not univocal: the same atom can be an operator in more than one class, as in the case of the infix and prefix minus signs.
%
An operator  specifier is a string like \texttt{xfy}, which gives both its class (infix, postfix and prefix) and its associativity: \texttt{xfy} specifies that the grouping on the right should be formed first, \texttt{yfx} on the left, \texttt{xfx} no priority.
%
An operator priority is a non-negative integer ranging from 0 (max priority) and 1200 (min priority).

Operators can be defined by means of either the \bt{op/3} predicate or directive.
%
No predefined operators are directly given by the raw \tuprolog{} engine, whereas a number of them is provided through libraries.
%
\item[Commas] |
The comma has three functions: it separates arguments of functors, it separates elements of lists, and it is an infix operator of priority 1000.
%
Thus \bt{(a,b)} (without a functor in front) is a compound, equivalent to \verb|','(a,b)|.
%
\item[Parenthesis] -- Parenthesis are allowed around any term.
%
The effect of parenthesis is to override any grouping that may
otherwise be imposed by operator priorieties.
%
Operators enclosed in parenthesis do not function as operators;
thus \bt{2(+)3} is a syntax error.
\end{description}

%---------------------------------------------------------------------
\section{Configuration of a \tuprolog{} Engine}
%---------------------------------------------------------------------
\noindent Prolog developers have four different means to configure a \tuprolog{} engine in order to fit their application needs.
%
In fact, a \tuprolog{} can be suitably configured by means of:

\begin{description}
%
\item[Theories] |
A \tuprolog{} theory is represented by a text, consisting of a sequence of clauses and/or directives.
%
Clauses and directives are terminated by a dot, and are separated by a whitespace character.
%
Theories can be loaded or unloaded by means of suitable library predicates, which are described in Chapter \ref{ch:standard-libraries}.
%
\item[Directives] |
A directive can be given by means of the \bt{:-/1} predicate, which is natively supported by the engine, and can be used to configure and use a \tuprolog{} engine (\bt{set\_prolog\_flag/1}, \bt{load\_library/1}, \bt{consult/1}, \bt{solve/1}), format and syntax of read-terms\footnote{As specified by the ISO standard, a read-term is a Prolog term followed by an end token, composed by an optional layout text sequence and a dot.} (\bt{op/3}).
%
Directives are described in detail in the following sections.
%
\item[Flags] |
A \tuprolog{} engine allows the dynamic definition of flags (or properties) describing some aspects of libraries and their predicates and evaluable functors.
%
A flag is identified by a name (an alphanumeric atom), a list of possible values, a default value, and a boolean value specifying if the flag value can be modified.
%
Dynamically, a flag value can be changed (if modifiable) with a new value included in the list of possible values.
%
\item[Libraries] |
A \tuprolog{} engine can be dynamically extended by loading or unloading libraries.
%
Each library can provide a specific set of predicates, functors, and a related theory, which also allows new flags and operators to be defined.
%
Libraries can be either pre-defined (see Chapter \ref{ch:standard-libraries}) or user-defined (see Chapter \ref{ch:howto-develop-libraries}).
%
A library can be loaded by means of the predicate \texttt{load\_library} (Prolog side), or by means of the method \texttt{loadLibrary} of the \tuprolog{} engine (Java side).
\end{description}
%
Currently \tuprolog{} does not support exception management: actually, an exception causes the predicate/functor in which it occurred to fail and be false.
%

%---------------------------------------------------------------------
\section{Built-in predicates}
%---------------------------------------------------------------------

\noindent This section contains a comprehensive list of the built-in predicates provided by the \tuprolog{} engine, that is, those predicates defined directly in its core.

Following an established convention in built-in argument template description, which takes root into an imperative interpretation, the symbol \bt{+} in front of an argument means an \emph{input argument}, \bt{-} means \emph{output argument}, \bt{?} means \emph{input/output} argument, \bt{@} means \emph{input argument} that must be bound.

%---------------------------------------------------------------------
\subsection{Control management}
%---------------------------------------------------------------------

\begin{itemize}
%
\item \bti{true/0}\\
\noindent\bt{true} is true.
%
\item \bti{fail/0}\\
\noindent\bt{fail} is false.
%
\item \verb|','/2|\\
\noindent\verb|','(First,Second)| is true if and only if both \bt{First}
and \bt{Second} are true.
%
\item \bti{!/0}\\
\noindent\bt{!} is true. All choice points between the cut and the
parent goal are removed. The effect is a commitment to use both the
current clause and the substitutions found at the point of the
cut.
%
\item \verb|'$call'/1|\\
\noindent\verb|'$call'(Goal)| is true if and only if \bt{Goal}
represents a goal which is true. It is not opaque to cut.\\
\template{'\$call'(+callable\_term)}
%
\item \bti{halt/0}\\
\noindent\bt{halt} terminates a Prolog demonstration, exiting the
Prolog processor and returning to the system that invoked the
processor.
%
\item \bti{halt/1}\\
\noindent\bt{halt(X)} terminates a Prolog demonstration, exiting
the Prolog processor and returning to the systems that invoked the
processor passing the value of \bt{X} as a message.\\
\template{halt(+int)}
%
\end{itemize}

%---------------------------------------------------------------------
\subsection{Term Unification and Management}
%---------------------------------------------------------------------

\begin{itemize}
%
\item \bti{is/2}\\
\noindent\bt{is(X, Y)} is true iff \bt{X} is unifiable with the
value of the expression \bt{Y}.\\
\noindent\template{is(?term, @evaluable)}
%
\item \verb|'='/2|\\
\noindent\verb|'='(X, Y)| is true iff \bt{X} and \bt{Y} are
unifiable.\\
\noindent\template{'='(?term, ?term)}
%
\item \verb|'\='/2|\\
\noindent\verb|'\='(X, Y)| is true iff \bt{X} and \bt{Y} are
not unifiable. \\
\noindent\template{'$\setminus$='(?term, ?term)}
%
%
\item \verb|'$tolist'/2|\\
\noindent\verb|'$tolist'(Compound, List)| is true if \bt{Compound} is
a compound term, and in this case \bt{List} is list representation
of the compound, with the name as first element and all the
arguments as other elements.\\
\noindent\template{'\$tolist'(@struct, -list)}
%
 \item \verb|'$fromlist'/2|\\
 \noindent\verb|'$fromlist'(Compound, List)| is true if \bt{Compound}
 unifies with the list representation of \bt{List}.\\
\noindent\template{'\$fromlist'(-struct, @list)}
%
 \item \bti{copy\_term/2}\\
 \noindent\bt{copy\_term(Term1, Term2)} is true iff \bt{Term2}
 unifies with the a renamed copy of \bt{Term1}.\\
\noindent\template{copy\_term(?term, ?term)}
%
 \item \verb|'$append'/2|\\
 \noindent\verb|'$append'(Element, List)| is true if \bt{List} is a
 list, with the side effect that the \bt{Element} is appended to
 the list.\\
\noindent\template{'\$append'(+term, @list)}
%
\end{itemize}

%---------------------------------------------------------------------
\subsection{Knowledge-base management}
%---------------------------------------------------------------------

\begin{itemize}
%
 \item \verb|'$find'/2|\\
 \noindent\verb|'$find'(Clause, ClauseList)| is true if \bt{ClauseList}
 is a list, and \bt{Clause} is a clause, with the side effect that
 all the clauses of the database matching \bt{Clause} are
 appended to the list.\\
\noindent\template{'\$find'(@clause, @list)}
%
\item \bti{abolish/1}\\
\noindent\bt{abolish(PI)} completely wipes out the dynamic
predicate matching the predicate indicator \texttt{PI}.\\
\noindent\template{\bt{abolish(@term)}}
%
\item \bti{asserta/1}\\
\noindent\bt{asserta(Clause)} is true, with the side effect that
the clause \bt{Clause} is added to the beginning of database.\\
\noindent\template{asserta(@clause)}
%
\item \bti{assertz/1}\\
\noindent\bt{assertz(Clause)} is true, with the side effect that
the clause \bt{Clause} is added to the end of the database.\\
\noindent\template{assertz(@clause)}
%
\item \verb|'$retract'/1|\\
\noindent\verb|'$retract'(Clause)| is true if the database contains
at least one clause unifying with \bt{Clause}. As a side effect, the
clause is removed from the database. It is not re-executable.\\
\noindent\template{'\$retract'(@clause)}
%
\end{itemize}

%---------------------------------------------------------------------
\subsection{Operators and Flags Management}
%---------------------------------------------------------------------

\begin{itemize}
%
\item \bti{op/3}\\
\noindent\bt{op(Priority, Specifier, Operator)} is true. It always succeeds,
modifying the operator table as a side effect. If \bt{Priority} is 0, then
\bt{Operator} is removed from the operator table; else, \bt{Operator} is
added to the operator table, with priority (lower binds tighter) \bt{Priority}
and associativity determined by \bt{Specifier}. If an operator with the same
\bt{Operator} symbol and the same \bt{Specifier} already exists in the operator
table, the predicate modifies its priority according to the specified \bt{Priority}
argument.\\
\noindent\template{op(+integer, +specifier, @atom\_or\_atom\_list)}
%
%  \item \bti{flag/2}\\
%  \noindent\bt{flag(FlagName, NewValue)} is true if \bt{FlagName} is
%  the name of a modifiable flag currently defined in the engine
%  and \bt{NewValue} is a valid value for the flag. As a side effect,
%  \bt{NewValue} becomes the new value of the flag \bt{FlagName}.\\
%  \noindent\template{flag(@string, @term)}
%
 \item \bti{flag\_list/1}\\
 \noindent\bt{flag\_list(FlagList)} is true and \bt{FlagList} is
 the list of the flags currently defined in the engine.\\
 \noindent\template{flag\_list(-list)}
%
\item \bti{set\_prolog\_flag/2}\\
\noindent\bt{set\_prolog\_flag(Flag, Value)} is true, and as a side
effect associates \bt{Value} with the flag \bt{Flag}, where
\bt{Value} is a value that is within the implementation defined
range of values for \bt{Flag}.\\
\noindent\template{set\_prolog\_flag(+flag, @nonvar)}
%
\item \bti{get\_prolog\_flag/2}\\
\noindent\bt{get\_prolog\_flag(Flag, Value)} is true iff \bt{Flag}
is a flag supported by the engine and \bt{Value} is the value
currently associated with it. Note that \bt{get\_prolog\_flag/2} is
not re-executable.\\
\noindent\template{get\_prolog\_flag(+flag, ?term)}
%
\end{itemize}

%---------------------------------------------------------------------
\subsection{Libraries Management}
%---------------------------------------------------------------------

\begin{itemize}
%
 \item \bti{load\_library/1}\\
 \noindent\bt{load\_library(LibraryName)} is true if
 \bt{LibraryName} is the name of a \tuprolog{} library available
 for loading. As side effect, the specified library is loaded by
 the engine. Actually \bt{LibraryName} is the full name of
 the Java class providing the library.\\
 \noindent\template{load\_library(@string)}
%
 \item \bti{unload\_library/1}\\
 \noindent\bt{unload\_library(LibraryName)} is true if
 \bt{LibraryName} is the name of a library currently loaded in the
 engine. As side effect, the library is unloaded from the engine. Actually \bt{LibraryName} is the full name of
 the Java class providing the library.\\
 \noindent\template{unload\_library(@string)}
%
\end{itemize}

%---------------------------------------------------------------------
\subsection{Directives}
%---------------------------------------------------------------------

Directives are used in Prolog text only as queries to be immediately executed when loading it. When a corresponding predicate with the same procedure name as a directive exists, they perform the same actions. Their arguments will satisfy the same constraints as those required for an errorless execution of the corresponding predicate, otherwise their behaviour is undefined.

In \tuprolog{}, directives are not composable: each query must contain one and only one directive. When you need to use multiple directives, you must employ multiple queries as well.

\begin{itemize}
 %
 \item \bti{:- op/3}\\
 \noindent\bt{op(Priority, Specifier, Operator)} adds \bt{Operator}
 to the operator table, with priority (lower binds tighter)
 \bt{Priority} and associativity determined by \bt{Specifier}.\\
 \noindent\template{op(+integer, +specifier, @atom\_or\_atom\_list)}
 %
 \item \bti{:- flag/4}\\
 \noindent\bt{flag(FlagName, ValidValuesList, DefaultValue, IsModifiable)}
 adds to the engine a new flag, identified by the \bt{FlagName}
 name, which can assume only the values listed in
 \bt{ValidValuesList} with \bt{DefaultValue} as default value, and
 that can be modified if \bt{IsModifiable} is true.\\
 \noindent\template{flag(@string, @list, @term, @{true, false})}
 %
 \item \bti{:- initialization/1}\\
 \noindent\bt{initialization(Goal)} sets the starting goal to be executed just
 after the theory has been consulted.\\
 \noindent\template{initialization(@goal)}
 %
 \item \bti{:- solve/1}\\
 \noindent Synonym for \bt{initialization/1}. \emph{Deprecated.}\\
 \noindent\template{solve(@goal)}
 %
 \item \bti{:- load\_library/1}\\
 \noindent\bt{load\_library(LibraryName)} is a valid directive if true if
 \bt{LibraryName} is the name of a \tuprolog{} library available
 for loading. This directive loads the specified library in the engine.
 Actually \bt{LibraryName} is the full name of the Java class providing the library.\\
 \noindent\template{load\_library(@string)}
 %
 \item \bti{:- include/1}\\
 \noindent\bt{include(Filename)} immediately loads the theory
 contained in the file specified by \bt{Filename}.\\
 \noindent\template{include(@string)}
 %
 \item \bti{:- consult/1}\\
 \noindent Synonym for \bt{include/1}. \emph{Deprecated.}\\
 \noindent\template{consult(@string)}
 %
\end{itemize}
%

%%=======================================================================
\chapter{\tuprolog{} Libraries}
\label{ch:standard-libraries}
%=======================================================================

Libraries are the means by which \tuprolog{} achieves its
fundamental characteristics of minimality and configurability.
%
The engine is by design choice a minimal, purely-inferential core: as
such, it only includes a few \emph{built-in} predicates, intended as
predicates statically defined inside the core, to establish the
foundation which the mechanisms of the engine are based on.
%
Instead, each and every other piece of functionality, in the form of
predicates, functors, flags and operators, is delivered by libraries,
and can be added to or subtracted from the engine at any time.
%
Thus, a \tuprolog{} engine can be dynamically extended by loading
(and unloading) any number of libraries. Each library can provide a
specific set of predicates, functors and a related theory, which can
be used to define new flags and operators.
%
Besides built-in and library predicates, new functionalities can also
be added to an engine by feeding it with a user-defined Prolog theory.

Libraries can be loaded at any time in the \tuprolog{} engine, both
from the Java side, by means of the \texttt{loadLibrary} method of
the \texttt{Prolog} object representing a \tuprolog{} engine, and
from the Prolog side, using the \texttt{load\_library/1} predicate.
%
For example, suppose you want to exploit some features defined in a
library whose name is \texttt{ExampleLibrary}. If, on the Java side,
you want to load the library immediately afterwards building a
\tuprolog{} engine, you would write the following code, using the
fully qualified Java class name for the library:
%
\begin{verbatim}
Prolog engine = new Prolog();
try {
    engine.loadLibrary("com.example.ExampleLibrary");
} catch (InvalidLibraryException e) {
}
\end{verbatim}
%
If, on the other hand, you just want to load the library on the Prolog
side for those clauses which actually make use of its predicates, you
would write the following code, using just the name of the library,
which can be different from its fully qualified class name:
%
\begin{verbatim}
% println/1 is defined in ExampleLibrary
run_test(Test, Result) :- run(Test, Result),
                          load_library('ExampleLibrary'),
                          println(Result).
\end{verbatim}
%
Correspondingly, means for unloading libraries are provided, in the
form of the \texttt{unloadLibrary} method of the \texttt{Prolog}
class on the Java side, and the \texttt{unload\_library/1} predicate
on the Prolog side.
%
It must be noted that predicates for loading or unloading libraries
are also available in the form of directives: they perform the same
actions, but as directives they are immediately executed when the
Prolog text containing them is feeded to the engine.

Since the core comes as a pure inferential engine, \tuprolog{}
includes in its distribution some standard libraries which are
loaded by default into the engine at construction time. While it is
possible to create an engine with no default libraries preloaded,
those standard libraries provide the fundamental bricks of a Prolog
engine, in the form of basic functionalities, ISO compliant
predicates and evaluable functors, I/O predicates and predicates for
interoperability and integration between Java and Prolog.
%
More user-defined libraries can be then loaded or unloaded, thus
exploiting the dynamic configurability of \tuprolog{} engines which
can be reconfigured on the fly enriching or reducing the set of
available functionalities by need.

The standard libraries are:
%
\begin{description}
\item[BasicLibrary] (class \texttt{alice.tuprolog.lib.BasicLibrary}) |
  provides common Prolog predicates and functors, and operators. No
  I/O predicates are included.
%
\item[DCGLibrary] (class \texttt{alice.tuprolog.lib.DCGLibrary}) |
provides support for Definite Clause Grammar, an extension of context
free grammars used for describing natural and formal languages.
%
\item[IOLibrary] (class \texttt{alice.tuprolog.lib.IOLibrary}) |
provides some basic and classic I/O predicates.
%
\item[ISOLibrary] (class \texttt{alice.tuprolog.lib.ISOLibrary}) |
provides predicates and functors that are part of the built-in
section in the ISO standard \cite{iso95}, and are not provided by
previous libraries.
%
\item[JavaLibrary] (class \texttt{alice.tuprolog.lib.JavaLibrary}) |
provides predicates and functors to create, access and deploy
(existent or new) Java resources, like objects and classes.
%
\end{description}
%
\noindent The description of each library is provided by discussing in
the order: predicates, functors, operators and flags defined by the
library.
%
For each library the dependencies with other libraries are specified:
%
that is, which other libraries are required in order to provide the
correct computational behaviour.
%

%-----------------------------------------------------------------------
\section{BasicLibrary}
%-----------------------------------------------------------------------

\noindent \emph{Library Dependencies}: none.

This library provides common Prolog built-in predicates,
functors, and operators. No I/O predicates are included.

Please note that in the following \texttt{string} means a single or
double quoted string, as detailed in Chapter \ref{ch:engine};
\texttt{expr} means an evaluable expression, that is a term that can
be interpreted as a value by some library functors.

%---------------------------------------------------------------------
\subsection{Predicates}
%---------------------------------------------------------------------

\noindent Here follows a list of predicates implemented by this
library, grouped by category.

%---------------------------------------------------------------------
\subsubsection{Type Testing}
%---------------------------------------------------------------------
%
\begin{itemize}
    \item \bti{constant/1}\\
    \noindent\bt{constant(X)} is true iff \bt{X} is a constant value.\\
    \template{constant(@term)}
    %
    \item \bti{number/1}\\
    \noindent\bt{number(X)} is true iff \bt{X} is an integer or a float.\\
    \template{number(@term)}
    %
    \item \bti{integer/1}\\
    \noindent\bt{integer(X)} is true iff \bt{X} is an integer.\\
    \template{integer(@term)}
    %
    \item \bti{float/1}\\
    \noindent\bt{float(X)} is true iff \bt{X} is an float.\\
    \template{float(@term)}
    %
    \item \bti{atom/1}\\
    \noindent\bt{atom(X)} is true iff \bt{X} is an atom.\\
    \template{atom(@term)}
    %
    \item \bti{compound/1}\\
    \noindent\bt{compound(X)} is true iff \bt{X} is a compound term,
    that is neither atomic nor a variable.\\
    \template{compound(@term)}
    %
    \item \bti{var/1}\\
    \noindent\bt{var(X)} is true iff \bt{X} is a variable.\\
    \template{var(@term)}
    %
    \item \bti{nonvar/1}\\
    \noindent\bt{nonvar(X)} is true iff \bt{X} is not a variable.\\
    \template{nonvar(@term)}
    %
    \item \bti{atomic/1}\\
    \noindent\bt{atomic(X)} is true iff \bt{X} is atomic (that is is an atom, an integer
    or a float).\\
    \template{atomic(@term)}
    %
    \item \bti{ground/1}\\
    \noindent\bt{ground(X)} is true iff \bt{X} is a ground term.\\
    \template{ground(@term)}
    %
    \item \bti{list/1}\\
    \noindent\bt{list(X)} is true iff \bt{X} is a list.\\
    \template{list(@term)}
    %
\end{itemize}

%---------------------------------------------------------------------
\subsubsection{Term Creation, Decomposition and Unification}
%---------------------------------------------------------------------
%
\begin{itemize}
%
\item \verb|'=..'/2| : \textit{univ}\\
\noindent\verb|'=..'(Term, List)| is true if \bt{List} is a list
consisting of the functor and all arguments of \bt{Term}, in
order. \\
\template{'=..'(?term, ?list)}
%
\item \bti{functor/3}\\
\noindent\bt{functor(Term, Functor, Arity)} is true if the term
\bt{Term} is a compound term, \bt{Functor} is its functor, and
\bt{Arity} (an integer) is its arity; or if \bt{Term} is an atom
or number equal to \bt{Functor} and \bt{Arity} is 0.\\
\template{functor(?term, ?term, ?integer)}
%
\item \bti{arg/3}\\
\noindent\bt{arg(N, Term, Arg)} is true if \bt{Arg} is the \bt{N}th
arguments of \bt{Term} (counting from 1).\\
\template{arg(@integer, @compound, -term)}
%
\item \bti{text\_term/2}\\
\noindent\bt{text\_term(Text, Term)} is true iff \bt{Text} is the
text representation of the term \bt{Term}.\\
\template{text\_term(?text, ?term)}
%
\item \bti{text\_concat/3}\\
\noindent\bt{text\_concat(TextSource1, TextSource2, TextDest)} is
true iff \bt{TextDest} is the text resulting by appending the text
\bt{TestSource2} to \bt{TextSource1}\.\\
\template{text\_concat(@string, @string, -string)}
%
\item \bti{num\_atom/2}\\
\noindent\bt{num\_atom(Number, Atom)} succeeds iff \bt{Atom}
is the atom representation of the number \bt{Number}\\
\template{number\_codes(+number, ?atom)}\\
\template{number\_codes(?number, +atom)}
%
\end{itemize}

%---------------------------------------------------------------------
\subsubsection{Occurs Check}
%---------------------------------------------------------------------

\noindent When the process of unification takes place between a
variable $S$ and a term $T$, the first thing a Prolog engine should do
before proceeding is to check that $T$ does not contain any occurences
of $S$. This test is known as \emph{occurs check} \cite{ss94} and is
necessary to prevent the unification of terms such as $s(X)$ and $X$,
for which no finite common instance exists. Most Prolog
implementations omit the occurs check from their unification algorithm
for reasons related to speed and efficiency: \tuprolog{} is no
exception. However, they provide a predicate for occurs check
augmented unification, to be used when the programmer wants to never
incur on an error or an undefined result during the process.
%
\begin{itemize}
%
\item \bti{unify\_with\_occurs\_check/2}\\
\noindent\bt{unify\_with\_occurs\_check(X, Y)} is true iff \bt{X}
and \bt{Y} are unifiable.\\
\noindent\template{unify\_with\_occurs\_check(?term, ?term)}
%
\end{itemize}
%
%---------------------------------------------------------------------
\subsubsection{Expression and Term Comparison}
%---------------------------------------------------------------------
\begin{itemize}
%
    \item expression comparison (generic template:
    \emph{pred}(@expr, @expr)):\\
        \verb|'=:=', '=\=', '>', '<', '>=', '=<'|;
    %
    \item term comparison (generic template:
    \emph{pred}(@term, @term)):\\
         \verb|'==', '\==', '@>', '@<', '@>=', '@=<'|.

\end{itemize}

%---------------------------------------------------------------------
\subsubsection{Finding Solutions}
%---------------------------------------------------------------------
\begin{itemize}
%
\item \bti{findall/3}\\
\noindent\bt{findall(Template, Goal, List)} is true if and only if
\bt{List} unifies with the list of values to which a variable X not
occurring in \bt{Template} or \bt{Goal} would be instantiated
by successive re-executions of\\
\bt{call(Goal), X = Template}\\
\noindent after systematic replacement of all variables in X by
new variables.\\
\template{\bt{findall(?term, +callable\_term, ?list)}}
%
\item \bti{bagof/3}\\
\noindent\bt{bagof(Template, Goal, Instances)} is true if
\bt{Instances} is a non-empty list of all terms such that each
unifies with \bt{Template} for a fixed instance W of the variables
of \bt{Goal} that are free with respect to \bt{Template}. The
ordering of the elements of \bt{Instances} is the order in which
the solutions are found.\\
\template{bagof(?term, +callable\_term, ?list)}
%
\item \bti{setof/3}\\
\noindent\bt{setof(Template, Goal, List)} is true if \bt{List} is a
sorted non-empty list of all terms that each unifies with
\bt{Template} for a fixed instance W of the variables of \bt{Goal}
that are free with respect to \bt{Template}.\\
\template{\bt{setof(?term, +callable\_term, ?list)}}
%
\end{itemize}

%---------------------------------------------------------------------
\subsubsection{Control Management}
%---------------------------------------------------------------------
\begin{itemize}
%
\item \bti{(->)/2} : \textit{if-then}\\
\noindent\verb|'->'(If, Then)| is true if and only if \bt{If} is true
and \bt{Then} is true for the first solution of \bt{If}.
%
\item \bti{(;)/2} : \textit{if-then-else}\\
\noindent\verb|';'(Either, Or)| is true iff either \bt{Either} or
\bt{Or} is true.
%
\item \bti{call/1}\\
\noindent\bt{call(Goal)} is true if and only if \bt{Goal}
represents a goal which is true. It is opaque to cut.\\
\template{call(+callable\_term)}
%
\item \bti{once/1}\\
\noindent\bt{once(Goal)} finds exactly one solution to \bt{Goal}.
It is equivalent to \bt{call((Goal, !))} and is opaque to cuts.\\
\template{once(@goal)}
%
\item \bti{repeat/0}\\
Whenever backtracking reaches \noindent\bt{repeat}, execution
proceeds forward again through the same clauses as if
another alternative has been found.\\
\template{repeat}
%
\item \verb|'\+'/1| : \textit{not provable}\\
\noindent\verb|'\+'(Goal)| is the negation predicate and is
opaque to cuts. That is, \verb|'\+'(Goal)| is like
\bt{call(Goal)} except that its success or failure is the opposite.\\
\template{'$\setminus$+'(@goal)}
%
\item \bti{not/1}\\
\noindent The predicate \bt{not/1} has the same semantics and
implementation as the predicate \verb|'\+'/1|.\\
\template{not(@goal)}
%
\end{itemize}

%---------------------------------------------------------------------
\subsubsection{Clause Retrival, Creation and Destruction}
%---------------------------------------------------------------------

\noindent Every Prolog engine lets programmers modify its logic
database during execution by adding or deleting specific clauses. The
ISO standard \cite{iso95} distinguishes between static and dynamic
predicates: only the latter can be modified by asserting or retracting
clauses. While typically the \emph{dynamic/1} directive is used to
indicate whenever a user-defined predicate is dynamically modifiable,
\tuprolog{} engines work differently, establishing two default
behaviors: library predicates are always of a static kind; every other
user-defined predicate is dynamic and modifiable at runtime.
%
The following list contains library predicates used to manipulate the
knowledge base of a \tuprolog{} engine during execution.

\begin{itemize}
%
\item \bti{clause/2}\\
\noindent\bt{clause(Head, Body)} is true iff \bt{Head} matches the
head of a dynamic predicate, and \bt{Body} matches its body. The
body of a fact is considered to be \bt{true}. \bt{Head} must be at
least partly instantiated.\\
\template{\bt{clause(@term, -term)}}
%
\item \bti{assert/1}\\
\noindent\bt{assert(Clause)} is true and adds \bt{Clause} to the
end of the database.\\
\template{\bt{assert(@term)}}
%
\item \bti{retract/1}\\
\noindent\bt{retract(Clause)} removes from the knowledge base a
dynamic clause that matches \texttt{Clause} (which must be at least
partially instantiated). Gives multiple solutions upon backtracking.\\
\template{\bt{retract(@term)}}
%
\item \bti{retractall/1}\\
\noindent\bt{retractall(Clause)} removes from the knowledge base all
the dynamic clauses matching with \texttt{Clause} (which must be at
least partially instantiated).\\
\template{\bt{retractall(@term)}}
%
\end{itemize}

%---------------------------------------------------------------------
\subsubsection{Operator Management}
%---------------------------------------------------------------------
\begin{itemize}
%
\item \bti{current\_op/3}\\
\noindent\bt{current\_op(Priority, Type, Name)} is true iff
\bt{Priority} is an integer in the range [0, 1200], \bt{Type} is
one of the \bt{fx}, \bt{xfy}, \bt{yfx}, \bt{xfx} values and
\bt{Name} is an atom, and as side effect it adds a new operator to
the engine operator list.\\
\template{current\_op(?integer, ?term, ?atom)}
%
\end{itemize}

%---------------------------------------------------------------------
\subsubsection{Flag Management}
%---------------------------------------------------------------------
\begin{itemize}
%
\item \bti{current\_prolog\_flag/3}\\
\noindent\bt{current\_prolog\_flag(Flag,Value)} is true if the
value of the flag \bt{Flag}
is \bt{Value}\\
\template{current\_prolog\_flag(?atom,?term)}
%
\end{itemize}

%---------------------------------------------------------------------
\subsubsection{Actions on Theories and Engines}
%---------------------------------------------------------------------
\begin{itemize}
%
%
\item \bti{set\_theory/1}\\
\noindent\bt{set\_theory(TheoryText)} is true iff \bt{TheoryText}
is the text representation of a valid \tuprolog{} theory, with the
side effect of setting it as the new theory of the engine.\\
\template{set\_theory(@string)}
%
\item \bti{add\_theory/1}\\
\noindent\bt{add\_theory(TheoryText)} is true iff \bt{TheoryText}
is the text representation of a valid \tuprolog{} theory, with the
side effect of appending it to the current theory of the engine.\\
\template{add\_theory(@string)}
%
\item \bti{get\_theory/1}\\
\noindent\bt{get\_theory(TheoryText)} is true, and
\bt{TheoryText} is the text representation of the current theory of the engine.\\
\template{get\_theory(-string)}
%
\item \bti{agent/1}\\
\noindent\bt{agent(TheoryText)} is true, and spawns a
\tuprolog{} agent with the knowledge base provided as a Prolog
textual form in \texttt{TheoryText} (the goal is described in the
knowledge base).\\
\template{agent(@string)}
%
\item \bti{agent/2}\\
\noindent\bt{agent(TheoryText, Goal)} is true, and spawn a
\tuprolog{} agent with the knowledge base provided as a Prolog
textual form in \texttt{TheoryText}, and solving the query
\texttt{Goal}
as a goal.\\
\template{agent(@string, @term)}
%
\end{itemize}
%
%---------------------------------------------------------------------
\subsubsection{Spy Events}
%---------------------------------------------------------------------
%
During each demonstration, the engine notifies to interested listeners so-called
{\em spy events}, containing informations on its internal state, such as the
current subgoal being evaluated, the configuration of the execution stack and
the available choice points. The different kinds of spy events currently
corresponds to the different states which the virtual machine realizing the
\tuprolog{}'s inferential core can be found into. \textit{Init} events are
spawned whenever the machine initialize a subgoal for execution; \textit{Call}
events are generated when a choice must be made for the next subgoal to be
executed; \textit{Eval} events represent actual subgoal evaluation; finally,
\textit{Back} events are notified when a backtracking occurs during the
demonstration process.
%
\begin{itemize}
%
\item \bti{spy/0}\\
\noindent\bt{spy} is true and enables the notification of spy
events occurring inside the engine.\\
\template{spy}
%
\item \bti{nospy/0}\\
\noindent\bt{nospy} is true and disables the notification of the
spy events inside the engine.\\
\template{nospy}
%
\end{itemize}

%---------------------------------------------------------------------
\subsubsection{Auxiliary predicates}
%---------------------------------------------------------------------

\noindent The following predicates are provided by the library's theory.

\begin{itemize}
%
\item \bti{member/2}\\
\noindent\bt{member(Element, List)} is true iff \bt{Element} is an
element of the list
\bt{List}\\
\template{member(?term, +list)}
%
\item \bti{length/2}\\
\noindent\bt{length(List, NumberOfElements)} is true in three
different cases: (1) if \bt{List} is instantiated to a list of
determinate length, then \bt{Length} will be unified with this
length; (2) if \bt{List} is of indeterminate length and \bt{Length}
is instantiated to an integer, then \bt{List} will be unified with a
list of length \bt{Length} and in such a case the list elements are
unique variables; (3) if \bt{Length} is unbound then \bt{Length}
will be unified with all possible lengths of \bt{List}.\\
\template{member(?list, ?integer)}
%
\item \bti{append/3}\\
\noindent\bt{append(What, To, Target)} is true iff \bt{Target} list
can be obtained by appending the \bt{To} list to the \bt{What}
list \\
\template{append(?list, ?list, ?list)}
%
\item \bti{reverse/2}\\
\noindent\bt{reverse(List, ReversedList)} is true iff
\bti{ReversedList} is the reverse list of \bt{List}\\
\template{reverse(+list, -list)}
%
\item \bti{delete/3}\\
\noindent\bt{delete(Element, ListSource, ListDest)} is true iff
\bt{ListDest} list can be obtained by removing the element
\bt{Element} from the list \bt{ListSource}.\\
\template{delete(@term, +list, -list)}
%
\item \bti{element/3}\\
\noindent\bt{element(Position, List, Element)} is true iff
\bt{Element} is the \bt{Position}th element of the list \bt{List}
(starting the count from 1).\\
\template{element(@integer, +list, -term)}
%
\item \bti{quicksort/3}\\
\noindent\bt{quicksort(List, ComparisonPredicate, SortedList)} is
true iff \bt{SortedList} is the list \bt{List} sorted by the comparison
predicate \bt{ComparisonPredicate}.\\
\template{element(@list, @pred, -list)}
%
\end{itemize}

%---------------------------------------------------------------------
\subsection{Functors}
%---------------------------------------------------------------------

Functors for expression evaluation (with usual semantics):
\begin{itemize}
    \item unary:  \verb|+, -, ~, +|
    \item binary:  \verb|+, -, *, \, **, <<, >>, /\, \/|
\end{itemize}

%---------------------------------------------------------------------
\subsection{Operators}
%---------------------------------------------------------------------

\begin{table}[h]
    %
    \begin{center}{\small\tt
    \begin{tabular}{p{2cm}|p{1cm}|p{1cm}}\hline\hline
    Name & Assoc. & Prio. \\ \hline\hline
    :-      &   fx  &   1200 \\
    :-      &   xfx &   1200 \\
    ?-      &   fx  &   1200 \\
    ;       &   xfy &   1100 \\
    ->      &   xfy &   1050 \\
    ,       &   xfy &   1000 \\
    not     &   fy  &   900 \\
    $\setminus$+   &   fy   & 900   \\
    =       &   xfx &   700 \\
    $\setminus$=    &  xfx  &   700 \\
    ==      &   xfx &   700 \\
    $\setminus$==   &  xfx  &   700 \\
    @>      &   xfx & 700   \\
    @<      &   xfx & 700   \\
    @=<    &   xfx & 700   \\
    @>=    &   xfx & 700   \\
    =:=    &   xfx & 700   \\
    =$\setminus$=   &   xfx & 700   \\
    >      &   xfx & 700   \\
    <      &   xfx & 700   \\
    >=      &   xfx & 700   \\
    =<      &   xfx & 700   \\
    is      &   xfx &   700 \\
    =..     &   xfx & 700 \\
    +       &   yfx & 500 \\
    -       &   yfx & 500 \\
    $/\setminus$    &   yfx &   500 \\
    $\setminus/$    &   yfx &   500 \\
    $\ast$  &   yfx & 400 \\
    /       &   yfx & 400 \\
    //      &   yfx & 400 \\
    >>      &   yfx & 400 \\
    <<      &   yfx & 400 \\
    >>      &   yfx & 400 \\
    $\ast$$\ast$  &   xfx & 200 \\
    \textasciicircum  &   xfy & 200 \\
    $\setminus$$\setminus$      &   fx & 200 \\
    -       &   fy & 200 \\
    \hline\hline
    \end{tabular}
    }\end{center}
\end{table}

\clearpage

%-----------------------------------------------------------------------
\section{ISOLibrary}
%-----------------------------------------------------------------------

\noindent \emph{Library Dependencies}: BasicLibrary.

This library contains almost\footnote{Currently ISO exceptions, ISO
I/O predicates and some ISO directives are not supported.} all the
built-in predicates and functors that are part of the ISO standard
and that are not part directly of the \tuprolog{} core engine or
other core libraries.
%
Moreover, some features are added, not currently ISO, such as the
support for definite clause grammars (DCGs).
%

%---------------------------------------------------------------------
\subsection{Predicates}
%---------------------------------------------------------------------

\noindent Here follows a list of predicates implemented by this
library, grouped by category.

%---------------------------------------------------------------------
%\subsubsection{Clause Retrieval and Information}
%---------------------------------------------------------------------

%\begin{itemize}
%
%\item \bti{current\_predicate/1}\\
%\noindent\bt{current\_predicate(Functor/Arity)} -- \emph{to be implemented}\\
%\template{\bt{current\_predicate(?term)}}
%
%\end{itemize}

%---------------------------------------------------------------------
\subsubsection{Type Testing}
%---------------------------------------------------------------------

\begin{itemize}
%
\item \bti{bound/1}\\
\noindent\bt{bound(Term)} is a synonym for the \bt{ground/1} predicate
defined in BasicLibrary.\\
\template{bound(+term)}
%
\item \bti{unbound/1}\\
\noindent\bt{unbound(Term)} is true iff \bt{Term} is not a ground
term.\\
\template{unbound(+term)}
%
\end{itemize}

%---------------------------------------------------------------------
\subsubsection{Atoms Processing}
%---------------------------------------------------------------------

\begin{itemize}
%
\item \bti{atom\_length/2}\\
\noindent\bt{atom\_length(Atom, Length)} is true iff the integer
\bt{Length} equals the number of characters in the name of atom
\bt{Atom}.\\
\template{atom\_length(+atom, ?integer)}
%
\item \bti{atom\_concat/3}\\
\noindent\bt{atom\_concat(Start, End, Whole)} is true iff the
\bt{Whole} is the atom obtained by concatenating the characters of
\bt{End} to those of \bt{Start}. If \bt{Whole} is instantiated, then
all decompositions of \bt{Whole} can be obtained by backtracking.\\
\template{atom\_concat(?atom, ?atom, +atom)}\\
\template{atom\_concat(+atom, +atom, -atom)}
%
\item \bti{sub\_atom/5}\\
\noindent\bt{sub\_atom(Atom, Before, Length, After, SubAtom)} is
true iff \bt{SubAtom} is the sub atom of \bt{Atom} of length
\bt{Length} that appears with \bt{Before} characters preceding it
and \bt{After} characters following. It is re-executable.\\
\template{sub\_atom(+atom, ?integer, ?integer, ?integer, ?atom)}
%
\item \bti{atom\_chars/2}\\
\noindent\bt{atom\_chars(Atom,List)} succeeds iff \bt{List} is a
list whose elements are the one character atoms that in order make
up \bt{Atom}.\\
\template{atom\_chars(+atom, ?character\_list)}\\
\template{atom\_chars(-atom, ?character\_list)}
%
\item \bti{atom\_codes/2}\\
\noindent\bt{atom\_codes(Atom, List)} succeeds iff \bt{List} is a
list whose elements are the character codes that in order correspond
to the characters that make up \bt{Atom}.\\
\template{atom\_codes(+atom, ?character\_code\_list)}\\
\template{atom\_chars(-atom, ?character\_code\_list)}
%
\item \bti{char\_code/2}\\
\noindent\bt{char\_code(Char, Code)} succeeds iff \bt{Code} is a
the character code that corresponds to the character \bt{Char}.\\
\template{char\_code(+character, ?character\_code)}\\
\template{char\_code(-character, +character\_code)}
%
\item \bti{number\_chars/2}\\
\noindent\bt{number\_chars(Number, List)} succeeds iff \bt{List}
is a list whose elements are the one character atoms that in
order make up \bt{Number}.\\
\template{number\_chars(+number, ?character\_list)}\\
\template{number\_chars(-number, ?character\_list)}
%
\item \bti{number\_codes/2}\\
\noindent\bt{number\_codes(Number, List)} succeeds iff \bt{List}
is a list whose elements are the codes for the one character atoms
that in order make up \bt{Number}.\\
\template{number\_codes(+number,?character\_code\_list)}\\
\template{number\_codes(-number,?character\_code\_list)}
%
\end{itemize}

%---------------------------------------------------------------------
\subsection{Functors}
%---------------------------------------------------------------------

\begin{itemize}
    \item Trigonometric functions: \bt{sin(+expr)}, \bt{cos(+expr)}, \bt{atan(+expr)}.
    %
    \item Logarithmic functions: \bt{exp(+expr)}, \bt{log(+expr)}, \bt{sqrt(+expr)}.
    %
    \item Absolute value functions: \bt{abs(+expr)}, \bt{sign(+Expr)}.
    %
    \item Rounding functions: \bt{floor(+expr)},
    \bt{ceiling(+expr)}, \bt{round(+expr)}, \bt{truncate(+expr)},
    \bt{float(+expr)}, \bt{float\_integer\_part(+expr)},\\\bt{float\_fractional\_part(+expr)}.
    %
    \item Integer division functions:
    \bt{div(+expr, +expr)}, \bt{mod(+expr, +expr)}, \bt{rem(+expr, +expr)}.
\end{itemize}

%---------------------------------------------------------------------
\subsection{Operators}
%---------------------------------------------------------------------

\begin{table}[h]
    %
    \begin{center}{\small\tt
    \begin{tabular}{p{2cm}|p{1cm}|p{1cm}}\hline\hline
    Name & Assoc. & Prio. \\ \hline
    mod   & yfx & 400\\
    div   & yfx & 300\\
    rem   & yfx & 300\\
    sin   & fx & 200\\
    cos   & fx & 200\\
    sqrt  & fx & 200\\
    atan  & fx & 200\\
    exp   & fx & 200\\
    log   & fx & 200\\
    \hline\hline
    \end{tabular}
    }\end{center}
\end{table}

%---------------------------------------------------------------------
\subsection{Flags}
%---------------------------------------------------------------------

\begin{table}[h]
    %
    \begin{center}{\small\tt
    \begin{tabular}{p{6cm}|p{3cm}|p{3cm}}\hline\hline
        Flag Name   & Possible Values & Default Value\\ \hline\hline
        bounded         & {true}           &  true \\
        max\_integer     & {2147483647}     &  2147483647 \\
        min\_integer     & {-2147483648}    &  -2147483648 \\
        integer\_rounding\_function & {down} & down \\
        char\_conversion & {off}           & off \\
        debug           & {off}           & off \\
        max\_arity       & {2147483647}    & 2147483647 \\
        undefined\_predicates & {fail}         & fail \\
        double\_quotes & {atom}         & atom \\
    \hline\hline
    \end{tabular}
    }\end{center}
\end{table}

\clearpage

%---------------------------------------------------------------------
\section{DCGLibrary}
%---------------------------------------------------------------------

\noindent \emph{Library Dependencies}: BasicLibrary.

This library provides support for Definite Clause Grammar
\cite{bra00}, also known as DCG,\footnote{The DCG formalism is not
defined as an ISO standard at the time of writing this document.} an
extension of context free grammars that have proven useful for
describing natural and formal languages, and that may be
conveniently expressed and executed in Prolog.
%
Note that this library is not loaded by default when a \tuprolog{}
engine is created.

A Definite Clause Grammar rule has the general form:\\\\
%
\begin{verbatim}
Head --> Body
\end{verbatim}
%
with the declarative interpretation that a possible form for \texttt{Head}
is \texttt{Body}.
%
A non-terminal symbol may be any term other than a variable or a
number.
%
A terminal symbol may be any term. In order to distinguish
terminals from nonterminals, a sequence of one or more terminal
symbols  is written within a grammar rule as a Prolog list, with the
empty sequence written as the empty list \verb|[]|.
%
The body can contain also executable blocks -- interpreted
according to normal Prolog rule -- enclosed by the \verb|{| and
\verb|}| parenthesis.
%
A simple example of DCG follows:
%
\begin{verbatim}
sentence --> noun_phrase, verb_phrase.
verb_phrase --> verb, noun_phrase.
noun_phrase --> [charles].
noun_phrase --> [linda].
verb --> [loves].
\end{verbatim}
%
So, you can verify that a phrase is correct according to
the grammar simply by the query:
%
\begin{verbatim}
?- phrase(sentence, [charles, loves, linda]).
\end{verbatim}
%
But also:
%
\begin{verbatim}
?- phrase(sentence, [Who, loves, linda]).
\end{verbatim}
%
which would give, according to the grammar, two solutions,
\texttt{Who} bound to \texttt{charles}, and \texttt{Who} bound to
\texttt{linda}.

%---------------------------------------------------------------------
\subsection{Predicates}
%---------------------------------------------------------------------

\noindent The classic built-in predicates provided for parsing DCG
sentences are:

\begin{itemize}
%
\item \bti{phrase/2}\\
\noindent\bt{phrase(Category, List)} is true iff the list \bt{List}
can be parsed as a phrase (i.e. sequence of terminals) of type
\bt{Category}.
%
\bt{Category} can be any term which would be accepted as a
nonterminal of the grammar (or in general, it can be any grammar
rule body), and must be instantiated to a non-variable term at the
time of the call.
%
This predicate is the usual way to commence execution of grammar
rules.
%
If \bt{List} is bound to a list of terminals by the time of the
call, then the goal corresponds to parsing \bt{List} as a phrase
of type \bt{Category}; otherwise if \bt{List} is unbound, then the
grammar is being used for generation.\\
%
\template{phrase(+term, ?list)}
%
%
\item \bti{phrase/3}\\
\noindent\bt{phrase(Category, List, Rest)} is true iff the segment
between the start of list \bt{List} and the start of list \bt{Rest}
can be parsed as a phrase (i.e. sequence of terminals) of type
\bt{Category}.
%
In other words, if the search for phrase Phrase is started at the
beginning of list \bt{List}, then \bt{Rest} is what remains
unparsed after \bt{Category} has been found.
%
Again, \bt{Category} can be any term which would be accepted as a
nonterminal of the grammar (or in general, any grammar rule body),
and must be instantiated to a non variable term at the time
of the call.\\
%
\template{phrase(+term, ?list, ?rest)}

\end{itemize}

%---------------------------------------------------------------------
\subsection{Operators}
%---------------------------------------------------------------------
\mbox{} % Thanks stupid LaTeX for putting the table below where you want
        % instead of where I say.
\begin{table}[!h]
    \begin{center}{\small\tt
    \begin{tabular}{p{2cm}|p{1cm}|p{1cm}}\hline\hline
    Name & Assoc. & Prio. \\ \hline
    --> & xfx & 1200\\
    \hline\hline
    \end{tabular}
    }\end{center}
\end{table}

%-----------------------------------------------------------------------
\section{IOLibrary}
%-----------------------------------------------------------------------

\noindent \emph{Library Dependencies}: BasicLibrary.

The IOLibrary defines classic Prolog built-ins predicates to enable
interaction between Prolog programs and external resources, typically
files and I/O channels.

%---------------------------------------------------------------------
\subsection{Predicates}
%---------------------------------------------------------------------

\noindent Here follows a list of predicates implemented by this
library, grouped by category.

%---------------------------------------------------------------------
\subsubsection{General I/O}
%---------------------------------------------------------------------

\begin{itemize}

\item \bti{see/1}\\
\noindent\bt{see(StreamName)} is used to create/open an input
stream; the predicate is true iff \bt{StreamName} is a string
representing the name of a file to be created or accessed as input
stream, or the string \texttt{stdin} selecting current standard
input as input stream.\\
\template{see(@atom)}
%
\item \bti{seen/0}\\
\noindent\bt{seen} is used to close the input stream previously
opened; the predicate is true iff the closing action is possible.\\
\template{seen}
%
\item \bti{seeing/1}\\
\noindent\bt{seeing(StreamName)} is true iff \texttt{StreamName}
is the name of the stream currently used as input stream.\\
\template{seeing(?term)}
%
\item \bti{tell/1}\\
\noindent\bt{tell(StreamName)} is used to create/open an output
stream; the predicate is true iff \bt{StreamName} is a string
representing the name of a file to be created or accessed as
output stream, or the string \texttt{stdout} selecting current
standard output as output stream.\\
\template{tell(@atom)}
%
\item \bti{told/0}\\
\noindent\bt{told} is used to close the output stream previously
opened; the predicate is true iff the closing action is possible.\\
\template{told}
%
\item \bti{telling/1}\\
\noindent\bt{telling(StreamName)} is true iff \texttt{StreamName}
is the name of the stream currently used as input stream.\\
\template{telling(?term)}
%
\item \bti{put/1}\\
\noindent\bt{put(Char)} puts the character \bt{Char} on
current output stream; it is true iff the operation is possible.\\
\template{put(@char)}
%
\item \bti{get0/1}\\
\noindent\bt{get0(Value)} is true iff \bt{Value} is the next
character (whose code can span on the entire ASCII codes)
available from the input stream, or -1 if no characters are
available;
%
as a side effect the character is removed from the input stream.\\
%
\template{get0(?charOrMinusOne)}
%
\item \bti{get/1}\\
\noindent\bt{get(Value)} is true iff \bt{Value} is the next
character (whose code can span on the range 32..255 as ASCII
codes) available from the input stream, or -1 if no characters are
available;
%
as a side effect the character (with all the characters that
precede this one not in the range 32..255) is removed from the
input stream.\\
%
\template{get(?charOrMinusOne)}
%
\item \bti{tab/1}\\
\noindent\bt{tab(NumSpaces)} inserts \bt{NumSpaces} space
characters (ASCII code 32) on output stream; the predicate is true
iff the operation is possible.\\
%
\template{tab(+integer)}
%
%
\item \bti{read/1}\\
\noindent\bt{read(Term)} is true iff \bt{Term} is Prolog term
available from the input stream.
%
The term must ends with the \emph{.} character; if no valid terms
are available, the predicate fails.
%
As a side effect, the term is removed from the input stream.\\
%
\template{read(?term)}
%
%
\item \bti{write/1}\\
\noindent\bt{write(Term)} writes the term \bt{Term} on current
output stream.
%
The predicate fails if the operation is not possible.\\
%
\template{write(@term)}
%
%
\item \bti{print/1}\\
\noindent\bt{print(Term)} writes the term \bt{Term} on current
output stream, removing apices if the term is an atom representing
a string.
%
The predicate fails if the operation is not possible.\\
%
\template{print(@term)}
%
\item \bti{nl/0}\\
\noindent\bt{nl} writes a new line control character on current
output stream.
%
The predicate fails if the operation is not possible.\\
\template{nl}
%
\end{itemize}

%---------------------------------------------------------------------
\subsubsection{I/O and Theories Helpers}
%---------------------------------------------------------------------
%
\begin{itemize}
%
\item \bti{text\_from\_file/2}\\
\noindent\bt{text\_from\_file(File, Text)} is true iff \bt{Text} is
the text contained in the file whose name is \texttt{File}.\\
\template{text\_from\_file(+string, -string)}
%
%
\item \bti{agent\_file/1}\\
\noindent\bt{agent\_file(TheoryFileName)} is true iff
\texttt{TheoryFileName} is an accessible file containing a Prolog
knowledge base, and as a side effect it spawns a \tuprolog{} agent
provided with that knowledge base.\\
\template{agent\_file(+string)}
%
%
\item \bti{solve\_file/2}\\
\noindent\bt{solve\_file(TheoryFileName, Goal)} is true iff
\texttt{TheoryFileName} is an accessible file containing a Prolog
knowledge base, and as a side effect it solves the query \texttt{Goal}
according to that knowledge base.\\
\template{solve\_file(+string, +goal)}
%
%
\item \bti{consult/1}\\
\noindent\bt{consult(TheoryFileName)} is true iff
\texttt{TheoryFileName} is an accessible file containing a Prolog
knowledge base, and as a side effect it consult that knowledge base,
by adding it to current knowledge base.\\
\template{consult(+string)}
%
\end{itemize}

%---------------------------------------------------------------------
\subsubsection{Random Generation of Numbers}
%---------------------------------------------------------------------

\noindent The random generation of number can be regarded as a form of
I/O.

\begin{itemize}
%
\item \bti{rand\_float/1}\\
\noindent\bt{rand\_float(RandomFloat)} is true iff
\texttt{RandomFloat} is a float random number generated by the
engine between 0 and 1.\\
\template{rand\_float(?float)}
%
\item \bti{rand\_int/2}\\
\noindent\bt{rand\_int(Seed, RandomInteger)} is true iff
\texttt{RandomInteger} is an integer random number generated by
the engine between 0 and \texttt{Seed}.\\
\template{rand\_int(?integer, @integer)}
%
\end{itemize}

%%=======================================================================
\chapter{Accessing Java from \tuprolog{}}
\label{java-library}
%=======================================================================
One of the main advantages of \tuprolog{} open architecture is
that any Java component can be directly accessed and used from
Prolog, in a simple and effective way, by means of the
\texttt{JavaLibrary} library: this delivers all the power of
existing Java components and packages to \tuprolog{} sources.
%
In this way, all Java packages involving interaction (such as Swing,
JDBC, the socket package, RMI) are immediately available to increase
the interaction abilities of \tuprolog:
%
{``one library for all libraries''} is the basic motto.
%
%
\section{Mapping data structures}

Complete bi-directional mapping is provided between Java primitive
types and \tuprolog{} data types.
%
By default, \tuprolog{} integers are mapped into Java \texttt{int}
or \texttt{long} as appropriate, while \texttt{byte} and
\texttt{short} types are mapped into \tuprolog{}'s \texttt{Int}
instances. Only Java \texttt{double} numbers are used to map
\tuprolog{} reals, but \texttt{float} values returned as result of
method invocations or field accesses are handled properly anyway,
without any loss of information.
%
Boolean Java values are mapped into specific \tuprolog{}
\texttt{Term} constants.
%
Java \texttt{char}s are mapped into Prolog atoms, but atoms are
mapped into Java \texttt{String}s by default.
%
The \emph{any} variable (\_) can be used to specify the Java
\texttt{null} value.

%---------------------------------------------------------------
\section{General predicates description}
%---------------------------------------------------------------

\begin{figure}
\caption{A sample Java class (a counter) used to explain JavaLibrary predicates behaviour.
\labelfig{jreflect-example}}
\begin{verbatim}
public class Counter {
    public String name;
    private long value = 0;

    public Counter() {}
    public Counter(String aName) { name = aName; }

    public void setValue(long val) { value=val; }
    public long getValue() { return value; }
    public void inc() { value++; }

    static public String getVersion() { return "1.0"; }
}
\end{verbatim}
\end{figure}

The library offers the following predicates:
%
\begin{enumerate}
  \renewcommand\labelenumi{\it(\roman{enumi})}
  %
  \item the \texttt{java\_object/3} predicate is used to create a new Java
        object of the specified class, according to the syntax:
        %
        \begin{center}
        \texttt{java\_object(\textit{ClassName},
                             \textit{ArgumentList},
                             \textit{ObjectRef})}
        \end{center}
        %
        \texttt{\textit{ClassName}} is a Prolog atom bound to the name of the
        proper Java class (e.g. \verb|'Counter'|, \verb|'java.io.FileInputStream'|),
        while the parameter \texttt{\textit{ArgumentList}} is a Prolog list used to supply
        the required arguments to the class
        constructor: the empty list matches the default constructor.
        %
        %%%%%% RICCI 020202
        Also Java arrays can be instantiated, by appending
        \texttt{[]} at the end of the \texttt{\textit{ClassName}}
        string.
        %%%%%%
        %
        The reference to the newly-created object is bound to \texttt{\textit{ObjectRef}},
        which is typically a ground Prolog term; alternatively, an unbound term
        may be used, in which case the term is bound to an automatically-generated
        %%%%%% RICCI 020202
        Prolog atom \verb|'$obj_N'|, where \texttt{N} is a progressive integer.
        %%%%%%
        %
        In both cases, these atoms are interpreted as object references --
        and therefore used to operate on the Java object from Prolog -- \textit{only}
        in the context of \texttt{JavaLibrary}'s predicates.
        %
        %%%%%% RICCI 020202
        %
        The predicate fails whenever \textit{ClassName} does not identify a valid Java class,
        or the constructor does not exists, or arguments in
        \texttt{\textit{ArgumentList}} are not ground, or \textit{ObjectRef}
        already identifies an object in the system.
        %
        %%%%%%

        According to the default behaviour of \texttt{java\_object},
        when a ground term is bound to a Java object by means of the predicate,
        the binding is kept for the full time of the demonstration
        (even in the case of backtracking).
        %
        This behaviour can be changed, getting the bindings
        created by the \texttt{java\_object} undone by
        backtracking, by changing the value of the flag \texttt{java\_object\_backtrackable}
        to \texttt{true} (the default is \texttt{false}).



  \item the \texttt{<-/2} predicate is used to invoke a method on a Java
        object according to a send-message pattern:
        %
        \begin{center}
        \texttt{\textit{ObjectRef} <- \textit{MethodName}(\textit{Arguments})}

        \texttt{\textit{ObjectRef} <- \textit{MethodName}(\textit{Arguments})
                returns \textit{Term}}
        \end{center}
        %
        \texttt{\textit{ObjectRef}} is an atom interpreted as a Java object
        reference as explained above, while \texttt{\textit{MethodName}}
        is the Java name of the method to be invoked, along with its
        \texttt{\textit{Arguments}}.
        %
        The \texttt{returns} keyword is used to retrieve the value returned
        from non-void Java methods and bind it to a Prolog term: if
        the type of the returned value can be mapped onto a primitive Prolog
        data type (a number or a string), \texttt{\textit{Term}} is unified
        with the corresponding Prolog value; if, instead, it is a Java object
        other than the ones above, \texttt{\textit{Term}} is handled
        as \texttt{\textit{ObjectRef}} in the case of \texttt{java\_object/3}.
        %
        %%%%%% RICCI 020202
        %
        % - static methods access
        %
        Static methods can be invoked using the compound
        term \texttt{class(\textit{ClassName})} in the place
        of \texttt{\textit{ObjectRef}}.
        %
        % - accennare al most specific method?
        %
        %%%%%%
        %%%%%% RICCI 020202
        %
        % - the predicates fails if...
        %
        If \textit{MethodName} does not identify a valid method for the object (class),
        or arguments in \texttt{\textit{ArgumentList}} are not
        valid (because of a wrong signature or not ground values) the predicate fails.
        %%%%%%

  \item the \texttt{.} infix operator is used, in conjunction with the \texttt{set}
        / \texttt{get} pseudo-method pair, to access the public fields of a Java
        object.
        %
        The syntax is based on the following constructs:
        %
        \begin{center}
        \tt
        \textit{ObjectRef} . \textit{Field} <- set(\textit{GroundTerm})\\
        \textit{ObjectRef} . \textit{Field} <- get(\textit{Term})\\
        \end{center}
        %
        As usual, \texttt{\textit{ObjectRef}} is the Prolog identifier for
        a Java object.
        %
        The first construct set the public field \texttt{\textit{Field}}
        to the specified \texttt{\textit{GroundTerm}}, which may be either
        a value of a primitive data type, or a reference to an existing
        object: if \texttt{\textit{GroundTerm}} is not ground, the infix
        predicate fails.
        %
        The second construct retrieves the value of the public field
        \texttt{\textit{Field}}, where \texttt{\textit{Term}} is handled
        once again as \texttt{\textit{ObjectRef}} in the case of
        \texttt{java\_object/3}.
        %
        %%%%%% RICCI 020202
        %
        % - static class access
        %
        As for methods, static fields of classes can be accessed using the compound
        term \texttt{class(\textit{ClassName})} in the place
        of \texttt{\textit{ObjectRef}}.
        %
        % - accesso ad array
        Some helper predicates are provided to access Java array
        elements:\\
        \texttt{java\_array\_set(\textit{ArrayRef}, \textit{Index}, \textit{Object})}\\
        \texttt{java\_array\_set\_\textit{\emph{Basic Type}}(\textit{ArrayRef}, \textit{Index}, \textit{Value})}\\
        to set elements,\\
        \texttt{java\_array\_get(\textit{ArrayRef}, \textit{Index}, \textit{Object})}\\
        \texttt{java\_array\_get\_\textit{\emph{Basic Type}}(\textit{ArrayRef}, \textit{Index}, \textit{Value})}\\
        to get elements,\\
        \texttt{java\_array\_length(\textit{ArrayObject}, \textit{Size})}
        to get the array length.\\
        %
        %%%%%%
        It is worth to point out that the \texttt{set} and \texttt{get} formal
        pseudo-methods above are \textit{not} methods of some class, but just
        part of the construct of the \texttt{.} infix operator, according to
        a JavaBeans-like approach.

  \item the \texttt{as} infix operator is used to explicitly specify (i.e., cast)
        method argument types:
        %
        \begin{center}
        \texttt{\textit{ObjectRef} as \textit{ClassName}}
        \end{center}
        %
        By writing so, the object represented by \texttt{\textit{ObjectRef}} is
        considered to belong to class \texttt{\textit{Classname}}: both
        \texttt{\textit{ObjectRef}} and \texttt{\textit{Classname}} have
        the usual meaning explained above.
        %
        %%%%%% RICCI 020202
        The operator works also with primitive Java types, specified
        as \texttt{\textit{Classname}} (for instance, \texttt{myNumber \textit{as int}}).
        %%%%%%
        %
        The purpose of this predicate is both to provide methods with the
        exact Java types required, and to solve possible overloading conflicts
        a-priori.
        %
        %%%%%% RICCI 020202
        %
        %In particular, \texttt{as} is needed when invoking a method whose a
        %formal argument is of type \texttt{\textit{T}} (e.g., \texttt{JButton})
        %with an actual object argument whose type is a subclass of
        %\texttt{\textit{T}} (say, \texttt{myButton}) -- that is, when upcasting
        %is needed to let Java identify the method signature unambiguously.
        %%%%%%

  %%%%%% RICCI 020202
  %
  \item The \texttt{java\_class/4} predicate makes it possible
        to create and load a new Java class from a source text provided as an
        argument, thus supporting \textit{dynamic compilation} of Java
        classes:
        %
        %
        \begin{center}
        \texttt{java\_class(\textit{SourceText},
                            \textit{FullClassName},
                            \textit{ClassPathList},
                            \textit{ObjectRef})}
        \end{center}
        %
        \texttt{\textit{SourceText}} is a string representing the
        text source of the Java class, \texttt{\textit{FullClassName}}
        is the full Java class name, and \texttt{\textit{ClassPathList}}
        is a (possibly empty) Prolog list of class paths that may
        be required for a successful dynamic compilation
        of this class.
        %
        \texttt{\textit{ObjectRef}} is a reference to an instance of the
        class \texttt{java.lang.Class} that represents the newly-created class.
        %
        The predicate fails whenever \texttt{\textit{SourceText}} contains errors,
        or the class cannot be located in the package hierarchy
        as specified, or \texttt{\textit{ObjectRef}} already identifies an object
        in the system.
  %
  %%%%%%

\end{enumerate}
%
%%%%%% RICCI 020202
%
\noindent Generally, exceptions thrown by method or constructor
calls cannot be explicitly managed and cause the failure of the
related predicate.
%

To taste the flavour of \texttt{JavaLibrary}, let us consider the
example below (refer to \xf{jreflect-example} for \texttt{Counter}
class definition):

%
{\small
\begin{verbatim}
    ?-  java_object('Counter', ['MyCounter'], myCounter),
        myCounter <- setValue(5),
        myCounter <- inc,
        myCounter <- getValue returns Value,
        write(X),

        class('Counter') <- getVersion return Version,

        myCounter.name <- get(Name),
        class('java.lang.System') . out <- get(Out),
        Out <- println(Name),

        myCounter.name <- set('MyCounter2'),

        java_object('Counter[]', [10], ArrayCounters),
        java_array_set(ArrayCounters, 0, myCounter).
\end{verbatim}}
%
\noindent Here, a \texttt{Counter} object is created providing the
\texttt{MyCounter} name as constructor argument: the reference to
the new object is bound to the Prolog atom \texttt{myCounter}.
%
This reference is then used for method invocation via the
\texttt{<-} operator, calling the \texttt{setValue(5)} method
(which is void and therefore returns nothing) first, incrementing
the counter (no arguments are specified) and invoking the
\texttt{getValue} method just after.
%
Since \texttt{getValue} returns an integer value, the
\texttt{returns} operator retrieves the method result (hopefully,
5) and binds it to the \texttt{X} Prolog variable, which is
printed via the Prolog \texttt{write/1} predicate.
%
Of course, if the Prolog variable \texttt{X} is already bound to
5, the predicate succeeds as well, while fails if \texttt{X} is
bound to anything else.
%
%%%%%%
Then, the static method \texttt{getVersion} is called, retrieving
the version of the class \texttt{Counter}, and printed using the
method \texttt{println} provided by the static \texttt{out} field
in the \texttt{java.lang.System} class.
%
The \texttt{name} public field of \texttt{myCounter} object is
then accessed, setting the \texttt{MyCounter2} value.
%
Finally, an array of 10 counters is created, and the
\texttt{myCounter} object assigned to its first element.
%%%%%%
%

The key point here is that the only requirement for this example
to run is the presence of the \texttt{Counter.class} file in the
proper position in the file system, according to Java naming
conventions: no other auxiliary information is needed -- no
headers, no pre-compilations, etc.
%
This enables the seamless reuse and exploitation of the large
amount of available Java libraries and resources, starting from
the standard ones, such as Swing to manage GUI components, JDBC to
access databases, RMI and CORBA for distributed computing, and so
on.
%
\xf{jexamples-swing} shows an example, where Java Swing API is
exploited to graphically choose a file from Prolog: a Swing
\texttt{JFileChooser} dialog is instantiated and bound to the
Prolog variable \texttt{Dialog} (a univocal Prolog atom of the form
\verb|'$obj_N'|, to be
used as the object reference, is automatically generated and bounded
to the variable) which
is then used to invoke methods \texttt{showOpenDialog} and
\texttt{getSelectedFile} of \texttt{JFileChooser}'s interface.
%
Further examples about exploiting standard Java libraries from
\tuprolog{}
%can be found in the Appendix A.
can be found in \cite{tuprolog-padl2001}.

\begin{figure}
\caption{Using a Swing componen from a \tuprolog{} program. Note the \texttt{\_} Prolog value used to represent the Java \texttt{null} value.
\labelfig{jexamples-swing}}
\begin{verbatim}
test_open_file_dialog(FileName) :-
    java_object('javax.swing.JFileChooser', [], Dialog),
    Dialog <- showOpenDialog(_),
    Dialog <- getSelectedFile returns File,
    File <- getName returns FileName.
\end{verbatim}
\end{figure}

Besides the Prolog predicates, \texttt{JavaLibrary} embeds the
\texttt{register} function, which, unlike the previous
functionalities, is to be used on the Java side.
%
Its purpose is to associate an existing Java object
\texttt{\textit{obj}} to a Prolog identifier
\texttt{\textit{ObjectRef}}, according to the syntax:
%
\begin{center}
 \small\tt
 boolean register(Struct \textit{ObjectRef}, Object \textit{obj})
    throws InvalidObjectIdException;\\
\end{center}
%
\texttt{\textit{ObjectRef}} is a ground term (otherwise an
exception is raised) that represents the Java object
\texttt{\textit{obj}} in the context of \texttt{JavaLibrary}'s
predicates: the function returns \texttt{false} if the object
represented by \texttt{\textit{obj}} is already registered under a
different \texttt{\textit{ObjectRef}}.
%
As an example of use, let us consider the following
case:\footnote{An
  explicit cast to \texttt{alice.tuprolog.lib.JavaLibrary} is needed because
  \texttt{loadLibrary} returns a reference to a generic
  \texttt{Library}, while the \texttt{register} primitive is defined in
  \texttt{JavaLibrary} only.}
%
{\small
\begin{verbatim}
Prolog core = new Prolog();
Library lib = core.loadLibrary("alice.tuprolog.lib.JavaLibrary");
((alice.tuprolog.lib.JavaLibrary)lib).register(new Struct("stdout"),
                                               System.out);
\end{verbatim}}
%
\noindent Here, the Java object \texttt{System.out} is registered
for use in \tuprolog{} under the name \texttt{stdout}.
%
So, within the scope of the \texttt{core} engine, a Prolog program
can now contain
\begin{verbatim}
stdout <- println('What a nice message!')
\end{verbatim}
as if \texttt{stdout} was a pre-defined \tuprolog{} identifier.

%---------------------------------------------------------------------
\section{Predicates}
%---------------------------------------------------------------------

\noindent Here follows a list of predicates implemented by this
library, grouped in categories corresponding to the functionalities
they provide.

%---------------------------------------------------------------------
\subsection{Method Invocation, Object and Class Creation}
%---------------------------------------------------------------------

\begin{itemize}
%
\item \bti{java\_object/3}\\
\noindent\bt{java\_object(ClassName, ArgList, ObjId)} is true iff
\bt{ClassName} is the full class name of a Java class available on
the local file system, \bt{ArgList} is a list of arguments that
can be meaningfully used to instantiate an object of the class,
and \bt{ObjId} can be used to reference such an object;
%
as a side effect, the Java object is created and the reference to
it is unified with \bt{ObjId}.
%
It is worth noting that \bt{ObjId} can be a Prolog variable (that
will be bound to a ground term) as well as a ground term (not a
number).\\
\template{java\_object(+full\_class\_name, +list, ?obj\_id)}
%
\item \bti{java\_object\_bt/3}\\
\noindent\bt{java\_object\_bt(ClassName, ArgList, ObjId)} has the same behaviour of \bt{java\_object/3}, but the binding that is established between the \bt{ObjId} term and the Java object is destroyed with backtracking.\\
\template{java\_object\_bt(+full\_class\_name, +list, ?obj\_id)}
%
\item \bti{destroy\_object/1}\\
\noindent\bt{destroy\_object(ObjId)} is true and as a side effect
the binding between \bt{ObjId} and a Java object,
possibly established, by previous predicates is destroyed.\\
\template{destroy\_object(@obj\_id)}
%
\item \bti{java\_class/4}\\
\noindent\bt{java\_class(ClassSourceText, FullClassName, ClassPathList, ObjId)}
is true iff \bt{ClassSouceText} is a source string describing a
valid Java class declaration, a class whose full name is
\bt{FullClassName}, according to the classes found in paths
listed in \bt{ClassPathList}, and \bt{ObjId} can be used as a
meaningful reference for a \texttt{java.lang.Class} object
representing that class;
%
as a side effect the described class is (possibly created and)
loaded and made available to the system.\\
\template{java\_class(@java\_source, @full\_class\_name, @list, ?obj\_id)}
%
\item \bti{java\_call/3}\\
\noindent\bt{java\_call(ObjId, MethodInfo, ObjIdResult)} is true iff
\bt{ObjId} is a ground term currently referencing a Java object,
which provides a method whose name is the functor name of the term
\bt{MethodInfo} and possible arguments the arguments of
\bt{MethodInfo} as a compound, and \bt{ObjIdResult} can be used as
a meaningful reference for the Java object that the method
possibly returns.
%
As a side effect the method is called on the Java object
referenced by the \bt{ObjId} and the object possibly returned by
the method invocation is referenced by the \bt{ObjIdResult} term.
%
The anonymous variable used as argument in the \bt{MethodInfo}
structure is interpreted as the Java \texttt{null} value.\\
\template{java\_call(@obj\_id, @method\_signature, ?obj\_id)}
%
\item \verb|'<-'/2|\\
\noindent\verb|'<-'(ObjId, MethodInfo)| is true iff \bt{ObjId} is
a ground term currently referencing a Java object, which provides a
method whose name is the functor name of the term \bt{MethodInfo}
and possible arguments the arguments of \bt{MethodInfo} as a
compound.
%
As a side effect the method is called on the Java object
referenced by the \bt{ObjId}.
%
The anonymous variable used as argument in the \bt{MethodInfo}
structure is interpreted as the Java \texttt{null} value.\\
\template{'<-'(@obj\_id, @method\_signature)}
%
\item \bti{return/2}\\
\noindent\verb|return('<-'(ObjId, MethodInfo), ObjIdResult)| is true
iff \bt{ObjId} is a ground term currently referencing a Java object,
which provides a method whose name is the functor name of the term
\bt{MethodInfo} and possible arguments the arguments of
\bt{MethodInfo} as a compound, and \bt{ObjIdResult} can be used as
a meaningful reference for the Java object that the method
possibly returns.
%
As a side effect the method is called on the Java object
referenced by the \bt{ObjId} and the object possibly returned by
the method invocation is referenced by the \bt{ObjIdResult} term.
%
The anonymous variable used as argument in the \bt{MethodInfo}
structure is interpreted as the Java \texttt{null} value.\\
%
It is worth noting that this predicate is equivalent to the
\texttt{java\_call} predicate.\\
\template{return('<-'(@obj\_id, @method\_signature), ?obj\_id)}
%
\end{itemize}

%---------------------------------------------------------------------
\subsection{Java Array Management}
%---------------------------------------------------------------------
\begin{itemize}
%
\item \bti{java\_array\_set/3}\\
\noindent\bt{java\_array\_set(ObjArrayId, Index, ObjId)} is true iff
\bt{ObjArrayId} is a ground term currently referencing a Java
array object, \bt{Index} is a valid index for the array and
\bt{ObjId} is a ground term currently referencing a Java object
that could inserted as an element of the array (according to Java
type rules).
%
As side effect, the object referenced by \bt{ObjId} is set in the
array referenced by \bt{ObjArrayId} in the position (starting from
0, following the Java convention) specified by \bt{Index}.
%
The anonymous variable used as \bt{ObjId} is interpreted as the
Java \texttt{null} value.
%
This predicate can be used for arrays of Java objects:
%
for arrays whose elements are Java primitive types (such as
\texttt{int}, \texttt{float}, etc.) the following predicates can
be used, with the same semantics of \bt{java\_array\_set} but
specifying directly the term to be set as a \tuprolog{} term
(according to the mapping described previously):\\
%
\mbox{~~~~}\bt{java\_array\_set\_int(ObjArrayId, Index, Integer)}\\
\mbox{~~~~}\bt{java\_array\_set\_short(ObjArrayId, Index, ShortInteger)}\\
\mbox{~~~~}\bt{java\_array\_set\_long(ObjArrayId, Index, LongInteger)}\\
\mbox{~~~~}\bt{java\_array\_set\_float(ObjArrayId, Index, Float)}\\
\mbox{~~~~}\bt{java\_array\_set\_double(ObjArrayId, Index, Double)}\\
\mbox{~~~~}\bt{java\_array\_set\_char(ObjArrayId, Index, Char)}\\
\mbox{~~~~}\bt{java\_array\_set\_byte(ObjArrayId, Index, Byte)}\\
\mbox{~~~~}\bt{java\_array\_set\_boolean(ObjArrayId, Index, Boolean)}\\
%
\template{java\_array\_set(@obj\_id, @positive\_integer, +obj\_id)}
%
%
%
\item \bti{java\_array\_get/3}\\
\noindent\bt{java\_array\_get(ObjArrayId, Index, ObjIdResult)} is
true iff \bt{ObjArrayId} is a ground term currently referencing a
Java array object, \bt{Index} is a valid index for the array, and
\bt{ObjIdResult} can be used as a meaningful reference for a Java
object contained in the array.
%
As a side effect, \bt{ObjIdResult} is unified with the reference to
the Java object of the array referenced by \bt{ObjArrayId} in the
\bt{Index} position.
%
This predicate can be used for arrays of Java objects:
%
for arrays whose elements are Java primitive types (such as
\texttt{int}, \texttt{float}, etc.) the following predicates can
be used, with the same semantics of \bt{java\_array\_get} but
binding directly the array element to a \tuprolog{} term
(according to the mapping described previously):\\
%
\mbox{~~~~}\bt{java\_array\_get\_int(ObjArrayId, Index, Integer)}\\
\mbox{~~~~}\bt{java\_array\_get\_short(ObjArrayId, Index, ShortInteger)}\\
\mbox{~~~~}\bt{java\_array\_get\_long(ObjArrayId, Index, LongInteger)}\\
\mbox{~~~~}\bt{java\_array\_get\_float(ObjArrayId, Index, Float)}\\
\mbox{~~~~}\bt{java\_array\_get\_double(ObjArrayId, Index, Double)}\\
\mbox{~~~~}\bt{java\_array\_get\_char(ObjArrayId, Index, Char)}\\
\mbox{~~~~}\bt{java\_array\_get\_byte(ObjArrayId, Index, Byte)}\\
\mbox{~~~~}\bt{java\_array\_get\_boolean(ObjArrayId, Index, Boolean)}\\
%
\template{java\_array\_get(@obj\_id, @positive\_integer, ?obj\_id)}
%
%
\item \bti{java\_array\_length/2}\\
\noindent\bt{java\_array\_length(ObjArrayId, ArrayLength)} is true
iff \bt{ArrayLength} is the length of the Java array referenced by
the term \bt{ObjArrayId}.\\
\template{java\_array\_length(@term, ?integer)}
%
\end{itemize}

%---------------------------------------------------------------------
\subsection{Helper Predicates}
%---------------------------------------------------------------------

\begin{itemize}
%
\item \bti{java\_object\_string/2}\\
\noindent\bt{java\_object\_string(ObjId, String)} is true iff
\bt{ObjId} is a term referencing a Java object and
\bt{PrologString} is the string representation of the object
(according to the semantics of the \texttt{toString} method
provided by the Java object).\\
\template{java\_object\_string(@obj\_id, ?string)}
%
\end{itemize}

%---------------------------------------------------------------------
\section{Functors}
%---------------------------------------------------------------------

No functors are provided by the \texttt{JavaLibrary} library.

%---------------------------------------------------------------------
\section{Operators}
%---------------------------------------------------------------------

\begin{table}[h]
    %
    \begin{center}{\small\tt
    \begin{tabular}{p{2cm}|p{1cm}|p{1cm}}\hline\hline
    Name & Assoc. & Prio. \\ \hline\hline
    <-   & xfx & 800\\
    returns     & xfx & 850 \\
    as   & xfx & 200\\
    .   & xfx & 600\\
    \hline\hline
    \end{tabular}
    }\end{center}
\end{table}

%\clearpage


%-----------------------------------------------------------------------
\section{Java Library Examples}
%-----------------------------------------------------------------------

The following examples are designed to show \texttt{JavaLibrary}'s
ease of use and flexibility.

%-------------------------------
\subsection{RMI Connection to a Remote Object}
%-------------------------------

Here we connect via RMI to a remote Java object.
%
In order to allow the reader to try this example with no need of
other objects, we connect to the remote Java object identified by
the name \verb|'prolog'|, which is an RMI server bundled with
the \tuprolog{} package, and can be spawned by typing:

{\small%
\texttt{java -Djava.security.all=policy.all  alice.tuprologx.runtime.rmi.Daemon}
}

\noindent Then, we invoke the object method whose signature is

{\small%
\texttt{SolveInfo solve(String goal);}
}
%
{\small%
\begin{verbatim}
    ?-  java_object('java.rmi.RMISecurityManager', [], Manager),
        class('java.lang.System') <- setSecurityManager(Manager),
        class('java.rmi.Naming') <- lookup('prolog') returns Engine,
        Engine <- solve('append([1],[2],X).') returns SolInfo,
        SolInfo <- success returns Ok,
        SolInfo <- getSubstitution returns Sub,
        Sub <- toString returns SubStr, write(SubStr), nl,
        SolInfo <- getSolution returns Sol,
        Sol <- toString returns SolStr, write(SolStr), nl.
\end{verbatim}
}
%
\noindent The Java version of the same code would be:
%
{\small%
\begin{verbatim}
        System.setSecurityManager(new RMISecurityManager());
        PrologRMI core = (PrologRMI) Naming.lookup("prolog");
        SolveInfo info = core.solve("append([1],[2],X).");
        boolean ok = info.success();
        String sub = info.getSubstiturion();
        System.out.println(sub);
        String sol = info.getSolution();
        System.out.println(sol);
\end{verbatim}
}


%-------------------------------
\subsection{Java Swing GUI from \tuprolog}
%-------------------------------

What about creating Java GUI components from the \tuprolog{}
environment?
%
Here is a little example, where a standard Java Swing open file
dialog windows is popped up:
%
{\small%
\begin{verbatim}
    open_file_dialog(FileName):-
        java_object('javax.swing.JFileChooser', [], Dialog ),
        Dialog <- showOpenDialog(_) returns Result,
        write(Result),
        Dialog <- getSelectedFile returns File,
        File <- getName returns FileName,
        class('java.lang.System') . out <- get(Out),
        Out <- println('you want to open file '),
        Out <- println(FileName).
\end{verbatim}
}

%-------------------------------
\subsection{Database access via JDBC from \tuprolog}
%-------------------------------

This example shows how to access a database via the Java standard
JDBC interface from \tuprolog{}.
%
The program computes the minimum path between two cities, fetching
the required data from the database called `distances'.
%
The entry point of the Prolog program is the \texttt{find\_path}
predicate.
%
{\small%
\begin{verbatim}
    find_path(From, To) :-
        init_dbase('jdbc:odbc:distances', Connection, '', ''),
        exec_query(Connection,
          'SELECT city_from, city_to, distance FROM distances.txt',
          ResultSet),
        assert_result(ResultSet),
        findall(pa(Length,L), paths(From,To,L,Length), PathList),
        current_prolog_flag(max_integer, Max),
        min_path(PathList, pa(Max,_), pa(MinLength,MinList)),
        outputResult(From, To, MinList, MinLength).

    paths(A, B, List, Length) :-
        path(A, B, List, Length, []).

    path(A, A, [], 0, _).
    path(A, B, [City|Cities], Length, VisitedCities) :-
        distance(A, City, Length1),
        not(member(City, VisitedCities)),
        path(City, B, Cities, Length2, [City|VisitedCities]),
        Length is Length1 + Length2.

    min_path([], X, X) :- !.
    min_path([pa(Length, List) | L],  pa(MinLen,MinList), Res) :-
        Length < MinLen, !,
        min_path(L, pa(Length,List), Res).
    min_path([_|MorePaths], CurrentMinPath, Res) :-
        min_path(MorePaths, CurrentMinPath, Res).

    writeList([]) :- !.
    writeList([X|L]) :- write(','), write(X), !, writeList(L).

    outputResult(From, To, [], _) :- !,
        write('no path found from '), write(From),
        write(' to '), write(To), nl.
    outputResult(From, To, MinList, MinLength) :-
        write('min path from '), write(From),
        write(' to '), write(To), write(': '),
        write(From), writeList(MinList),
        write('  - length: '), write(MinLength).

    % Access to Database

    init_dbase(DBase, Username, Password, Connection) :-
        class('java.lang.Class') <- forName('sun.jdbc.odbc.JdbcOdbcDriver'),
        class('java.sql.DriverManager') <- getConnection(DBase, Username, Password)
            returns Connection,
        write('[ Database '), write(DBase), write(' connected ]'), nl.

    exec_query(Connection, Query, ResultSet):-
        Connection <- createStatement returns Statement,
        Statement <- executeQuery(Query) returns ResultSet,
        write('[ query '), write(Query), write(' executed ]'), nl.

    assert_result(ResultSet) :-
        ResultSet <- next returns Valid, Valid == true, !,
        ResultSet <- getString('city_from') returns From,
        ResultSet <- getString('city_to') returns To,
        ResultSet <- getInt('distance') returns Dist,
        assert(distance(From, To, Dist)),
        assert_result(ResultSet).
    assert_result(_).
\end{verbatim}
}

%-------------------------------
\subsection{Dynamic compilation}
%-------------------------------

As already said, the \texttt{java\_class} predicate performs
\textit{dynamic compilation}, creating an instance of a Java
\texttt{Class} class that represents the public class declared in
the source text provided as argument.
%
The created \texttt{Class} instance, referenced by a Prolog term,
can be used to create instances via the \texttt{newInstance}
method, to retrieve specific constructors via the
\texttt{getConstructor} method, to analyze class methods and
fields, and for other above-mentioned meta-services: a sketch is
reported in \xf{dynamic-compilation}.
%
The \texttt{java\_class} arguments in the example specify, besides
the source text and the binding variable, the full class name
(\texttt{Counter}), which is necessary to locate the class in the
package hierarchy, and possibly a list of class paths required
for a successful compilation (if any).

\begin{figure}
\caption{Predicate \texttt{java\_class} performing dynamic compilation of Java code in \tuprolog{}.
\labelfig{dynamic-compilation}}
\begin{verbatim}
    ?- Source = 'public class Counter { ... }',
       java_class(Source, 'Counter', [], counterClass),
       counterClass <- newInstance returns myCounter,
       myCounter <- setValue(5),
       myCounter <- getValue returns X,
       write(X).
\end{verbatim}
\end{figure}

\xf{jcexamples} shows a more complex example, where a Java source
is retrieved via FTP and then exploited first to create a new
(previously unknown) class, and then a new instance of that class.
(The FTP service is provided by a shareware Java library.)
%
\begin{figure}
\caption{A new Java class is compiled and used after being retrieved via FTP.
\labelfig{jcexamples}}
{\scriptsize
\begin{verbatim}
% A user whose name is 'myName' and whose password is 'myPwd' gets the content of the file
% 'Counter.java' from the server whose IP address is 'srvAddr', creates the corresponding
% Java class and exploits it to instantiate and deploy an object

test :-
    get_remote_file('alice/tuprolog/test', 'Counter.java', srvAddr, myName, myPwd, Content),
    % creating the class
    java_class(Content, 'Counter', [], CounterClass),
    % instantiating (and using) an object of such a class
    CounterClass <- newInstance returns MyCounter,
    MyCounter <- setValue(303),
    MyCounter <- inc,
    MyCounter <- inc,
    MyCounter <- getValue returns Value,
    write(Value), nl.

% +DirName: Directory on the server where the file is located
% +FileName: Name of the file to be retrieved
% +FTPHost: IP address of the FTP server
% +FTPUser: User name of the FTP client
% +FTPPwd: Password of the FTP client
% -Content: Content of the retrieved file

get_remote_file(DirName, FileName, FTPHost, FTPUser, FTPPwd, Content) :-
    java_object('com.enterprisedt.net.ftp.FTPClient', [FTPHost], Client),
    % get file
    Client <- login(FTPUser, FTPPwd),
    Client <- chdir(DirName),
    Client <- get(FileName) returns Content,
    Client <- quit.
\end{verbatim}
}
\end{figure}
%
Though a lot remains to explore, \texttt{java\_class} features
seem quite interesting: in perspective one might think, for
instance, of a Prolog intelligent agent that dynamically acquires
information on a Java resource, and then autonomously builds up,
at run-time, the proper Java machinery enabling efficient
interaction with the resource.

%\section{The IDE}

The \tuprolog\ system comes with a simple application providing an user friendly integrated development environment to interact with a \tuprolog\ engine, manipulate its knowledge base, make queries and explore solutions.
%
In addition, means to dynamically manage the loading and unloading of \tuprolog\ libraries are provided.
%
After a proper installation of the \tuprolog\ distribution, the application is spawned by launching the executable class \classname{alice.tuprologx.ide.GUILauncher}.
%
The console user interface version, providing a command-line shell, can be accessed by launching the executable class \classname{alice.tuprologx.ide.CUIConsole}.

\begin{figure}
\centering
\includegraphics[scale=0.60]{images/tuPrologIDE}
\caption{\tuprolog\ IDE.}
\label{tuprolog-ide}
\end{figure}

The main window of the \tuprolog\ IDE is shown in \figref{tuprolog-ide}.
%
It is divided in two sections:
%
\begin{itemize}
\item an editing area on the middle, providing means to edit the engine's current theory;
\item a console on the bottom, providing means to ask queries and display their solutions.
\end{itemize}
%
In the main window there also are:
%
a toolbar at the top, providing facilities to manage theories, such as load, save as well as create a new theory, to load and unload libraries into and from the \tuprolog\ engine, and to view in a separate window the debug informations activated by means of the \predicate{spy/0} predicate;
%
and a status bar at the very bottom, providing status informations for the IDE and the engine.

\subsection{Editing the theory}

The editing area allows multiple theories to be created and modified at the same time, by allocating a tab with a new text area for each theory.
%
The text area provides syntax highlighting for comments, string and list literals, and predefined predicates.
%
Undo and Redo actions are supported through the usual \keycap{Ctrl}+\keycap{Z} and \keycap{Ctrl}+\keycap{Shift}+\keycap{Z} key bindings.

\begin{figure}
\centering
\includegraphics[scale=0.60]{images/syntaxErrorFound}
\caption{A syntax error is found when setting the content of the editor area as the new engine's theory.}
\label{syntax-error-found}
\end{figure}

\begin{figure}
\centering
\includegraphics[scale=0.60]{images/setTheorySucceeded}
\caption{The syntax error is removed and the \guibutton{Set Theory} operation succeeds.}
\label{set-theory-succeeded}
\end{figure}

Above the editing tabs, a control area is found, where two buttons are provided to get the text of the engine's current theory into a new tab and to set the text contained in the editor of the selected tab as a new theory for the engine, and two buttons are provided for mouse-clicking support of Undo and Redo
actions.
%
An apposite action for retrieving the engine's current theory in an editor (shown in \figref{syntax-error-found}) is needed because whenever that theory gets modified by other means, such as calling the \texttt{consult/1} predicate, the changes are not automatically reflected in any text area.
%
On the left side of the control area, there also is an indicator of the line where the caret is currently positioned in the edit area.
%
Informations about the result of the action issued by the control area are provided in the status bar at the very bottom of the IDE's window:
%
for instance, when setting an invalid theory to the engine because of syntax errors, details about the error are provided.

\subsection{Solving goals}

The console at the bottom of the tuProlog IDE's window is subdivided in two logical panes:
%
\begin{itemize}
\item a query pane composed by a textfield where queries can be inserted, and two buttons to trigger the solving process.
%
The leftmost (\guibutton{Solve}) button asks the engine to find the first solution to the query, allowing the user to possibly navigate through further solutions;
%
the rightmost (\guibutton{Solve All}) button forces the engine to find all solutions to the given query.
%
Pressing the \keycap{Enter} key in the textfield has the same effect as pressing the \guibutton{Solve} button.
\item an answer pane, where answers and output informations are visualized.
%
Answers to Prolog queries are composed by both solutions, showed in a free form within a read-only text area, and bindings, displayed in tabular form.
%
The output tab provides a read-only view on the standard output where informations are possibly written by Prolog programs, by means of the I/O predicates supplied by the \classname{IOLibrary}.\footnote{The information written on standard output by methods invoked on Java objects from the \classname{JavaLibrary} -- for instance using the \varname{stdout} object -- are not displayed on this view.}
%
Control buttons are also provided to iterate through possibly multiple solutions, clear the bindings and output panes, and export tabular data in a convenient CSV format.
\end{itemize}
%
Goals are asked through the query input box, and answers (bindings and solutions) are provided in the related text area.
%
Query and answers are traced in a proper chronological history, that can be explored by means of \keycap{Up} and \keycap{Down} arrow keys from the query input textfield.
%
When open alternatives are found solving a goal, the \guibutton{Next} and \guibutton{Accept} buttons are enabled in the answer pane to interact with the engine, in order to let the user specify if the current solution is accepted or if other alternatives need to be explored.

Note that the theory contained in the currently selected edit pane does not have to be explicitly feeded to the Prolog engine before it could be possible to solve queries against that theory's knowledge base.
%
In fact, any time a goal is asked to be solved, the theory contained in the active edit area is automatically feeded to the engine if its knowledge base has been modified since the last solved goal.
%
(This obviously happens also on the first time a query is asked.)
%
However, whenever the engine's theory is modified by other means than the editor, it does need to be explicitly acquired and presented to the programmer in the text area.
%
In fact, if the theory in the engine is augmented by a call to the predicate \predicate{consult/1} issued from a query, for example, the contents of the newly consulted theory will not be automatically inserted in the editor:
%
when the programmer needs an up-to-date view of the knowledge base contained in the underlying \tuprolog\ engine, its display has to be explicitly triggered by means of the \guibutton{GetTheory} button, available
in the editing area.

An example of the user interaction involving multiple solutions is shown in the following sequence of figures:
%
in \figref{query-issued}, the user issued the query \userinput{test\_color(test, X).}, using the knowledge base written in the edit area (a solution to the Map Coloring problem,\footnote{The problem is to color a planar map so that no two adjoining regions have the same color. A famous conjecture was proved in 1976 showing that four colors are sufficient to color any planar map.} with a test map composed of six areas).
%
The first solution is displayed, and multiple open alternatives can be explored:
%
in \figref{next-solution-asked}, the user asked to get the next possible solution by pressing the \guibutton{Next} button, and another solution is provided;
%
finally, in \figref{accept-solution}, the user, after having explored the first two solutions, accepts the third one by pressing the \guibutton{Accept} button.
%
During the resolution of a goal, all the theory-related buttons are disabled, included the \guibutton{Library Manager} button, since each library can have its theory to be feeded into the engine.

\begin{figure}
\centering
\includegraphics[scale=0.605]{images/queryIssued}
\caption{The user issued a query \userinput{test\_color(test, X).} and the first solution is displayed.}
\label{query-issued}
\end{figure}

\begin{figure}
\centering
\includegraphics[scale=0.605]{images/nextSolutionAsked}
\caption{The user issued a \guibutton{Next} command and got another solution.}
\label{next-solution-asked}
\end{figure}

\begin{figure}
\centering
\includegraphics[scale=0.605]{images/acceptSolution}
\caption{The user accepted the third solution by pressing the \guibutton{Accept} button.}
\label{accept-solution}
\end{figure}

Near to the \guibutton{Next} and \guibutton{Accept} buttons, a \guibutton{Stop} button is found, providing the user with a means to halt the engine if a computation takes too long or a bug in the knowledge base feeded to the engine results in an infinite loop.

% Such a bug is contained in the following theory:
% \begin{verbatim}
% p(a).
% p(b) :- p(b).
% p(c).
% \end{verbatim}
% When solving a goal like \code{p(b)} or asking for the second solution to the query \code{p(X)}, the \tuprolog\ engine
% will be trapped in an infinite loop due to the particular recursive nature of the second clause in the feeded theory.
% By pressing the \guibutton{Stop} button, which is enabled only during computations, the user will be able to halt the
% engine and perform the necessary changes to the knowledge base before issuing another query, instead of being forced
% to close and reopen the IDE.

Finally, a \guibutton{Clear} button is provided, with the aim of allowing the user to clear the bindings and output panes when they get overfull with informations.
%
The button is enabled only when the proper tabs are selected.

\subsection{Debug Informations}

By pressing the \guibutton{View Debug Information} button, a new window is opened, providing a view on the warnings, produced by events such as the attempt at redefining a library predicate, and the spy information, concerning basic steps of the engine computation and state, possibly supplied by the engine during a goal demonstration:
%
Warnings are always active;
%
in order to activate the spy information notification, the \predicate{spy/0} built-in predicate (provided by \classname{BasicLibrary}) must be issued;
%
\predicate{nospy/0} can be used to stop this notification.
%
As an example, \figref{view-debug-information} shows the content of the spy information view after the
execution of a goal involving the activation of spy inspection.

It is worth noting that a computation may contain a huge number of traced steps.
%
For this reason, a toolbar at the top of the window allows to collapse and expand all nodes in the spy information pane, or to expand and collapse selected nodes only.
%
Finally, the content of the warnings and spy panes can be cleared using the \guibutton{Clear} button at the leftmost end of the toolbar.

\begin{figure}
\centering
\includegraphics[scale=0.605]{images/viewDebugInformation}
\caption{Debug Information View after the execution of a goal.}
\label{view-debug-information}
\end{figure}

\subsection{Dynamic library management}


A \tuprolog\ engine can be extended by loading any number of libraries, each provinding a specific set of built-in
predicates and functors, and a related theory. The \tuprolog\ IDE allows a dynamic management of libraries through a
GUI dialog, which can be displayed by pressing the \guibutton{Open Library Manager} button in the toolbar. The Library
Manager dialog is shown in \figref{library-manager-dialog}.

\begin{figure}
\centering
\includegraphics[scale=0.605]{images/libraryManagerDialog}
\caption{The Library Manager dialog.}
\label{library-manager-dialog}
\end{figure}

This dialog displays a list of the libraries currently loaded into the \tuprolog\ engine. For a new instance of the
engine, that list will typically contain the four standard libraries coming with the application core, that is
\classname{BasicLibrary}, \classname{IOLibrary}, \classname{ISOLibrary}, \classname{JavaLibrary}, along with their
current status. The user can add a library to the Library Manager simply by providing the fully qualified name of the
library's class in the textfield on the top of the dialog, then pressing the \guibutton{Add} button: the added library
will be displayed with an initial Unload status. The user can further select the status of each library in the list,
and commit changes to the \tuprolog\ engine by pressing the \guibutton{OK} button, or dismiss the dialog by pressing
the \guibutton{Cancel} button.

The library manager is also capable of updating itself accordingly to the events of libraries load and unload fired by
the \tuprolog\ engine.  Such events are triggered by the use of the \verb|load_library/1| and \verb|unload_library/1|
predicates or directives in query issued or theories feeded to the engine. So, if an user asks to solve the goal \verb|load_library('TestLibrary'), test(X).|,
for example, the manager would immediately reflect the change occurred in the engine's libraries pool, adding a new
entry if \verb|TestLibrary| had not been previously loaded or, if necessary, changing the library's entry status to
show the result of the loading action.

Both the action of adding a library to the manager and the action of loading a library into the engine can fail. If,
for example, the classname provided does not identify a \tuprolog\ library (i.e. it identifies a class not extending the
\classname{alice.tuprolog.Library} class) or the identified class does not exist, an appropriate message will appear
in the status bar at the bottom of the dialog. When adding or loading a library, please remember that every class
needed by that library must be in the classpath in order to have the library correctly added to the manager's list or
loaded into the engine. 
%%=======================================================================
\chapter{Using \tuprolog{} from Java}
\label{java-api}
%=======================================================================

\section{Getting started}

Let's begin with your first Java program using \tuprolog{}.
%
{\small{
\begin{verbatim}
    import alice.tuprolog.*;

    public class Test2P {
        public static void main(String[] args) throws Exception {
            Prolog engine = new Prolog();
            SolveInfo info = engine.solve("append([1],[2,3],X).");
            System.out.println(info.getSolution());
        }
    }
\end{verbatim}
}}
\noindent In this first example a \tuprolog{} engine is
created, and asked to solve a query, provided in a textual form.
%
This query in the Java environment is equivalent to the query\\\\
%
{\small{\texttt{?- append([1],[2,3],X).\\\\}}}
%
\noindent in a classic Prolog environment, and accounts for
finding the list that is obtained by appending the list
\texttt{[2,3]} to the list \texttt{[1]} (\texttt{append} is
included in the theory provided by the
\texttt{alice.tuprolog.lib.BasicLibrary}, which is downloaded by
the default when instantiating the engine).
%

By properly compiling and executing this simple program,\footnote{Save the program in a file called \texttt{Test2P.java}, then compile it with
%
\texttt{javac -classpath tuprolog.jar Test2P.java}
%
and then execute it with
%
\texttt{java -cp .;tuprolog.jar Test2P.java}} the string
\texttt{append([1],[2,3],[1,2,3])} -- that is the solution of out
query -- will be displayed on the standard output.
%
%

\noindent Then, let's consider a little bit more involved example:

{\small{
\begin{verbatim}
public class Test2P {
    public static void main(String[] args) throws Exception {
        Prolog engine = new Prolog();
        SolveInfo info = engine.solve("append(X,Y,[1,2]).");
        while (info.isSuccess()) {
            System.out.println("solution: " + info.getSolution() +
                               " - bindings: " + info);
            if (engine.hasOpenAlternatives()) {
                info = engine.solveNext();
            } else {
                break;
            }
        }
    }
}
\end{verbatim}
}}

In this case, all the solutions of a query are retrieved and
displayed, with also the variable bindings:
{\small{\begin{verbatim}

solution: append([],[1,2],[1,2]) - bindings: Y / [1,2]  X / []
solution: append([1],[2],[1,2]) - bindings: Y / [2]  X / [1]
solution: append([1,2],[],[1,2]) - bindings: Y / []  X / [1,2]

\end{verbatim}
}}

\section{Basic Data Structures}

\noindent All Prolog data objects are mapped onto Java objects:
\texttt{Term} is the base class for Prolog untyped terms such as
atoms and compound terms (represented by the \texttt{Struct} class),
Prolog variables (\texttt{Var} class) and Prolog typed terms such
as numbers (\texttt{Int}, \texttt{Long}, \texttt{Float},
\texttt{Double} classes).
%
In particular:

%
\begin{itemize}
    %
    \item \texttt{Term} -- this abstract class represents a generic Prolog term, and it is the
    root base class of \tuprolog{} data structures, defining the basic common services, such
    as term comparison, term unification and so on.
    %
    It is worth noting that it is an abstract class, so no direct \texttt{Term}
    objects can be instantiated;
    %
    %
    %
    \item \texttt{Var} -- this class (derived from \texttt{Term})
    represents \tuprolog{} variables.
    %
    A variable can be anonymous, created by means of the default constructor, with no
    arguments, or identified by a name, that must starts with an upper case letter or an
    underscore;
    %
    \item \texttt{Struct} -- this class (derived from \texttt{Term})
    represents un-typed \tuprolog{} terms, such as atoms, lists and compound terms;
    %
    \texttt{Struct} objects are characterised by a functor name and a
    list of arguments (which are \texttt{Term}s themselves), while
    \texttt{Var} objects are labelled by a string representing the
    Prolog name.
    %
    Atoms are mapped as \texttt{Struct}s with functor with a name and
    no arguments;
    %
    Lists are mapped as \texttt{Struct} objects with functor
    \verb|'.'|, and two \texttt{Term} arguments (head and tail of the list);
    %
    lists can be also built directly by exploiting the 2-arguments constructor, with
    head and tail terms as arguments.
    %
    Empty list is constructed by means of the no-argument constructor of \texttt{Struct}
    (default constructor).
    %
    \item \texttt{Number} -- this abstract class represents typed, numerical Prolog terms,
        and it is the  base class of \tuprolog{} number classes;
    %
    \item \texttt{Int, Long, Double, Float} -- these classes (derived from \texttt{Number})
    represent  typed numerical \tuprolog{} terms, respectively integer, long, double and
    float numbers.
    %
    Following the Java conventions, the default type for integer number is \texttt{Int}
    (integer, not long, number), and for \texttt{Double} (and so double), for floating point
    number.
\end{itemize}


\noindent Some examples of term creation follow:
%
{\small{\begin{verbatim}

// constructs the atom vodka
Struct drink = new Struct("vodka");

// constructs the number 40
Term degree = new alice.tuprolog.Int(40);

// constructs the compound degree(vodka, 40)
Term drinkDegree = new Struct("degree",
                              new Struct("vodka"),
                              new Int(40));
// second way to constructs the compound degree(vodka,40)
Struct drinkDegree2 = new Struct("degree", drink, degree);

// constructs the compound temperature('Rome', 25.5)
Struct temperature = new Struct("temperature",
                                new Struct("Rome"),
                                new alice.tuprolog.Float(25.5));

// constructs the compound equals(X, X)
Struct t1 = new Struct("equals", new Var("X"), new Var("X"));
t1.resolveTerm();

// mother(John,Mary)
Struct father = new Struct(new Struct("John"), new Struct("Mary")));

// father(John, _)
Term  father = new Struct(new Struct("John"), new Var());

// p(1, _, q(Y, 3.03, 'Hotel'))
Term  t2 = new Struct("p",
                      new Int(1),
                      new Var(),
                      new Struct("q",
                                 new Var("Y"),
                                 new Float(3.03f),
                                 new Struct("Hotel")));

// The Long number 130373303303
Term t3 = new alice.tuprolog.Long(130373303303h);

// The double precision number 1.7625465346573
Term t4 = new alice.tuprolog.Double(1.7625465346573);

// an empty list
Struct empty = new Struct();

// the list [303]
Struct l = new Struct(new Int(303), new Struct());

// the list [1,2,apples]
Struct alist = new Struct(
                   new Int(1),
                   new Struct(
                       new Int(2),
                       new Struct("apples")));

// fruits([apple, orange | _ ])
Term list2 = new Struct("fruits", new Struct(
                                      new Struct("apple",
                                          new Struct("orange"),
                                          new Var())));

// complex_compound(1, _, q(Y, 3.03, 'Hotel', k(Y,X)), [303, 13, _, Y])
Term t5 = Term.parse(
     "complex_compound(1, _, q(Y, 3.03, 'Hotel', k(Y,X)), [303, 13, _, Y])"
);

\end{verbatim}}}
%
\noindent The name of the \tuprolog{} number classes
(\texttt{Int}, \texttt{Float}, \texttt{Long}, \texttt{Double})
follows the name of the primitive Java data type they represents.
%
Note that due to class name clashes (for instance between classes
\texttt{java.lang.Long} and \texttt{alice.tuprolog.Long}), it
could be necessary to use the full class name to identify
\tuprolog{} classes.
%

\section{Engine, Theories and Libraries}

\noindent Then, the other main classes that make \tuprolog{} Core
API concern \tuprolog{} engines, theories and libraries. In
particular:
%
\begin{itemize}
    \item \texttt{Prolog} -- this class represent \tuprolog{}
    engines.
    %
    This class provides a minimal interface that enables users
    to: \\
    %
    \indent{-- set/get the theory to be used for
    demonstrations;}\\
        %
    \indent{-- load/unload libraries;} \\
        %
    \indent{-- solve a goal, represented either by a \texttt{Term} object or by a
        textual representation (a \texttt{String} object) of a
        term.}\\
    %
    A \tuprolog{} engine can be instantiated either with some standard default
    libraries loaded, by means of the default constructor, or with
    a starting set of libraries, which can be empty, provided as
    argument to the constructor (see JavaDoc documentation for
    details).
    %
    Accordingly, a raw, very lightweight, \tuprolog{} engine can
    be created by specifying an empty set of library, providing
    natively a very small set of built-in primitives.


    \item \texttt{Theory} -- this class represent \tuprolog{}
    theories.
    %
    A theory is represented by a text, consisting of a series of
    clauses and/or directives, each followed by a dot and a
    whitespace character.
    %
    Instances of this class are built either from a textual representation,
    directly provided as a string or taken by any possible input
    stream, or from a list of terms representing Prolog clauses.
    %
    %
    %
    %
    %
    \item \texttt{Library} -- this class represents \tuprolog{}
    libraries;
    %
    A \texttt{tuprolog} engine can be dynamically extended by loading
    (and unloading) any number of libraries; each library can provide
    a specific set of of built-ins predicates, functors and a related
    theory.
    %
    A library can be loaded by means of the built-in by means of the method
    \texttt{loadLibrary} of the \tuprolog{} engine.
    %
    Some standard libraries are provided in the
    \texttt{alice.tuprolog.lib} package and loaded by the default
    instantiation of a \tuprolog{} engine:
    %
    \texttt{alice.tuprolog.lib.BasicLibrary}, providing basic and
    well-known Prolog built-ins, \texttt{alice.tuprolog.lib.IOLibrary}
    providing \textit{de facto} standard Prolog I/O predicates, \texttt{alice.tuprolog.lib.ISOLibrary}
    providing some ISO predicates/functors not directly provided
    by \texttt{BasicLibrary} and \texttt{IOLibrary}, and
    \texttt{alice.tuprolog.lib.JavaLibrary}, which enables the
    creation and usage of Java objects from \tuprolog{} programs,
    in particular enabling the reuse of any existing Java resources.
    %
    %
    \item \texttt{SolveInfo} -- this class represents the result of a
    demonstration and instances of these class are returned by the
    \texttt{solve} methods the \texttt{Prolog} engines;
    %
    in particular \texttt{SolveInfo} objects provide services to test the
    success of the demonstration (\texttt{isSuccess} method),
    to access to the term solution of the query
    (\texttt{getSolution} method)  and to access the list of the
    variable with their bindings.
\end{itemize}
%

Some notes about \tuprolog{} terms and the services they provide:
\begin{itemize}
%
\item the static \texttt{parse} method provides a quick way to get a
term from its string representation.
%
\item \tuprolog{} terms provides directly methods for unification and matching: \\
%
{\tt{\small{public boolean unify(Term t)}}}\\
%
{\tt{\small{public boolean match(Term t)}}}\\
%
Terms that have been subject to unification outside a
demonstration context (that is invoking directly these methods,
and not passing through the solving service of an engine) should
not be used then in queries to engines.
%
\item some services are provided to compare terms, according to the
Prolog rules, and to check their type;
%
in particular the standard Java method \texttt{equals} has the
same semantics of the method \texttt{isEqual} which follows the
Prolog comparison semantics.
%
\item some services makes it possible to copy a term as it is
or to get a renamed copy of the term (\texttt{copy} and
\texttt{getRenamedCopy});
%
it is worth noting that the design of \tuprolog{} promotes a
stateless usage of terms; in particular, it is good practice not
to reuse the same terms in different demonstration contexts, as
part of different queries.
%
\item the method \texttt{getTerm} is useful in the case of
variables, providing the term linked possibly considering all the
linking chain in the case of variables referring other variables.
%
\item when a term is created by means of the proper constructor,
consider as example: \\\\
%
{\tt{\scriptsize{Struct myTerm = new Struct("p", new Var("X"), new
Int(1), new Var("X"))}}}\\\\
%
it is \emph{not resolved}, in the sense that possible variable
terms with the same name in the term do not refer each other;
%
so in the example the first and the third argument of the compound
\texttt{myTerm} point to different variable objects.
%
A term is resolved the first time it is involved in a matching or
unification context.
%
\end{itemize}

\noindent Some notes about \tuprolog{} engines, theories,
libraries and the services they provide:

\begin{itemize}
%
\item \tuprolog{} engines support natively some
\emph{directives}, that can be defined by means of the :-/1
predicate in theory specification.
%
Directives are used to specify properties of clauses and of the
engine (\emph{solve/1}, \emph{initialization/1},
\emph{set\_prolog\_flag/1}, \emph{load\_library/1},
\emph{consult/1}), format and syntax of read-terms (\emph{op/3},
\emph{char\_conversion/2}).
%
\item \tuprolog{} engines support natively the dynamic definition
and management of \emph{flags} (or property), used to describe
some aspects of libraries and their built-ins.
%
A flag is identified by a name (an alphanumeric atom), a list of
possible values, a default value and a boolean value specifying if
the flag value can be modified.
%
\item \tuprolog{} engines are thread-safe. The methods that could
create problems in being used in a multi-threaded context are now
synchronised.
%
\item \tuprolog{} engines have no (static) dependencies with each
other, multiple engines can be created independently as simple
objects on the same Java virtual machine,  each with its own
configuration (theory and loaded libraries).
%
Moreover, accordingly to the design of \tuprolog{} system in
general, engines are very lightweight, making suitable the use of
multiple engines in the same execution context.
%
\item \tuprolog{} engines can be serialised and stored as a persistent
object or sent through the network.
%
This is true also for engines with pre-loaded standard libraries:
%
in the case that other libraries are loaded, these must be
serializable in order to have the engine serializable.
%
\end{itemize}

%
%

\section{Some more examples of \tuprolog{} usage}

\noindent Creation of an engine (with default libraries
pre-loaded):

{\tt\scriptsize{
\begin{verbatim}
    import alice.tuprolog.*;

    ...
    Prolog engine = new Prolog();
\end{verbatim} }}


\noindent Creation of an engine specifying only the
\texttt{BasicLibrary} as pre-loaded library:

{\tt\scriptsize{
\begin{verbatim}
    import alice.tuprolog.*;

    ...
    Prolog engine = new Prolog(new String[]{"alice.tuprolog.lib.BasicLibrary"});
\end{verbatim} }}

\noindent Creation and loading of a theory from a string:

{\tt\scriptsize{\begin{verbatim}

    String theoryText = "my_append([],X,X).\n" +
                        "my_append([X|L1],L2,[X|L3]) :- my_append(L1,L2,L3).\n";

    Theory theory = new Theory(theoryText);
    try {
        engine.setTheory(theory);
    } catch(InvalidTheoryException e) {
    }
\end{verbatim} }}

\noindent Creation and loading of a theory from an input stream:

{\tt\scriptsize{\begin{verbatim}

    Theory theory = new Theory(new FileInputStream("test.pl");
    try {
        engine.setTheory(theory);
    } catch(InvalidTheoryException e) {
    }
\end{verbatim} }}

\noindent Goal demonstration (provided as a string):

{\tt\scriptsize{\begin{verbatim}

    // ?- append(X,Y,[1,2,3]).
    try {
        SolveInfo info = engine.solve("append(X,Y,[1,2,3]).");
        Term solution = info.getSolution();
    } catch(MalformedGoalException mge) {
        ...
    } catch(NoSolutionException nse) {
        ...
    }
\end{verbatim} }}

\noindent Goal demonstration (provided as a Term):

{\tt\scriptsize{\begin{verbatim}

    try {
        Term goal = new Struct("p", new Int(1), new Var("X"));
        try {
            // ?- p(1,X).
            SolveInfo info = engine.solve(goal);
            Term solution = info.getSolution();
 
        } catch (NoSolutionException nse) {
        }
    } catch (InvalidVarNameException ivne) {
    }
\end{verbatim} }}

\noindent Getting another solution:

{\tt\scriptsize{\begin{verbatim}
    try {
        SolveInfo info = engine.solve(goal);
        info = engine.solveNext();
    } catch(NoMoreSolutionException e)
\end{verbatim} }}

\noindent Loading a library:

{\tt\scriptsize{\begin{verbatim}
    try {
        engine.loadLibrary('alice.tuprologx.lib.TucsonLibrary');
    } catch(InvalidLibraryException e) {
    }
\end{verbatim} }}

\noindent Here, a complete example of interaction with a
\tuprolog{} engine is shown (refer to the JavaDoc documentation for
details about interfaces):

{\tt\scriptsize{\begin{verbatim}

import alice.tuprolog.*; import java.io.*;

public class Test2P {
    public static void main (String args[]) {
        Prolog engine = new Prolog();
        try {

            // solving a goal
            SolveInfo info = engine.solve(new Struct("append",
                                              new Var("X"),
                                              new Var("Y"),
                                              new Struct(new Term[]{new Struct("hotel"),
                                                                    new Int(303),
                                                                    new Var()})));


            // note we could use strings:
            // SolveInfo info = engine.solve("append(X, Y, [hotel, 303, _]).");

            // test for demonsration success
            if (info.isSuccess()) {

                // acquire solution and substitution
                Term sol = info.getSolution();
                System.out.println("Solution: " + sol);

                System.out.println("Bindings: " + info);

                // open choice points?
                if (engine.hasOpenAlternatives()) {

                    // ask for another solution
                    info = engine.solveNext();

                    if (info.isSuccess()) {
                        System.out.println("An other substitution: " + info);
                    }
                }
            }

            // other frequent interactions

            // setting a new theory in the engine
            String theory = "p(X,Y) :- q(X), r(Y).\n" +
                            "q(1).\n" +
                            "r(1).\n" +
                            "r(2).\n";
            engine.setTheory(new Theory(theory));

            SolveInfo info2 = engine.solve("p(1,X).");
            System.out.println(info2);

            // retrieving the theory from a file
            FileOutputStream os=new FileOutputStream("test.pl");
            os.write(theory.getBytes());
            os.close();
            engine.setTheory(new Theory(new FileInputStream("test.pl")));
            info2 = engine.solve("p(X,X).");
            System.out.println(info2.getSolution());

        } catch (Exception ex) {
            ex.printStackTrace();
        }
    }
}
\end{verbatim}}}

With the program execution, the following string are displayed on
the standard output:

{\tt\small{

\begin{verbatim}
Solution: append([],[hotel,303,_],[hotel,303,_])
Bindings: Y /[hotel,303,_] X / []
An other substitution: Y / [303,_]  X / [hotel]
X / 1
p(1,1)
\end{verbatim}}}
%%=======================================================================
\chapter{How to Develop New Libraries}
\label{ch:howto-develop-libraries}
%=======================================================================

Libraries are \tuprolog{}'s way to achieve the desired characteristics
of minimality, dynamic configurability, and straightforward
Prolog-to-Java integration.
%
Libraries are reflection-based, and can be written both in Prolog
and Java: other languages may be used indirectly, via JNI (Java
Native Interface).
%
At the \tuprolog{} side, exploiting a library written in Java
requires no pre-declaration of the new built-ins, nor any other
special mechanism: all is needed is the presence of the
corresponding \texttt{.class} library file in the proper location
in the file system.

\section{Implementation details}

Syntactically, a library developed in Java must extend the base
abstract class \texttt{alice.tuprolog.Library}, provided within
the \tuprolog{} package, and define new \textit{predicates} and/or
\textit{evaluable functors} and/or \textit{directives} in the form
of methods, following a simple signature convention.
%
In particular, new predicates must adhere to the signature:
%
\begin{center}
\small\tt
    public boolean <\textit{pred name}>\_<\textit{N}>(\textit{T1} arg1,
\textit{T2} arg2, ...,\textit{Tn} argN)
\end{center}
%
while evaluable functors must follow the form:
%
\begin{center}
    \small\tt
    public Term <\textit{eval funct name}>\_<\textit{N}>(\textit{T1} arg1,
\textit{T2} arg2, ...,\textit{Tn} argN)
\end{center}
%
and directives must be provided with the signature:
%
\begin{center}
    \small\tt
    public void <\textit{dir name}>\_<\textit{N}>(\textit{T1} arg1,
\textit{T2} arg2, ..., \textit{Tn} argN)
\end{center}
%
where \textit{T1}, \textit{T2}, ... \textit{Tn} are \texttt{Term} or derived
classes, such as \texttt{Struct}, \texttt{Var}, \texttt{Long}, etc., defined in
the \tuprolog{} package, constituting  the Java counterparts of
the corresponding Prolog data types.
%
The parameters represent the arguments actually passed to the built-in
predicate, functor, or directive.

%
A library defines also a new piece of theory, which is collected
by the Prolog engine through a call to the library method \texttt{String
getTheory()}.
%
By default, this method returns an empty theory: libraries which need to
add a Prolog theory must override it accordingly.
%
Note that only the external representation of a library's theory is
constrained to be in \texttt{String} form; the internal implementation
can be freely chosen by the library designer. However, using a Java
\texttt{String} for wrapping a library's Prolog code guarantees
self-containment while loading libraries through remote mechanisms such
as RMI.

\begin{table}[h]
    %
    \caption{Predicate and functor definitions in Java and their use
    in a \tuprolog{} program.\labeltab{java-preds}}
    %
    \begin{center}{\small\tt
    \begin{tabular}{p{10cm}|p{3.25cm}}
     \hline
     & \\
    \textit{// sample library} & \textit{\% tuProlog test program}\\
     import alice.tuprolog.*; & \\
     & \\
    public class TestLibrary extends Library \{                            &
 test :-\\
    ~~\textit{// builtin functor sum(A,B)}                          & ~~N is sum(5,6),\\
    ~~public Term sum\_2(Number arg0, Number arg1)\{    & ~~println(N).\\
    ~~~~float   arg0 = arg0.floatValue();        & \\
    ~~~~float   arg1 = arg1.floatValue();        & \\
    ~~~~return new Float(arg0+arg1);                                & \\
    ~~\}                                                            & \\
    ~~\textit{// builtin predicate println(Message)}                & \\
    ~~public boolean println\_1(Term arg)\{                      & \\
    ~~~~System.out.println(arg);                                   & \\
    ~~~~return true;                                                & \\
    ~~\}                                                            & \\
    \}                                                              & \\
     & \\
     \hline
    \end{tabular}
    }\end{center}
\end{table}

\xt{java-preds} shows a couple of examples about how a predicate
(such as \texttt{println/1}) and an evaluable functor (such as
\texttt{sum/2}) can be defined in Java and exploited from
\tuprolog{}.
%
The Java method \texttt{sum\_2}, which implements the evaluable
functor \texttt{sum/2}, is passed two \texttt{Number} terms (5 and 6)
which are then used (via \texttt{getTerm}) to retrieve the two
(float) arguments to be summed.
%
In the same way, method \texttt{println\_1}, which implements the
predicate \texttt{println/1}, receives \texttt{N} as \texttt{arg},
and retrieves its actual value via \texttt{getTerm}: since this is
a predicate, a boolean value (\texttt{true} in this case) is returned.
%

The developer of a library may face two corner case as far as method
naming is concerned: the first happens when the name of the
predicate, functor or directive she is defining contains a symbol
which cannot legally appear in a Java method's name; the second
occurs when he has to define a predicate and a directive with the
same Prolog signature, which Java would not be able to tell apart
because it cannot distinguish signatures of methods differing for
their return type only.
%
To overcome this kind of issues, a {\em synonym map} can be
constructed under the form of an array of \texttt{String} arrays,
and returned by the appropriate \texttt{getSynonymMap} method,
defined as abstract by the \texttt{Library} class. In both the cases
described above, another name must be chosen for the Prolog
executable element the library's developer want to define: then, by
means of the synonym map, that fake name can be associated with the
real name and the type of the element, be it a predicate, a functor
or a directive.
%
For example, if a definition for an evaluable functor representing
addition is needed, but the symbol \texttt{+} cannot appear in a
Java method's name, a method called \texttt{add} can be defined and
associated to its original Prolog name and its function by inserting
the array \texttt{\{"+", "add", "functor"\}} in the synonym map.

\section{Library Name}

% Lib name
%
By default, the name of the library coincides with the full class name of the
class implementing it.
%
However, it is possible to define explicitly the name of a library by
overriding the \texttt{getName} method, and returning as a string
the real name.
%
For example:
%
\begin{verbatim}
package acme;
import alice.tuprolog.*;
public class MyLib_ver00 extents Library {
    public String getName(){
        return "MyLibrary";
    }
    ...
}
\end{verbatim}

This class defines a library called \texttt{MyLibrary}.
%
It can be loaded into a Prolog engine by using the
\texttt{loadLibrary} method on the Java side, or a
\texttt{load\_library} built-in predicate on the Prolog side,
specifying the full class name (\texttt{acme.MyLib\_{ve00}}).
%
It can be unloaded then dynamically using the \texttt{unloadLibrary}
method (or the corresponding \texttt{unload\_library} built-in),
specifying instead the \textit{library name} (\texttt{MyLibrary}).

%%=======================================================================
\chapter{\tuprolog{} Exceptions}
ss\label{ch:exceptions}
%=======================================================================

%=======================================================================
\section{Exceptions in ISO Prolog}
\label{sec:exceptions in ISO prolog}
%=======================================================================
Exception handling was first introduced in the ISO Prolog standard (ISO/IEC 13211-1) in 1995.

The first distinction has to be made between \textit{errors} and \textit{exceptions}.
%
An \textit{error} is a particular circumstance that interrupts the execution of a Prolog program: when a Prolog engine encounters an error, it raises an \textit{exception}.
%
The exception handling support is supposed to intercept the exception and transfer the execution flow to a suitable exception handler, with any relevant information. Two basic principles are followed during this operation:

\begin{itemize}
  \item \textit{error bounding} -- an error must be bounded and not propagate through the entire program: in particular, an error occurring inside a given component must either be captured at the component's frontier, or remain invisible and be reported nicely.
      According to ISO Prolog, this is done via the \texttt{catch/3} predicate.

  \item \textit{atomic jump} -- the exception handling mechanism must be able to exit atomically from any number of nested execution contexts. According to ISO Prolog, this is done via the \texttt{throw/1} predicate.
\end{itemize}
%
In practice, the \texttt{catch(\textit{Goal}, \textit{Catcher}, \textit{Handler})} predicate enables the controlled execution of a goal, while the \texttt{throw(\textit{Error})} predicates makes it possible to raise an exception---very much like the \texttt{try/catch} construct of many imperative languages.

Semantically, executing the \texttt{catch(\textit{Goal}, \textit{Catcher}, \textit{Handler})} means that \texttt{\textit{Goal}} is first executed: if an error occurs, the subgoal where the error occurred is replaced by the corresponding \texttt{throw(\textit{Error})}, which raises the exception.
%
Then, a matching \texttt{catch/3} clause -- that is, a clause whose second argument
unifies with \texttt{\textit{Error}} -- is searched among the ancestor nodes in the resolution tree: if one is found, the path in the resolution tree is cut, the catcher itself is removed (because it only applies to the protected goal, not to the handler), and the \texttt{\textit{Handler}} predicate is executed. If, instead, no such matching clause is found, the execution simply fails.

So, \texttt{catch(\textit{Goal}, \textit{Catcher}, \textit{Handler})} performs exactly like \texttt{\textit{Goal}} if no exception are raised: otherwise, all the choicepoints generated by \texttt{\textit{Goal}} are cut, a matching \texttt{\textit{Catcher}} is looked for, and if one is found \texttt{\textit{Handler}} is executed, maintaining the substitutions made during the previous unification process.
%
Then, execution continues with the subgoal following \texttt{catch/3}.
%
Any side effects possibly occurred during the execution of a goal are \textit{not} undone in case of exceptions---as it normally happens when a predicate fails.

Summing up, \texttt{catch/3} succeeds if:
\begin{itemize}
  \item \texttt{call(\textit{Goal})} succeeds \textit{(standard behaviour)};\\\\
        --OR--
  \item \texttt{call(\textit{Goal})} is interrupted by a call to
      \texttt{throw(\textit{Error})} whose \texttt{\textit{Error}} unifies with
      \texttt{\textit{Catcher}}, and the subsequent \texttt{call(\textit{Handler})} succeeds.
\end{itemize}

\noindent If \texttt{\textit{Goal}} is non-deterministic, it can be executed again in
backtracking. However, since all the choicepoints of \texttt{\textit{Goal}} are cut in case of exception, \texttt{\textit{Handler}} \textit{is possibly executed just once}.

\smallskip

\noindent As an example, let us consider the following toy program:
\begin{verbatim}
    p(X):- throw(error), write('---').
    p(X):- write('+++').
\end{verbatim}

\noindent with the following query:

\begin{verbatim}
    ?:- catch(p(0), E, write(E)), fail.
\end{verbatim}
which tries to execute \texttt{p(0)}, catching any exception \texttt{E} and handling the error by just printing it on the standard output (\texttt{write(E)}).

Perhaps surprisingly, the program will just print \texttt{'error'}, not \texttt{'error---'} or \texttt{'error+++'}. The reason is that once the exception is raised, the execution of \texttt{p(X)} is aborted, and after the handler terminates the execution proceeds with the subgoal following \texttt{catch/3}, i.e. \texttt{fail}.
So, \texttt{write('---')} is never reached, nor is \texttt{write('+++')} since all the choicepoints are cut upon exception.

%-----------------------------------------------------------------------
\subsection{Error classification}
%-----------------------------------------------------------------------
This classification was already presented in Section \ref{sec:exception-support} above as a hint to predicate and functor readability: however, we report it here too both for completeness and for the reader's convenience.

When an exception is raised, the relevant error information is also transferred by instantiating a suitable \textit{error term}.

The ISO Prolog standard prescribes that such a term follows the pattern
\texttt{error(\textit{Error\_term}, \textit{Implementation\_defined\_term})} where
\texttt{\textit{Error\_term}} is constrained by the standard to a pre-defined set of values (the error categories), and \texttt{\textit{Implementation\_defined\_term}} is an optional term providing implementation-specific details.
%
Ten error categories are defined:
\begin{enumerate}
  \item \texttt{instantiation\_error}: when the argument of a predicate or one of its components is an unbound variable, which should have been instantiated. Example: \texttt{X is Y+1} when \texttt{Y} is not instantiated at the time \texttt{is/2} is evaluated.

  \item \texttt{type\_error(\textit{ValidType}, \textit{Culprit})}: when the type of an argument of a predicate, or one of its components, is instantiated, but is bound to the wrong type of data. \texttt{\textit{ValidType}} represents the expected data type (one of \texttt{atom}, \texttt{atomic}, \texttt{byte}, \texttt{callable}, \texttt{character}, \texttt{evaluable}, \texttt{in\_byte}, \texttt{in\_character}, \texttt{integer}, \texttt{list}, \texttt{number}, \texttt{predicate\_indicator}, \texttt{variable}), and \texttt{\textit{Culprit}} is the actual (wrong) type found.
      Example: a predicate expecting months to be represented as integers in the range 1--12 called with an argument like \texttt{march} instead of \texttt{3}.

  \item \texttt{domain\_error(\textit{ValidDomain}, \textit{Culprit})}: when the argument type is correct, but its value falls outside the expected range.
      \texttt{\textit{ValidDomain}} is one of \texttt{character\_code\_list},
      \texttt{not\_empty\_list}, \texttt{not\_less\_than\_zero}, \texttt{close\_option}, \texttt{io\_mode}, \texttt{operator\_priority}, \texttt{operator\_specifier}, \texttt{flag\_value}, \texttt{prolog\_flag}, \texttt{read\_option}, \texttt{write\_option}, \texttt{source\_sink}, \texttt{stream}, \texttt{stream\_option}, \texttt{stream\_or\_alias}, \texttt{stream\_position},\\
      \texttt{stream\_property}. Example: a predicate expecting months as above, called with an out-of-range argument like \texttt{13}.

  \item \texttt{existence\_error(\textit{ObjectType}, \textit{ObjectName}}): when the referenced object does not exist. \texttt{\textit{ObjectType}} is
      the type of the unexisting object (one of \texttt{procedure}, \texttt{source\_sink}, or \texttt{stream}), and \texttt{\textit{ObjectName}} is the missing object's name. Example: trying to access an unexisting file like \texttt{usr/goofy} leads to an
      \texttt{existence\_error(stream, 'usr/goofy')}.

  \item \texttt{permission\_error(\textit{Operation}, \textit{ObjectType}, \textit{Object})}: whenever\\
       \texttt{\textit{Operation}} (one of \texttt{access}, \texttt{create}, \texttt{input}, \texttt{modify}, \texttt{open}, \texttt{output}, or \texttt{reposition}) is not allowed on \texttt{\textit{Object}}, of type \texttt{\textit{ObjectType}} (one of  \texttt{binary\_stream}, \texttt{past\_end\_of\_stream}, \texttt{operator}, \texttt{private\_procedure}, \texttt{static\_procedure}, \texttt{source\_sink}, \texttt{stream}, \texttt{text\_stream}, \texttt{flag}).

  \item \texttt{representation\_error(\textit{Flag})}: when an implementation-defined limit, whose category is given by \texttt{\textit{Flag}} (one of
      \texttt{character}, \texttt{character\_code}, \texttt{in\_character\_code}, \texttt{max\_arity}, \texttt{max\_integer}, \texttt{min\_integer}), is violated during execution.

  \item \texttt{evaluation\_error(\textit{Error})}: when the evaluation of a function produces an out-of-range value (one of \texttt{float\_overflow}, \texttt{int\_overflow}, \texttt{undefined}, \texttt{underflow}, \texttt{zero\_divisor}).

  \item \texttt{resource\_error(\textit{Resource})}: when the Prolog engine does not have enough resources to complete the execution of the goal. \texttt{Resource} can be any term useful to describe the situation. Examples: maximum number of opened files reached, no further available memory, etc.

  \item \texttt{syntax\_error(\textit{Message})}: when data read from an external source have an incorrect format or cannot be processed for some reason. \texttt{\textit{Message}} can be any term useful to describe the situation.

  \item \texttt{system\_error}: any other unexpected error not falling into the previous categories.
\end{enumerate}

%=======================================================================
\section{Exceptions in \tuprolog}
\label{sec:exceptions in tuprolog}
%=======================================================================

\tuprolog{} aims to fully comply to ISO Prolog exceptions.
%
In the following, a set of mini-examples are presented which highlight each one single aspect of \tuprolog{} compliance to the ISO standard.

%-----------------------------------------------------------------------
\subsection{Examples}
%-----------------------------------------------------------------------

\medskip\noindent
\textit{\textbf{Example 1:} \texttt{Handler} must be executed maintaining the substitutions made during the unification process between \texttt{Error} and \texttt{Catcher}}

Program: \texttt{p(0) :- throw(error).}

Query: \texttt{ ?- catch(p(0), E, atom\_length(E, Length)).}

Answer: \texttt{ yes.}

Substitutions: \texttt{E/error}, \texttt{Length/5}


\medskip\noindent
\textit{\textbf{Example 2:} the selected \texttt{Catcher} must be the nearest in the resolution tree whose second argument unifies with \texttt{Error}}

Program: \texttt{p(0) :- throw(error).}\\
\mbox{\texttt{~~~~~~~~~~~}}\texttt{p(1).}

Query: \texttt{ ?- catch(p(1), E, fail),  catch(p(0), E, true).}

Answer: \texttt{ yes.}

Substitutions: \texttt{E/error}


\medskip\noindent
\textit{\textbf{Example 3:} execution must fail if an error occurs during a goal execution and there is no matching \texttt{catch/3} predicate whose second argument unifies with \texttt{Error}}

Program: \texttt{p(0) :- throw(error).}

Query: \texttt{ ?- catch(p(0), error(X), true).}

Answer: \texttt{ no.}


\medskip\noindent
\textit{\textbf{Example 4:} execution must fail if \texttt{Handler} is false}

Program: \texttt{p(0) :- throw(error).}

Query: \texttt{ ?- catch(p(0), E, false).}

Answer: \texttt{ no.}


\medskip\noindent
\textit{\textbf{Example 5:} if \texttt{Goal} is non-deterministic, it is executed again on backtracking, but in case of exception all the choicepoints must be cut, and \texttt{Handler} must be executed only once}

Program: \texttt{p(0).}\\
\mbox{\texttt{~~~~~~~~~~~}}\texttt{p(1) :- throw(error).}\\
\mbox{\texttt{~~~~~~~~~~~}}\texttt{p(2).}

Query: \texttt{ ?- catch(p(X), E, true).}

Answer: \texttt{ yes.}

Substitutions: \texttt{X/0}, \texttt{E/error}

Choice: \texttt{ Next solution?}

Answer: \texttt{ yes.}

Substitutions: \texttt{X/1}, \texttt{E/error}

Choice: \texttt{ Next solution?}

Answer: \texttt{ no.}


\medskip\noindent
\textit{\textbf{Example 6:} execution must fail if an exception occurs in \texttt{Handler}}

Program: \texttt{p(0) :- throw(error).}

Query: \texttt{ ?- catch(p(0), E, throw(err)).}

Answer: \texttt{ no.}

%-----------------------------------------------------------------------
\subsection{Handling Java/.NET Exceptions from \tuprolog}
\label{ssec:java-exceptions-in-tuprolog}
%-----------------------------------------------------------------------

One peculiar aspect of \tuprolog{} is the ability to support multi-paradigm programming, mixing object-oriented (mainly, but not exclusively, Java) and Prolog in several ways---in particular, by enabling Java objects to be accessed and exploited from Prolog world via OOLibrary (see Section \ref{sec:java-library}) and by enabling .NET objects to be accessed and exploited from Prolog world via OOLibrary (see Section \ref{sec:dotnet-oolibrary})
%
In this context, the problem arises of properly sensing and handling Java/.NET exceptions from the Prolog side.

At a first sight, one might think of re-mapping such exceptions and constructs onto the Prolog ones, but this approach is unsatisfactory for three reasons:
%
\begin{itemize}
  \item the semantics of the Java/.NET mechanism should not be mixed with the Prolog one, and vice-versa;

  \item the Java/.NET construct admits also a \texttt{finally} clause which has no counterpart in ISO Prolog exceptions;

  \item the Java/.NET catching mechanisms operates hierarchically, while the \texttt{catch/3} predicate operates via pattern matching and unification, allowing for a finer-grain, more flexibly exception filtering.
\end{itemize}

%\noindent Accordingly, Java/.NET exceptions in \tuprolog{} programs are handled by means of two further, \textit{ad hoc} predicates: \texttt{java\_throw/1} and \texttt{java\_catch/3} in the Java case, and \texttt{oo\_throw/1} and \texttt{oo\_catch/3} in the .NET case, respectively.
%%
%Since their behavior can be fully understood only in the context of JavaLibrary/OOLibrary, we forward the reader to Sections \ref{sec:java-library} and \ref{sec:dotnet-oolibrary}, respectively, for further information.
%
\noindent Accordingly, Java/.NET exceptions in \tuprolog{} programs are handled by means of an \textit{ad hoc} predicate, called \texttt{java\_catch/3} in the Java case and \texttt{oo\_catch/3} in the .NET case, respectively.
%
Since their behavior can be fully understood only in the context of OOLibrary, we forward the reader to Sections \ref{sec:java-library} and \ref{sec:dotnet-oolibrary}, respectively, for further information.

%%---------------------------------------------------------------------------------------
%\section*{Appendix: Implementation notes}
%%---------------------------------------------------------------------------------------
%
%Implementing exceptions in \tuprolog{} does not mean just to extend the engine to support the above mechanisms: given its library-based design, and its intrinsic support to multi-paradigm programming, adding exceptions in \tuprolog{} has also meant (1) to revise all the existing libraries, modifying any library predicate so that it raises the appropriate type of exception instead of just failing; and (2) to carefully define and implement a model to make Prolog exceptions not only coexist, but also fruitfully operate with the Java or .NET imperative world, which brings its own concept of exception and its own handling mechanism.
%
%As a preliminary step, the finite-state machine which constitutes the core of the \tuprolog{} engine was extended with a new \textit{Exception} state, between the existing \textit{Goal Evaluation} and \textit{Goal Selection} states \cite{iuliani-masterthesis-2009}.
%
%Then, all the \tuprolog{} libraries were revised, according to clearness and efficiency criteria --- that is, the introduction of the new checks required for proper exception raising should not reduce performance unacceptably. This issue was particularly relevant for runtime checks, such as \texttt{existence\_error}s or \texttt{evaluation\_error}s; moreover, since \tuprolog{} libraries could also be implemented partly in Prolog and partly in Java, careful choices had to be made so as to introduce such checks at the most adequate level in order to intercept all errors while maintaining code readability and overall organisation, while guaranteeing efficiency.
%
%This led to intervene with extra Java checks for libraries fully implemented in Java, and with new ''Java guards'' for predicates implemented in Prolog, keeping the use of Prolog meta-predicates (such as \texttt{integer/1}) to a minimum.
%
%\bigskip
%
%Per quel che riguarda il modo in cui \`{e} stato implementato il meccanismo di controllo degli errori, bisogna distinguere i predicati espressi in Java da quelli espressi in Prolog.
%
%Nel primo caso le eccezioni (cio\`{e} le opportune istanze di \texttt{PrologError}) vengono lanciate direttamente dai corrispondenti metodi Java ogniqualvolta si verifica un errore, mentre nel secondo caso sono lanciate da metodi ``guardia" (sempre espressi in Java) invocati per controllare i parametri prima dell'esecuzione del predicato Prolog.
%
%Nell'implementazione si \`{e} cercato di individuare il maggior numero possibile di condizioni di errore, rispettando per\`{o} sempre il requisito fondamentale di correttezza: se una chiamata a un predicato non falliva prima dell'introduzione del meccanismo delle eccezioni, non deve fallire neanche ora---ovvero, il lancio di una eccezione deve avvenire soltanto in circostanze in cui il motore tuProlog originario falliva.
%
%La correttezza del comportamento del motore \`{e} garantita anche se ci si dimentica di identificare qualche condizione di errore inaspettata: in questo caso infatti il motore non lancia un'eccezione, ma comunque fallisce.
%%
%Ci\`{o} permette ad un utente sia di gestire gli errori che si possono verificare durante l'esecuzione, sia di non gestirli, nel qual caso l'esecuzione fallir\`{a} e dunque l'estensione rester\`{a} trasparente.
%
%A parte le inevitabili modifiche ai built-in e alle librerie (\textit{BasicLibrary}, \textit{ISOLibrary}, \textit{IOLibrary}, \textit{DCGLibrary}), sono state necessarie le seguenti semplici modifiche al motore:
%
%\begin{itemize}
%
%\item alla classe \texttt{alice.tuprolog.FlagManager} sono stati aggiunti due metodi per ricavare informazioni su un flag:
%
%\begin{itemize}
%\item \texttt{boolean isModifiable(String name)}\\
%        che restituisce true se esiste nel motore un flag di nome \texttt{name}, e tale flag \`{e} modificabile;
%
%\item \texttt{boolean isValidValue(String name, Term Value)}\\
%        che restituisce true se esiste nel motore un flag di nome \texttt{name}, e \texttt{Value} \`{e} un valore ammissibile per tale flag.
%\end{itemize}
%
%  \item il metodo \texttt{getEngineManager} della classe \texttt{alice.tuprolog.Prolog} \`{e} ora pubblico (in precedenza aveva visibilit\`{a} di package) per permettere alle librerie di ricavare dal motore l'informazione sul goal correntemente in esecuzione e inserirla nell'eccezione lanciata;
%
%  \item il metodo \texttt{evalAsFunctor} della classe \texttt{alice.tuprolog.PrimitiveInfo} lancia ora un'istanza di \texttt{Throwable} in caso di errore durante la valutazione del goal, mentre prima ritornava \texttt{null}, per permettere di discriminare il tipo di errore verificatosi durante la valutazione di un funtore;
%
%  \item analogamente, il metodo \texttt{evalExpression} di \texttt{alice.tuprolog.Library} rilancia ora l'istanza di \texttt{Throwable} ricevuta dal metodo \texttt{evalAsFuntor} di \texttt{alice.tuprolog.PrimitiveInfo}.
%\end{itemize}


%\bibliography{bibliography/bibliography}
%******************************************************************************%
%=======================================================================
\chapter{What is \tuprolog{}}
\label{what-is}
%=======================================================================

\tuprolog{} is a light-weight Prolog framework for distributed applications and infrastructures.
%
\tuprolog{} is developed and maintained by the \alice{} research group\footnote{\url{http://www.alice.unibo.it}} at the \textsc{Alma Mater Studiorum}---Universit\`{a} di Bologna.
%
It is built as an Open Source software, released under the LGPL license -- thus allowing also for commercial derivative work --, and made available through the pages of the \apice{} web portal\footnote{\url{http://tuprolog.apice.unibo.it}}.

\tuprolog{} is designed to be \emph{minimal}, dynamically \emph{configurable}, \emph{interoperable}, straightforwardly \emph{integrated} with Java and .NET, and easily \emph{deployable}.

First of all, \tuprolog{} is designed with \textit{minimality} in mind.
%
Accordingly, \tuprolog{} core is a tiny Java object that contains only the essential properties of a Prolog engine.
%
Only the required Prolog features -- like, say, ISO compliance, I/O predicates, DCG operators -- are then to be added to or removed from a \tuprolog{} engine according to the contingent application needs.

The obvious counterpart of minimality is \tuprolog{} \textit{configurability}.
%
In fact, a simple yet powerful mechanism based on the notion of \tuprolog{} \textit{library} is provided tha allows required predicates, functors and operators to be loaded and unloaded in a \tuprolog{} engine, both statically and dynamically.
%
Libraries can be either included in the standard \tuprolog{} distribution, or defined \textit{ad hoc} by the \tuprolog{} user / developer. 

A \tuprolog{} library can be built in different ways. 
%
First of all, a \tuprolog{} library could be straightforwardly written in Prolog.
%
On the other hand, a \tuprolog{} library could also be implemented using either Java or any language of .NET framework---depending on the chosen \tuprolog{} implementation.
%
Finally, a \tuprolog{ library could be built by combining Prolog and Java / .NET languages, thus paving the way for multi-language / multi-paradigm integration.
%
Whatever the language(s) used, a \tuprolog{} library can be either used to configure a \tuprolog{} engine when this is started up, or loaded -- and then unloaded -- dynamically at any time during the engine execution.

\tuprolog{} was first implemented upon Java, then ported upon .NET, and is now available on both platforms.
%
\textit{Deployability} of \tuprolog{} owes a lot to Java and .NET.
%
On the Java side, the requirements for \tuprolog{} installation simply amount to the presence of a standard Java VM, and a Java invocation upon a single JAR file is everything needed to start a \tuprolog{} activity.
%
On the .NET side, \ldots ENRICO ENRICO ENRICO ENRICO

\tuprolog{} \textit{integration} with other languages and paradigms is kept as clean as possible, so that the components of a \tuprolog{} application can be developed by choosing at any step the most suitable paradigm---either declarative/logic or imperative/object-oriented.
%
On the Prolog side, thanks to the \texttt{JavaLibrary} library, any Java entity (object, class, package) can be represented as a Prolog term, and exploited from Prolog.
%
So, for instance, Java packages like Swing and {JDBC} can be directly used from within Prolog, straightforwardly enhancing \tuprolog{} with graphics and database access capabilities.
%
In the same way, \texttt{DotNetLibrary} \ldots o come cavolo si chiama ENRICO ENRICO ENRICO ENRICO
%
On the Java side, a \tuprolog{} engine can be invoked and used as a simple Java object, possibly embedded in beans, or exploited in a multi-threaded context, according to the application needs.
%
Also, a multiplicity of different \tuprolog{} engines can be used from a Java program at the same time, each one configured with its own libraries and knowledge base.
%
In the same way, sbrodolata .NET di ENRICO ENRICO ENRICO ENRICO


Interoperability misteriosa: cosa diciamo??? Ancora quanto segue???

Finally, \textit{interoperability} is developed along two main lines:
Internet standard patterns, and coordination models.
%
So, \tuprolog{} supports interaction via TCP/IP and RMI, and can be
also provided as a CORBA service.
%
In addition, \tuprolog{} supports tuple-based coordination under many
forms.
%
First, components of a \tuprolog{} application can be organised around
Java-based tuple spaces, logic tuple spaces, and \respect{} tuple
centres \cite{respect-scico2001}.
%
Then, \tuprolog{} applications can exploit Internet infrastructures
providing tuple-based coordination services, like \luce{}
\cite{luce-aamas2001} and \tucson{} \cite{tucson-aamas99}.



%******************************************************************************%
%=====================================================================
\chapter{Installing \tuprolog{}}
\label{installation}
%=====================================================================

Quite obviously, the installation procedure depends on the platform of choice.
For Java, Microsoft .NET and Android, the first step is to manually download the desired distribution (or even just the single binary file) from the \tuprolog{} web site, \texttt{tuprolog.alice.unibo.it}, or directly from the Google code repository, \texttt{tuprolog.googlecode.com}; for Eclipse the procedure is different, since the plug-in installation has to be performed via the Eclipse Plugin Manager.

As a further alternative, users wishing to have a look at \tuprolog{} and trying it without installing anything on their computer can do so by exploiting the `Run via Java Web Start' option, available on the \tuprolog{} web site.

\section{Installation in Java}

The complete Java distribution has the form of a single \texttt{zip} file which contains everything (binaries, sources, documentation, examples, etc.) and unzips into a multi-level directory tree, similar to the following (only first-level sub-dirs are shown):

\begin{Verbatim}[frame=single, framerule=0.5mm, samepage=true, boxwidth=5cm]
    2p
    |---ant
    |---bin
    |---build
    |   |---archives
    |   |---classes
    |   |---release
    |   |---reports
    |   |---tests
    |---doc
    |   |---javadoc
    |---lib
    |---src
    |---test
    |   |---fit
    |   |---unit
    |---tmp
    |---test
\end{Verbatim}

An alternative distribution, without sources, is also available in the \textit{Download} section of the \tuprolog{} repository: obviously, in this case only a subset of the above folders is present (namely, only \texttt{bin}, \texttt{doc}, \texttt{lib} and \texttt{reports}).

If you are only interested in the Java binaries, just look into the \texttt{build/archives} directory, which contains two JAR files:
%
\begin{itemize}
%
\item \texttt{2p.jar}, which contains everything you need to use \tuprolog{},
  such as the core API, the \texttt{Agent} application, libraries, GUI,
  etc.; this is a runnable JAR, that open the \tuprolog{} IDE when double-clicked.
%
\item \texttt{tuprolog.jar}, which contains only the core part of \tuprolog{},
  namely, what you will need to include in a Java application project to be able to access the \tuprolog{} classes, and write multi-paradigm Java/Prolog applications.
\end{itemize}

The other folders contain project-specific files: \texttt{src} contains all the sources, \texttt{doc} all the documentation, \texttt{lib} the libraries used by the \tuprolog{} project, \texttt{test} the sources for the \tuprolog{} test suite (partly as FIT test, partly as JUnit tests), \texttt{ant} some Ant scripts to automate the build of parts of the \tuprolog{} project, etc.


\section{Installation in .NET}

The complete .NET distribution has also the form of a single \texttt{zip} file containing everything; however, due to the automatic generation of \tuprolog{} .NET binaries via IKVM from Java (more on this in Chapter \ref{ch:mpp-in-dotnet}), the unzipped directory tree is simpler, as there are no sources (and therefore no tests, no ant tasks, etc), except for \texttt{OOLibrary} and Conventions, which are NET-specific and therefore written in C\#.
%
So, the resulting tree is similar to the following:
%
\begin{verbatim}
    2p
    |---build
    |   |---examples
    |   |---lib
    |---OOLibrary
    |   |---Conventions
    |   |---Fixtures
    |   |---OOLibrary
\end{verbatim}
%
Here, too, an alternative distribution, without the OOLibrary and conventions sources, is also available in the \textit{Download} section of the \tuprolog{} repository: again, only a subset of the above folders is present in this case.

The .NET binary, \texttt{2p.exe}, can be found in the \texttt{build} folder.


\section{Installation in Android}

The Android distribution has the form of a single \texttt{apk} file, to be installed via install mechanism provided by the Android OS.
So, unless you are interested in the implementation details, there should be no need to download the whole project distribution.
If, however, you like to do so, you will eventually get to a directory tree similar to the following (only the most relevant first-level sub-folders are shown):
%
\begin{verbatim}
    2p
    |---assets
    |---bin
    |   |---classes
    |   |---res
    |---doc
    |---gen
    |---libs
    |---res
    |---screenshots
    |---src
\end{verbatim}
%
The APK binary can be found into the \texttt{bin} folder.

As for the Java case, the other folders contain project-specific files: in particular, \texttt{src} contains the sources, \texttt{res} the Android resources automatically generated during the project build process, \texttt{libs} the libraries used by this project---mainly, the \texttt{tuprolog.jar} file of the corresponding Java version, imported here as an external dependency.


\section{Installation in Eclipse}

The installation procedure is different for the Eclipse platform due to the need to conform to the Eclipse standard procedure for plug-in installation via Plugin Manager.
%Please see the specific section on the \tuprolog{} web site for detailed, screenshot-driven instruction.
%
In order to exploit Eclipse's built-in plugin installation manager, a properly-configured \textit{update site}
must be added to the Eclipse Update Site List first.

To do so:

\begin{enumerate}
  \item open the Eclipse preferences (menu Window $>$ Preferences) and choose the Install/Update item
  and choose the Available Software Sites sub-item. You might want to type ``tuprolog'' in the text field 
  just to check that no other update sites are already defined for it.

  \item now click on the ``Add'' button to add a new software site: in the dialog that appears, type a 
  description in the upper field (e.g. ``tuProlog update site''), and enter the following URL in the lower field:\\
     {\footnotesize{\texttt{http://tuprolog.googlecode.com/svn/2p-plugin/trunk/tuPrologUpdateSite/}}}\\
  The dialog should now look as in Figure \ref{fig:tuPrologPluginInstall-12}.
  Clicking OK, you should now see the new site in the site list.

  \item close this window, and go back to the main Eclipse window. Open the Help menu and choose the 
  \textit{Install new software} item (Figure \ref{fig:tuPrologPluginInstall-34}, top).
  Select the \tuprolog{} software by typing ``tuProlog'' in the filter text field, or by scrolling the site 
  list: after selecting the site, you should see something like the window shown in 
  Figure \ref{fig:tuPrologPluginInstall-4}, bottom.

  \item Now select the \tuprolog{} feature by clicking on the checkbox. If multiple feature versions are 
  proposed (that depends whether you checked the ``show older versions'' option), choose the version you 
  prefer: if unsure, select the most recent.
  Once selected, click Next: installation will take place automatically.
\end{enumerate}

The \tuprolog{} plugin is now installed on your Eclipse system.

\begin{figure}
\centering
  \includegraphics[width=300px]{images/tuPrologPluginInstall-1.png}\\
  \includegraphics[width=300px]{images/tuPrologPluginInstall-2.png}
  \caption{Plugin installation: adding the Update Site, phase 1}\label{fig:tuPrologPluginInstall-12}
\end{figure}

\begin{figure}
\centering
  \includegraphics[width=300px]{images/tuPrologPluginInstall-3.png}\\
  \includegraphics[width=300px]{images/tuPrologPluginInstall-4.png}
  \caption{Plugin installation: adding the Update Site, phase 2}\label{fig:tuPrologPluginInstall-34}
\end{figure}

%******************************************************************************%
%=======================================================================
\chapter{Getting Started}
\label{getting-started}
%=======================================================================

The \tuprolog{} distribution offers some tools either to consult and execute already existing Prolog programs, or to help developing new Prolog theories and interact with a Prolog engine. %
Depending on the use you would like to make of \tuprolog{}, you may want to start exploring the distribution tools along different directions.

%=======================================================================
\section{Prolog Programmer Quick Start}
%=======================================================================

As a Prolog programmer, you would like to start trying \tuprolog{} by running your
already existing Prolog programs. You can execute your programs in the form of source
text files using the \tuprolog{} Agent tool. This tool accepts as arguments the name of
a text file containing a Prolog theory and, optionally, the goal to be solved; then it starts
the demonstration. Once you have properly installed \tuprolog{} in the \emph{dir}
directory, you can use the following template to invoke the Agent tool from the
command line:\\\\
%
\texttt{java -cp \emph{dir}/2p.jar\\
\mbox{~~~~~~~~~}alice.tuprolog.Agent \textit{PrologTextFile}
\{\textit{Goal}\}\\\\}
%
For instance, suppose a text file named \verb|hello.pl| in your current directory contains the following line:
\begin{verbatim}
go :- write('hello, world!'), nl.
\end{verbatim}
In order to execute this Prolog program, you can type at the command prompt:\\\\
%
\texttt{java -cp \emph{dir}/2p.jar alice.tuprolog.Agent hello.pl go.\\\\}
%
Then, the Agent tool tries to prove the goal \texttt{go} with respect to the theory contained in \texttt{hello.pl}. 
%
As a result, the string \texttt{hello, world!} should appear on your standard output.

Also, the goal to be proven can be embedded within the Prolog source by means of the \texttt{solve} directive.
%
For instance, suppose that the text file \texttt{hellogo.pl} in your current directory contains the following lines:
\begin{verbatim}
:- solve(go).
go :- write('hello, world!'), nl.
\end{verbatim}
Then, type:\\\\
%
\texttt{java -cp \emph{dir}/2p.jar alice.tuprolog.Agent hellogo.pl\\\\}
%
Again, this will make \texttt{hello, world!} appear on your standard output.

%=======================================================================
\section{Developer Quick Start}
%=======================================================================

The first thing you may want to do as a developer would probably be to take advantage
of the tools embedded in the Graphical User Interface included in the \tuprolog{}
distribution. The GUI can be obtained by issuing the following command:\\\\
%
\texttt{java -cp \emph{dir}/2p.jar alice.tuprologx.ide.GUILauncher\\\\}
%
The development environment provided by the GUI makes standard Prolog features
easily accessible, such as asking queries, viewing the current solution along
with the related variable substitution, backtracking, and so on. Also, it
enables you to view and edit the current Prolog theory contained in the engine,
and to spy \tuprolog{} at work during goal demonstrations. Finally, it also
offers a facility to dynamically load and unload predicate libraries within the
\tuprolog{} engine.

It is worth remembering that the file \texttt{2p.jar} is an executable Java Archive, so
by invoking the command:\\\\
%
\texttt{java -jar 2p.jar\\\\}
%
in the \emph{dir} directory, or by double-clicking it under most operating systems, the
graphic user interface console is automatically spawned.

You may also want to experience a pure interactive environment on a \tuprolog{}
engine. In this case, you need to get the \tuprolog{} prompt using the command line
shell provided within the distribution. To obtain it, just type:\\\\
%
\texttt{java -cp \emph{dir}/2p.jar alice.tuprologx.ide.CUIConsole\\\\}
%
\noindent which starts a \tuprolog{} interpreter to be used via console, in a sort of
Command Line User Interface mode. To exit the \tuprolog{} console, you have to
issue the standard \texttt{halt.} command.

%******************************************************************************%
%=======================================================================
\chapter{\tuprolog{} Basics}
\label{ch:engine}
%=======================================================================
%
% General stuff about 2P engine features
%
%

\noindent This chapter provides a brief introduction to the basic elements and structure of the \tuprolog{} engine, covering syntax, programming support, and built-in predicates directly provided by the engine.

%---------------------------------------------------------------------
\section{Structure of a \tuprolog{} Engine}
%---------------------------------------------------------------------

\noindent A \tuprolog{} engine has a layered structure, where provided and recognised predicates are organised into three different categories:
%
\begin{description}
\item[built-in predicates] |
Predicates embedded in any \tuprolog{} engine are called built-in predicates.
%
Whatever modification is made to the engine either before or during execution time, it does not affect the number and properties of the built-in predicates.
%
\item[library predicates] |
Predicates loaded in a \tuprolog{} engine by means of a \tuprolog{} library are called library predicates.
%
Since libraries can be loaded and unloaded in \tuprolog{} engines freely at the system start-up, or dynamically at execution time, the set of the library predicates of a \tuprolog{} engine is not fixed, and can change from engine to engine, and in the same engine at different times.
%
\tuprolog{} libraries can be built by mixing Java and Prolog code. Prolog
library predicates can be overridden by Prolog theory predicates. Both Java and
Prolog library predicates cannot be individually retracted: if you want to
remove a single library predicate from the engine, you need to unload the whole
library containing that predicate.
\item[theory predicates] |
Predicates loaded in a \tuprolog{} engine by means of a \tuprolog{} theory are called theory predicates.
%
Since theories can be loaded and unloaded in \tuprolog{} engines freely at the system start-up, or dynamically at execution time, the set of the theory predicates of a \tuprolog{} engine is not fixed, and can change from engine to engine, and in the same engine at different times. 
%
\tuprolog{} theories are simple collections of Prolog clauses.
\end{description}
%
Even though they may seem similar, library and theory predicates are handled differently in a \tuprolog{} engine.

First of all, they are conceptually different. In fact, while theory predicates should be used to axiomatically represent domain knowledge at the time the proof is performed, library predicates should more or less be used to represent what is required (procedural knowledge, utility predicates) in order to actually and effectively perform a (number of) proof(s) in the domain of interest: therefore, library predicates represent more ``stable'' knowledge, which is encapsulated once and for all (at least approximately) within a library container.

Since library and theory predicates are also structurally different, they are handled differently by the engine, and represented differently in the run-time: correspondingly, they have different level of observation when monitoring or debugging a working \tuprolog{} engine.
%
As a consequence, developer tools provided by \tuprolog{} IDE typically show in a separate way the theory axioms or predicates and the loaded libraries or predicates.
%
In addition, the debugging phase typically neglects library predicates (which, as mentioned above, are also conceived as more stable and well-tested), while the effect of the theory predicates is dutifully put in evidence during controlled execution.

%---------------------------------------------------------------------
\section{Prolog syntax}
%---------------------------------------------------------------------

\noindent The term syntax supported by \tuprolog{} engine is basically ISO compliant,\footnote{Currently ISO exceptions, ISO I/O predicates and some ISO directives are not supported.}
and accounts for several elements:
%
\begin{description}
\item[Comments and Whitespaces] -- Whitespaces consist of blanks (including tabs and formfeeds), end-of-line marks, and comments. A whitespace can be put before and after any term, operator, bracket, or argument separator, as long as it does not break up an atom or number or separate a functor from the opening parenthesis that introduces its argument lists.
%
For instance, atom \bt{p(a,b,c)} can be written as \bt{p(\mbox{~a~},\mbox{~b~},\mbox{~c~})}, but not as \bt{\mbox{p~}(a,b,c)}).
%
Two types of comments are supported: one type begins with \bt{/*} and ends with \bt{*/}, the other begins with \bt{\%} and ends at the end of the line.
%
Nested comments are not allowed.
%
\item[Variables] |
A variable name begins with a capital
letter or the underscore mark (\bt{\_}), and consists of letters,
digits, and/or underscores.
%
A single underscore mark denotes an anonymous variable.
%
\item[Atoms] |
There are four types of atoms:
\emph{(i)} a series of letters, digit, and/or underscores, beginning with a lower-case letter; \emph{(ii)} a series of one or more characters from the set \{\texttt{\#}, \texttt{\$}, \texttt{\&}, \texttt{*}, \texttt{+}, \texttt{-}, \texttt{.}, \texttt{/}, \texttt{:}, \texttt{<}, \texttt{=}, \texttt{>}, \texttt{?}, \texttt{@}, \texttt{\textasciicircum}, \texttt{\~}\}, provided it does not begin with \texttt{/*};
\emph{(iii)} The special atoms \texttt{[]} and \texttt{\{\}};
\emph{(iv)} a single-quoted string.
%
\item[Numbers] |
Integers and float are supported.
%
The formats supported for integer numbers are decimal, binary (with \verb|0b|
prefix), octal (with \verb|0o| prefix), and hexadecimal (with \verb|0x|
prefix). The character code format for integer numbers (prefixed by \verb|0'|) is supported only for alphanumeric characters, the white space, and characters in the set \{\texttt{\#}, \texttt{\$}, \texttt{\&}, \texttt{*}, \texttt{+}, \texttt{-}, \texttt{.}, \texttt{/}, \texttt{:}, \texttt{<}, \texttt{=}, \texttt{>}, \texttt{?}, \texttt{@}, \texttt{\textasciicircum}, \texttt{\~}\}.
%
The range of integers is -2147483648 to 2147483647; the range of floats is
-2E+63 to 2E+63-1.
%
Floating point numbers can be expressed also in the exponential format (e.g. \bt{-3.03E-05}, \bt{0.303E+13}).
%
A minus can be written before any number to make it negative (e.g. \bt{-3.03}).
%
Notice that the minus is the sign-part of the number itself; hence \bt{-3.4} is a number, not an expression (by contrast, \bt{- 3.4} is an expression).
%
\item[Strings] |
A series of ASCII characters, embedded in quotes \verb|'| or \verb|"|.
%
Within single quotes, a single quote is written double (e.g, \verb|'don''t forget'|).
%
A backslash at the very end of the line denotes continuation to the next line, so that: \\
\verb|'this is \ |\\
\verb|a single line'|\\
is equivalent to \verb|'this is a single line'| (the line break is ignored).
%
Within a string, the backslash can be used to denote special characters, such as \verb|\n| for a newline,
\verb|\r| for a return without newline,
\verb|\t| for a tab character,
\verb|\\| for a backslash,
\verb|\'| for a single quote,
\verb|\"| for a double quote.
%
\item[Compounds] |
The ordinary way to write a compound is to write the functor (as an atom), an opening parenthesis, without spaces between them, and then a series of terms separated by commas, and a closing parenthesis: \bt{f(a,b,c)}.
%
This notation can be used also for functors that are normally written as operators, e.g. \bt{2+2} = \verb|'+'(2,2)|.
%
Lists are defined as rightward-nested structures using the dot operator \verb|'.'|; so, for example: \\
\bt{[a] =} \verb|'.'(a,[])|\\
\bt{[a,b] =} \verb|'.'(a,'.'(b,[]))|\\
\bt{[a,b|c] =} \verb|'.'(a,'.'(b,c))|\\
%
There can be only one \bt{|} in a list, and no commas after it.
%
Also curly brackets are supported: any term enclosed with \bt{$\{$} and \bt{$\}$} is treated as the argument of the special functor \verb|'{}'|:  \verb|{hotel}| = \verb|'{}'(hotel)|, \bt{$\{$1,2,3$\}$} = \verb|'{}'(1,2,3)|.
%
Curly brackets can be used in the Definite Clause Grammars theory.

\item[Operators] |
Operators are characterised by a name, a specifier, and a priority.
%
An operator name is an atom, which is not univocal: the same atom can be an operator in more than one class, as in the case of the infix and prefix minus signs.
%
An operator  specifier is a string like \texttt{xfy}, which gives both its class (infix, postfix and prefix) and its associativity: \texttt{xfy} specifies that the grouping on the right should be formed first, \texttt{yfx} on the left, \texttt{xfx} no priority.
%
An operator priority is a non-negative integer ranging from 0 (max priority) and 1200 (min priority).

Operators can be defined by means of either the \bt{op/3} predicate or directive.
%
No predefined operators are directly given by the raw \tuprolog{} engine, whereas a number of them is provided through libraries.
%
\item[Commas] |
The comma has three functions: it separates arguments of functors, it separates elements of lists, and it is an infix operator of priority 1000.
%
Thus \bt{(a,b)} (without a functor in front) is a compound, equivalent to \verb|','(a,b)|.
%
\item[Parenthesis] -- Parenthesis are allowed around any term.
%
The effect of parenthesis is to override any grouping that may
otherwise be imposed by operator priorieties.
%
Operators enclosed in parenthesis do not function as operators;
thus \bt{2(+)3} is a syntax error.
\end{description}

%---------------------------------------------------------------------
\section{Configuration of a \tuprolog{} Engine}
%---------------------------------------------------------------------
\noindent Prolog developers have four different means to configure a \tuprolog{} engine in order to fit their application needs.
%
In fact, a \tuprolog{} can be suitably configured by means of:

\begin{description}
%
\item[Theories] |
A \tuprolog{} theory is represented by a text, consisting of a sequence of clauses and/or directives.
%
Clauses and directives are terminated by a dot, and are separated by a whitespace character.
%
Theories can be loaded or unloaded by means of suitable library predicates, which are described in Chapter \ref{ch:standard-libraries}.
%
\item[Directives] |
A directive can be given by means of the \bt{:-/1} predicate, which is natively supported by the engine, and can be used to configure and use a \tuprolog{} engine (\bt{set\_prolog\_flag/1}, \bt{load\_library/1}, \bt{consult/1}, \bt{solve/1}), format and syntax of read-terms\footnote{As specified by the ISO standard, a read-term is a Prolog term followed by an end token, composed by an optional layout text sequence and a dot.} (\bt{op/3}).
%
Directives are described in detail in the following sections.
%
\item[Flags] |
A \tuprolog{} engine allows the dynamic definition of flags (or properties) describing some aspects of libraries and their predicates and evaluable functors.
%
A flag is identified by a name (an alphanumeric atom), a list of possible values, a default value, and a boolean value specifying if the flag value can be modified.
%
Dynamically, a flag value can be changed (if modifiable) with a new value included in the list of possible values.
%
\item[Libraries] |
A \tuprolog{} engine can be dynamically extended by loading or unloading libraries.
%
Each library can provide a specific set of predicates, functors, and a related theory, which also allows new flags and operators to be defined.
%
Libraries can be either pre-defined (see Chapter \ref{ch:standard-libraries}) or user-defined (see Chapter \ref{ch:howto-develop-libraries}).
%
A library can be loaded by means of the predicate \texttt{load\_library} (Prolog side), or by means of the method \texttt{loadLibrary} of the \tuprolog{} engine (Java side).
\end{description}
%
Currently \tuprolog{} does not support exception management: actually, an exception causes the predicate/functor in which it occurred to fail and be false.
%

%---------------------------------------------------------------------
\section{Built-in predicates}
%---------------------------------------------------------------------

\noindent This section contains a comprehensive list of the built-in predicates provided by the \tuprolog{} engine, that is, those predicates defined directly in its core.

Following an established convention in built-in argument template description, which takes root into an imperative interpretation, the symbol \bt{+} in front of an argument means an \emph{input argument}, \bt{-} means \emph{output argument}, \bt{?} means \emph{input/output} argument, \bt{@} means \emph{input argument} that must be bound.

%---------------------------------------------------------------------
\subsection{Control management}
%---------------------------------------------------------------------

\begin{itemize}
%
\item \bti{true/0}\\
\noindent\bt{true} is true.
%
\item \bti{fail/0}\\
\noindent\bt{fail} is false.
%
\item \verb|','/2|\\
\noindent\verb|','(First,Second)| is true if and only if both \bt{First}
and \bt{Second} are true.
%
\item \bti{!/0}\\
\noindent\bt{!} is true. All choice points between the cut and the
parent goal are removed. The effect is a commitment to use both the
current clause and the substitutions found at the point of the
cut.
%
\item \verb|'$call'/1|\\
\noindent\verb|'$call'(Goal)| is true if and only if \bt{Goal}
represents a goal which is true. It is not opaque to cut.\\
\template{'\$call'(+callable\_term)}
%
\item \bti{halt/0}\\
\noindent\bt{halt} terminates a Prolog demonstration, exiting the
Prolog processor and returning to the system that invoked the
processor.
%
\item \bti{halt/1}\\
\noindent\bt{halt(X)} terminates a Prolog demonstration, exiting
the Prolog processor and returning to the systems that invoked the
processor passing the value of \bt{X} as a message.\\
\template{halt(+int)}
%
\end{itemize}

%---------------------------------------------------------------------
\subsection{Term Unification and Management}
%---------------------------------------------------------------------

\begin{itemize}
%
\item \bti{is/2}\\
\noindent\bt{is(X, Y)} is true iff \bt{X} is unifiable with the
value of the expression \bt{Y}.\\
\noindent\template{is(?term, @evaluable)}
%
\item \verb|'='/2|\\
\noindent\verb|'='(X, Y)| is true iff \bt{X} and \bt{Y} are
unifiable.\\
\noindent\template{'='(?term, ?term)}
%
\item \verb|'\='/2|\\
\noindent\verb|'\='(X, Y)| is true iff \bt{X} and \bt{Y} are
not unifiable. \\
\noindent\template{'$\setminus$='(?term, ?term)}
%
%
\item \verb|'$tolist'/2|\\
\noindent\verb|'$tolist'(Compound, List)| is true if \bt{Compound} is
a compound term, and in this case \bt{List} is list representation
of the compound, with the name as first element and all the
arguments as other elements.\\
\noindent\template{'\$tolist'(@struct, -list)}
%
 \item \verb|'$fromlist'/2|\\
 \noindent\verb|'$fromlist'(Compound, List)| is true if \bt{Compound}
 unifies with the list representation of \bt{List}.\\
\noindent\template{'\$fromlist'(-struct, @list)}
%
 \item \bti{copy\_term/2}\\
 \noindent\bt{copy\_term(Term1, Term2)} is true iff \bt{Term2}
 unifies with the a renamed copy of \bt{Term1}.\\
\noindent\template{copy\_term(?term, ?term)}
%
 \item \verb|'$append'/2|\\
 \noindent\verb|'$append'(Element, List)| is true if \bt{List} is a
 list, with the side effect that the \bt{Element} is appended to
 the list.\\
\noindent\template{'\$append'(+term, @list)}
%
\end{itemize}

%---------------------------------------------------------------------
\subsection{Knowledge-base management}
%---------------------------------------------------------------------

\begin{itemize}
%
 \item \verb|'$find'/2|\\
 \noindent\verb|'$find'(Clause, ClauseList)| is true if \bt{ClauseList}
 is a list, and \bt{Clause} is a clause, with the side effect that
 all the clauses of the database matching \bt{Clause} are
 appended to the list.\\
\noindent\template{'\$find'(@clause, @list)}
%
\item \bti{abolish/1}\\
\noindent\bt{abolish(PI)} completely wipes out the dynamic
predicate matching the predicate indicator \texttt{PI}.\\
\noindent\template{\bt{abolish(@term)}}
%
\item \bti{asserta/1}\\
\noindent\bt{asserta(Clause)} is true, with the side effect that
the clause \bt{Clause} is added to the beginning of database.\\
\noindent\template{asserta(@clause)}
%
\item \bti{assertz/1}\\
\noindent\bt{assertz(Clause)} is true, with the side effect that
the clause \bt{Clause} is added to the end of the database.\\
\noindent\template{assertz(@clause)}
%
\item \verb|'$retract'/1|\\
\noindent\verb|'$retract'(Clause)| is true if the database contains
at least one clause unifying with \bt{Clause}. As a side effect, the
clause is removed from the database. It is not re-executable.\\
\noindent\template{'\$retract'(@clause)}
%
\end{itemize}

%---------------------------------------------------------------------
\subsection{Operators and Flags Management}
%---------------------------------------------------------------------

\begin{itemize}
%
\item \bti{op/3}\\
\noindent\bt{op(Priority, Specifier, Operator)} is true. It always succeeds,
modifying the operator table as a side effect. If \bt{Priority} is 0, then
\bt{Operator} is removed from the operator table; else, \bt{Operator} is
added to the operator table, with priority (lower binds tighter) \bt{Priority}
and associativity determined by \bt{Specifier}. If an operator with the same
\bt{Operator} symbol and the same \bt{Specifier} already exists in the operator
table, the predicate modifies its priority according to the specified \bt{Priority}
argument.\\
\noindent\template{op(+integer, +specifier, @atom\_or\_atom\_list)}
%
%  \item \bti{flag/2}\\
%  \noindent\bt{flag(FlagName, NewValue)} is true if \bt{FlagName} is
%  the name of a modifiable flag currently defined in the engine
%  and \bt{NewValue} is a valid value for the flag. As a side effect,
%  \bt{NewValue} becomes the new value of the flag \bt{FlagName}.\\
%  \noindent\template{flag(@string, @term)}
%
 \item \bti{flag\_list/1}\\
 \noindent\bt{flag\_list(FlagList)} is true and \bt{FlagList} is
 the list of the flags currently defined in the engine.\\
 \noindent\template{flag\_list(-list)}
%
\item \bti{set\_prolog\_flag/2}\\
\noindent\bt{set\_prolog\_flag(Flag, Value)} is true, and as a side
effect associates \bt{Value} with the flag \bt{Flag}, where
\bt{Value} is a value that is within the implementation defined
range of values for \bt{Flag}.\\
\noindent\template{set\_prolog\_flag(+flag, @nonvar)}
%
\item \bti{get\_prolog\_flag/2}\\
\noindent\bt{get\_prolog\_flag(Flag, Value)} is true iff \bt{Flag}
is a flag supported by the engine and \bt{Value} is the value
currently associated with it. Note that \bt{get\_prolog\_flag/2} is
not re-executable.\\
\noindent\template{get\_prolog\_flag(+flag, ?term)}
%
\end{itemize}

%---------------------------------------------------------------------
\subsection{Libraries Management}
%---------------------------------------------------------------------

\begin{itemize}
%
 \item \bti{load\_library/1}\\
 \noindent\bt{load\_library(LibraryName)} is true if
 \bt{LibraryName} is the name of a \tuprolog{} library available
 for loading. As side effect, the specified library is loaded by
 the engine. Actually \bt{LibraryName} is the full name of
 the Java class providing the library.\\
 \noindent\template{load\_library(@string)}
%
 \item \bti{unload\_library/1}\\
 \noindent\bt{unload\_library(LibraryName)} is true if
 \bt{LibraryName} is the name of a library currently loaded in the
 engine. As side effect, the library is unloaded from the engine. Actually \bt{LibraryName} is the full name of
 the Java class providing the library.\\
 \noindent\template{unload\_library(@string)}
%
\end{itemize}

%---------------------------------------------------------------------
\subsection{Directives}
%---------------------------------------------------------------------

Directives are used in Prolog text only as queries to be immediately executed when loading it. When a corresponding predicate with the same procedure name as a directive exists, they perform the same actions. Their arguments will satisfy the same constraints as those required for an errorless execution of the corresponding predicate, otherwise their behaviour is undefined.

In \tuprolog{}, directives are not composable: each query must contain one and only one directive. When you need to use multiple directives, you must employ multiple queries as well.

\begin{itemize}
 %
 \item \bti{:- op/3}\\
 \noindent\bt{op(Priority, Specifier, Operator)} adds \bt{Operator}
 to the operator table, with priority (lower binds tighter)
 \bt{Priority} and associativity determined by \bt{Specifier}.\\
 \noindent\template{op(+integer, +specifier, @atom\_or\_atom\_list)}
 %
 \item \bti{:- flag/4}\\
 \noindent\bt{flag(FlagName, ValidValuesList, DefaultValue, IsModifiable)}
 adds to the engine a new flag, identified by the \bt{FlagName}
 name, which can assume only the values listed in
 \bt{ValidValuesList} with \bt{DefaultValue} as default value, and
 that can be modified if \bt{IsModifiable} is true.\\
 \noindent\template{flag(@string, @list, @term, @{true, false})}
 %
 \item \bti{:- initialization/1}\\
 \noindent\bt{initialization(Goal)} sets the starting goal to be executed just
 after the theory has been consulted.\\
 \noindent\template{initialization(@goal)}
 %
 \item \bti{:- solve/1}\\
 \noindent Synonym for \bt{initialization/1}. \emph{Deprecated.}\\
 \noindent\template{solve(@goal)}
 %
 \item \bti{:- load\_library/1}\\
 \noindent\bt{load\_library(LibraryName)} is a valid directive if true if
 \bt{LibraryName} is the name of a \tuprolog{} library available
 for loading. This directive loads the specified library in the engine.
 Actually \bt{LibraryName} is the full name of the Java class providing the library.\\
 \noindent\template{load\_library(@string)}
 %
 \item \bti{:- include/1}\\
 \noindent\bt{include(Filename)} immediately loads the theory
 contained in the file specified by \bt{Filename}.\\
 \noindent\template{include(@string)}
 %
 \item \bti{:- consult/1}\\
 \noindent Synonym for \bt{include/1}. \emph{Deprecated.}\\
 \noindent\template{consult(@string)}
 %
\end{itemize}
%

%******************************************************************************%
%=======================================================================
\chapter{\tuprolog{} Libraries}
\label{ch:standard-libraries}
%=======================================================================

Libraries are the means by which \tuprolog{} achieves its
fundamental characteristics of minimality and configurability.
%
The engine is by design choice a minimal, purely-inferential core: as
such, it only includes a few \emph{built-in} predicates, intended as
predicates statically defined inside the core, to establish the
foundation which the mechanisms of the engine are based on.
%
Instead, each and every other piece of functionality, in the form of
predicates, functors, flags and operators, is delivered by libraries,
and can be added to or subtracted from the engine at any time.
%
Thus, a \tuprolog{} engine can be dynamically extended by loading
(and unloading) any number of libraries. Each library can provide a
specific set of predicates, functors and a related theory, which can
be used to define new flags and operators.
%
Besides built-in and library predicates, new functionalities can also
be added to an engine by feeding it with a user-defined Prolog theory.

Libraries can be loaded at any time in the \tuprolog{} engine, both
from the Java side, by means of the \texttt{loadLibrary} method of
the \texttt{Prolog} object representing a \tuprolog{} engine, and
from the Prolog side, using the \texttt{load\_library/1} predicate.
%
For example, suppose you want to exploit some features defined in a
library whose name is \texttt{ExampleLibrary}. If, on the Java side,
you want to load the library immediately afterwards building a
\tuprolog{} engine, you would write the following code, using the
fully qualified Java class name for the library:
%
\begin{verbatim}
Prolog engine = new Prolog();
try {
    engine.loadLibrary("com.example.ExampleLibrary");
} catch (InvalidLibraryException e) {
}
\end{verbatim}
%
If, on the other hand, you just want to load the library on the Prolog
side for those clauses which actually make use of its predicates, you
would write the following code, using just the name of the library,
which can be different from its fully qualified class name:
%
\begin{verbatim}
% println/1 is defined in ExampleLibrary
run_test(Test, Result) :- run(Test, Result),
                          load_library('ExampleLibrary'),
                          println(Result).
\end{verbatim}
%
Correspondingly, means for unloading libraries are provided, in the
form of the \texttt{unloadLibrary} method of the \texttt{Prolog}
class on the Java side, and the \texttt{unload\_library/1} predicate
on the Prolog side.
%
It must be noted that predicates for loading or unloading libraries
are also available in the form of directives: they perform the same
actions, but as directives they are immediately executed when the
Prolog text containing them is feeded to the engine.

Since the core comes as a pure inferential engine, \tuprolog{}
includes in its distribution some standard libraries which are
loaded by default into the engine at construction time. While it is
possible to create an engine with no default libraries preloaded,
those standard libraries provide the fundamental bricks of a Prolog
engine, in the form of basic functionalities, ISO compliant
predicates and evaluable functors, I/O predicates and predicates for
interoperability and integration between Java and Prolog.
%
More user-defined libraries can be then loaded or unloaded, thus
exploiting the dynamic configurability of \tuprolog{} engines which
can be reconfigured on the fly enriching or reducing the set of
available functionalities by need.

The standard libraries are:
%
\begin{description}
\item[BasicLibrary] (class \texttt{alice.tuprolog.lib.BasicLibrary}) |
  provides common Prolog predicates and functors, and operators. No
  I/O predicates are included.
%
\item[DCGLibrary] (class \texttt{alice.tuprolog.lib.DCGLibrary}) |
provides support for Definite Clause Grammar, an extension of context
free grammars used for describing natural and formal languages.
%
\item[IOLibrary] (class \texttt{alice.tuprolog.lib.IOLibrary}) |
provides some basic and classic I/O predicates.
%
\item[ISOLibrary] (class \texttt{alice.tuprolog.lib.ISOLibrary}) |
provides predicates and functors that are part of the built-in
section in the ISO standard \cite{iso95}, and are not provided by
previous libraries.
%
\item[JavaLibrary] (class \texttt{alice.tuprolog.lib.JavaLibrary}) |
provides predicates and functors to create, access and deploy
(existent or new) Java resources, like objects and classes.
%
\end{description}
%
\noindent The description of each library is provided by discussing in
the order: predicates, functors, operators and flags defined by the
library.
%
For each library the dependencies with other libraries are specified:
%
that is, which other libraries are required in order to provide the
correct computational behaviour.
%

%-----------------------------------------------------------------------
\section{BasicLibrary}
%-----------------------------------------------------------------------

\noindent \emph{Library Dependencies}: none.

This library provides common Prolog built-in predicates,
functors, and operators. No I/O predicates are included.

Please note that in the following \texttt{string} means a single or
double quoted string, as detailed in Chapter \ref{ch:engine};
\texttt{expr} means an evaluable expression, that is a term that can
be interpreted as a value by some library functors.

%---------------------------------------------------------------------
\subsection{Predicates}
%---------------------------------------------------------------------

\noindent Here follows a list of predicates implemented by this
library, grouped by category.

%---------------------------------------------------------------------
\subsubsection{Type Testing}
%---------------------------------------------------------------------
%
\begin{itemize}
    \item \bti{constant/1}\\
    \noindent\bt{constant(X)} is true iff \bt{X} is a constant value.\\
    \template{constant(@term)}
    %
    \item \bti{number/1}\\
    \noindent\bt{number(X)} is true iff \bt{X} is an integer or a float.\\
    \template{number(@term)}
    %
    \item \bti{integer/1}\\
    \noindent\bt{integer(X)} is true iff \bt{X} is an integer.\\
    \template{integer(@term)}
    %
    \item \bti{float/1}\\
    \noindent\bt{float(X)} is true iff \bt{X} is an float.\\
    \template{float(@term)}
    %
    \item \bti{atom/1}\\
    \noindent\bt{atom(X)} is true iff \bt{X} is an atom.\\
    \template{atom(@term)}
    %
    \item \bti{compound/1}\\
    \noindent\bt{compound(X)} is true iff \bt{X} is a compound term,
    that is neither atomic nor a variable.\\
    \template{compound(@term)}
    %
    \item \bti{var/1}\\
    \noindent\bt{var(X)} is true iff \bt{X} is a variable.\\
    \template{var(@term)}
    %
    \item \bti{nonvar/1}\\
    \noindent\bt{nonvar(X)} is true iff \bt{X} is not a variable.\\
    \template{nonvar(@term)}
    %
    \item \bti{atomic/1}\\
    \noindent\bt{atomic(X)} is true iff \bt{X} is atomic (that is is an atom, an integer
    or a float).\\
    \template{atomic(@term)}
    %
    \item \bti{ground/1}\\
    \noindent\bt{ground(X)} is true iff \bt{X} is a ground term.\\
    \template{ground(@term)}
    %
    \item \bti{list/1}\\
    \noindent\bt{list(X)} is true iff \bt{X} is a list.\\
    \template{list(@term)}
    %
\end{itemize}

%---------------------------------------------------------------------
\subsubsection{Term Creation, Decomposition and Unification}
%---------------------------------------------------------------------
%
\begin{itemize}
%
\item \verb|'=..'/2| : \textit{univ}\\
\noindent\verb|'=..'(Term, List)| is true if \bt{List} is a list
consisting of the functor and all arguments of \bt{Term}, in
order. \\
\template{'=..'(?term, ?list)}
%
\item \bti{functor/3}\\
\noindent\bt{functor(Term, Functor, Arity)} is true if the term
\bt{Term} is a compound term, \bt{Functor} is its functor, and
\bt{Arity} (an integer) is its arity; or if \bt{Term} is an atom
or number equal to \bt{Functor} and \bt{Arity} is 0.\\
\template{functor(?term, ?term, ?integer)}
%
\item \bti{arg/3}\\
\noindent\bt{arg(N, Term, Arg)} is true if \bt{Arg} is the \bt{N}th
arguments of \bt{Term} (counting from 1).\\
\template{arg(@integer, @compound, -term)}
%
\item \bti{text\_term/2}\\
\noindent\bt{text\_term(Text, Term)} is true iff \bt{Text} is the
text representation of the term \bt{Term}.\\
\template{text\_term(?text, ?term)}
%
\item \bti{text\_concat/3}\\
\noindent\bt{text\_concat(TextSource1, TextSource2, TextDest)} is
true iff \bt{TextDest} is the text resulting by appending the text
\bt{TestSource2} to \bt{TextSource1}\.\\
\template{text\_concat(@string, @string, -string)}
%
\item \bti{num\_atom/2}\\
\noindent\bt{num\_atom(Number, Atom)} succeeds iff \bt{Atom}
is the atom representation of the number \bt{Number}\\
\template{number\_codes(+number, ?atom)}\\
\template{number\_codes(?number, +atom)}
%
\end{itemize}

%---------------------------------------------------------------------
\subsubsection{Occurs Check}
%---------------------------------------------------------------------

\noindent When the process of unification takes place between a
variable $S$ and a term $T$, the first thing a Prolog engine should do
before proceeding is to check that $T$ does not contain any occurences
of $S$. This test is known as \emph{occurs check} \cite{ss94} and is
necessary to prevent the unification of terms such as $s(X)$ and $X$,
for which no finite common instance exists. Most Prolog
implementations omit the occurs check from their unification algorithm
for reasons related to speed and efficiency: \tuprolog{} is no
exception. However, they provide a predicate for occurs check
augmented unification, to be used when the programmer wants to never
incur on an error or an undefined result during the process.
%
\begin{itemize}
%
\item \bti{unify\_with\_occurs\_check/2}\\
\noindent\bt{unify\_with\_occurs\_check(X, Y)} is true iff \bt{X}
and \bt{Y} are unifiable.\\
\noindent\template{unify\_with\_occurs\_check(?term, ?term)}
%
\end{itemize}
%
%---------------------------------------------------------------------
\subsubsection{Expression and Term Comparison}
%---------------------------------------------------------------------
\begin{itemize}
%
    \item expression comparison (generic template:
    \emph{pred}(@expr, @expr)):\\
        \verb|'=:=', '=\=', '>', '<', '>=', '=<'|;
    %
    \item term comparison (generic template:
    \emph{pred}(@term, @term)):\\
         \verb|'==', '\==', '@>', '@<', '@>=', '@=<'|.

\end{itemize}

%---------------------------------------------------------------------
\subsubsection{Finding Solutions}
%---------------------------------------------------------------------
\begin{itemize}
%
\item \bti{findall/3}\\
\noindent\bt{findall(Template, Goal, List)} is true if and only if
\bt{List} unifies with the list of values to which a variable X not
occurring in \bt{Template} or \bt{Goal} would be instantiated
by successive re-executions of\\
\bt{call(Goal), X = Template}\\
\noindent after systematic replacement of all variables in X by
new variables.\\
\template{\bt{findall(?term, +callable\_term, ?list)}}
%
\item \bti{bagof/3}\\
\noindent\bt{bagof(Template, Goal, Instances)} is true if
\bt{Instances} is a non-empty list of all terms such that each
unifies with \bt{Template} for a fixed instance W of the variables
of \bt{Goal} that are free with respect to \bt{Template}. The
ordering of the elements of \bt{Instances} is the order in which
the solutions are found.\\
\template{bagof(?term, +callable\_term, ?list)}
%
\item \bti{setof/3}\\
\noindent\bt{setof(Template, Goal, List)} is true if \bt{List} is a
sorted non-empty list of all terms that each unifies with
\bt{Template} for a fixed instance W of the variables of \bt{Goal}
that are free with respect to \bt{Template}.\\
\template{\bt{setof(?term, +callable\_term, ?list)}}
%
\end{itemize}

%---------------------------------------------------------------------
\subsubsection{Control Management}
%---------------------------------------------------------------------
\begin{itemize}
%
\item \bti{(->)/2} : \textit{if-then}\\
\noindent\verb|'->'(If, Then)| is true if and only if \bt{If} is true
and \bt{Then} is true for the first solution of \bt{If}.
%
\item \bti{(;)/2} : \textit{if-then-else}\\
\noindent\verb|';'(Either, Or)| is true iff either \bt{Either} or
\bt{Or} is true.
%
\item \bti{call/1}\\
\noindent\bt{call(Goal)} is true if and only if \bt{Goal}
represents a goal which is true. It is opaque to cut.\\
\template{call(+callable\_term)}
%
\item \bti{once/1}\\
\noindent\bt{once(Goal)} finds exactly one solution to \bt{Goal}.
It is equivalent to \bt{call((Goal, !))} and is opaque to cuts.\\
\template{once(@goal)}
%
\item \bti{repeat/0}\\
Whenever backtracking reaches \noindent\bt{repeat}, execution
proceeds forward again through the same clauses as if
another alternative has been found.\\
\template{repeat}
%
\item \verb|'\+'/1| : \textit{not provable}\\
\noindent\verb|'\+'(Goal)| is the negation predicate and is
opaque to cuts. That is, \verb|'\+'(Goal)| is like
\bt{call(Goal)} except that its success or failure is the opposite.\\
\template{'$\setminus$+'(@goal)}
%
\item \bti{not/1}\\
\noindent The predicate \bt{not/1} has the same semantics and
implementation as the predicate \verb|'\+'/1|.\\
\template{not(@goal)}
%
\end{itemize}

%---------------------------------------------------------------------
\subsubsection{Clause Retrival, Creation and Destruction}
%---------------------------------------------------------------------

\noindent Every Prolog engine lets programmers modify its logic
database during execution by adding or deleting specific clauses. The
ISO standard \cite{iso95} distinguishes between static and dynamic
predicates: only the latter can be modified by asserting or retracting
clauses. While typically the \emph{dynamic/1} directive is used to
indicate whenever a user-defined predicate is dynamically modifiable,
\tuprolog{} engines work differently, establishing two default
behaviors: library predicates are always of a static kind; every other
user-defined predicate is dynamic and modifiable at runtime.
%
The following list contains library predicates used to manipulate the
knowledge base of a \tuprolog{} engine during execution.

\begin{itemize}
%
\item \bti{clause/2}\\
\noindent\bt{clause(Head, Body)} is true iff \bt{Head} matches the
head of a dynamic predicate, and \bt{Body} matches its body. The
body of a fact is considered to be \bt{true}. \bt{Head} must be at
least partly instantiated.\\
\template{\bt{clause(@term, -term)}}
%
\item \bti{assert/1}\\
\noindent\bt{assert(Clause)} is true and adds \bt{Clause} to the
end of the database.\\
\template{\bt{assert(@term)}}
%
\item \bti{retract/1}\\
\noindent\bt{retract(Clause)} removes from the knowledge base a
dynamic clause that matches \texttt{Clause} (which must be at least
partially instantiated). Gives multiple solutions upon backtracking.\\
\template{\bt{retract(@term)}}
%
\item \bti{retractall/1}\\
\noindent\bt{retractall(Clause)} removes from the knowledge base all
the dynamic clauses matching with \texttt{Clause} (which must be at
least partially instantiated).\\
\template{\bt{retractall(@term)}}
%
\end{itemize}

%---------------------------------------------------------------------
\subsubsection{Operator Management}
%---------------------------------------------------------------------
\begin{itemize}
%
\item \bti{current\_op/3}\\
\noindent\bt{current\_op(Priority, Type, Name)} is true iff
\bt{Priority} is an integer in the range [0, 1200], \bt{Type} is
one of the \bt{fx}, \bt{xfy}, \bt{yfx}, \bt{xfx} values and
\bt{Name} is an atom, and as side effect it adds a new operator to
the engine operator list.\\
\template{current\_op(?integer, ?term, ?atom)}
%
\end{itemize}

%---------------------------------------------------------------------
\subsubsection{Flag Management}
%---------------------------------------------------------------------
\begin{itemize}
%
\item \bti{current\_prolog\_flag/3}\\
\noindent\bt{current\_prolog\_flag(Flag,Value)} is true if the
value of the flag \bt{Flag}
is \bt{Value}\\
\template{current\_prolog\_flag(?atom,?term)}
%
\end{itemize}

%---------------------------------------------------------------------
\subsubsection{Actions on Theories and Engines}
%---------------------------------------------------------------------
\begin{itemize}
%
%
\item \bti{set\_theory/1}\\
\noindent\bt{set\_theory(TheoryText)} is true iff \bt{TheoryText}
is the text representation of a valid \tuprolog{} theory, with the
side effect of setting it as the new theory of the engine.\\
\template{set\_theory(@string)}
%
\item \bti{add\_theory/1}\\
\noindent\bt{add\_theory(TheoryText)} is true iff \bt{TheoryText}
is the text representation of a valid \tuprolog{} theory, with the
side effect of appending it to the current theory of the engine.\\
\template{add\_theory(@string)}
%
\item \bti{get\_theory/1}\\
\noindent\bt{get\_theory(TheoryText)} is true, and
\bt{TheoryText} is the text representation of the current theory of the engine.\\
\template{get\_theory(-string)}
%
\item \bti{agent/1}\\
\noindent\bt{agent(TheoryText)} is true, and spawns a
\tuprolog{} agent with the knowledge base provided as a Prolog
textual form in \texttt{TheoryText} (the goal is described in the
knowledge base).\\
\template{agent(@string)}
%
\item \bti{agent/2}\\
\noindent\bt{agent(TheoryText, Goal)} is true, and spawn a
\tuprolog{} agent with the knowledge base provided as a Prolog
textual form in \texttt{TheoryText}, and solving the query
\texttt{Goal}
as a goal.\\
\template{agent(@string, @term)}
%
\end{itemize}
%
%---------------------------------------------------------------------
\subsubsection{Spy Events}
%---------------------------------------------------------------------
%
During each demonstration, the engine notifies to interested listeners so-called
{\em spy events}, containing informations on its internal state, such as the
current subgoal being evaluated, the configuration of the execution stack and
the available choice points. The different kinds of spy events currently
corresponds to the different states which the virtual machine realizing the
\tuprolog{}'s inferential core can be found into. \textit{Init} events are
spawned whenever the machine initialize a subgoal for execution; \textit{Call}
events are generated when a choice must be made for the next subgoal to be
executed; \textit{Eval} events represent actual subgoal evaluation; finally,
\textit{Back} events are notified when a backtracking occurs during the
demonstration process.
%
\begin{itemize}
%
\item \bti{spy/0}\\
\noindent\bt{spy} is true and enables the notification of spy
events occurring inside the engine.\\
\template{spy}
%
\item \bti{nospy/0}\\
\noindent\bt{nospy} is true and disables the notification of the
spy events inside the engine.\\
\template{nospy}
%
\end{itemize}

%---------------------------------------------------------------------
\subsubsection{Auxiliary predicates}
%---------------------------------------------------------------------

\noindent The following predicates are provided by the library's theory.

\begin{itemize}
%
\item \bti{member/2}\\
\noindent\bt{member(Element, List)} is true iff \bt{Element} is an
element of the list
\bt{List}\\
\template{member(?term, +list)}
%
\item \bti{length/2}\\
\noindent\bt{length(List, NumberOfElements)} is true in three
different cases: (1) if \bt{List} is instantiated to a list of
determinate length, then \bt{Length} will be unified with this
length; (2) if \bt{List} is of indeterminate length and \bt{Length}
is instantiated to an integer, then \bt{List} will be unified with a
list of length \bt{Length} and in such a case the list elements are
unique variables; (3) if \bt{Length} is unbound then \bt{Length}
will be unified with all possible lengths of \bt{List}.\\
\template{member(?list, ?integer)}
%
\item \bti{append/3}\\
\noindent\bt{append(What, To, Target)} is true iff \bt{Target} list
can be obtained by appending the \bt{To} list to the \bt{What}
list \\
\template{append(?list, ?list, ?list)}
%
\item \bti{reverse/2}\\
\noindent\bt{reverse(List, ReversedList)} is true iff
\bti{ReversedList} is the reverse list of \bt{List}\\
\template{reverse(+list, -list)}
%
\item \bti{delete/3}\\
\noindent\bt{delete(Element, ListSource, ListDest)} is true iff
\bt{ListDest} list can be obtained by removing the element
\bt{Element} from the list \bt{ListSource}.\\
\template{delete(@term, +list, -list)}
%
\item \bti{element/3}\\
\noindent\bt{element(Position, List, Element)} is true iff
\bt{Element} is the \bt{Position}th element of the list \bt{List}
(starting the count from 1).\\
\template{element(@integer, +list, -term)}
%
\item \bti{quicksort/3}\\
\noindent\bt{quicksort(List, ComparisonPredicate, SortedList)} is
true iff \bt{SortedList} is the list \bt{List} sorted by the comparison
predicate \bt{ComparisonPredicate}.\\
\template{element(@list, @pred, -list)}
%
\end{itemize}

%---------------------------------------------------------------------
\subsection{Functors}
%---------------------------------------------------------------------

Functors for expression evaluation (with usual semantics):
\begin{itemize}
    \item unary:  \verb|+, -, ~, +|
    \item binary:  \verb|+, -, *, \, **, <<, >>, /\, \/|
\end{itemize}

%---------------------------------------------------------------------
\subsection{Operators}
%---------------------------------------------------------------------

\begin{table}[h]
    %
    \begin{center}{\small\tt
    \begin{tabular}{p{2cm}|p{1cm}|p{1cm}}\hline\hline
    Name & Assoc. & Prio. \\ \hline\hline
    :-      &   fx  &   1200 \\
    :-      &   xfx &   1200 \\
    ?-      &   fx  &   1200 \\
    ;       &   xfy &   1100 \\
    ->      &   xfy &   1050 \\
    ,       &   xfy &   1000 \\
    not     &   fy  &   900 \\
    $\setminus$+   &   fy   & 900   \\
    =       &   xfx &   700 \\
    $\setminus$=    &  xfx  &   700 \\
    ==      &   xfx &   700 \\
    $\setminus$==   &  xfx  &   700 \\
    @>      &   xfx & 700   \\
    @<      &   xfx & 700   \\
    @=<    &   xfx & 700   \\
    @>=    &   xfx & 700   \\
    =:=    &   xfx & 700   \\
    =$\setminus$=   &   xfx & 700   \\
    >      &   xfx & 700   \\
    <      &   xfx & 700   \\
    >=      &   xfx & 700   \\
    =<      &   xfx & 700   \\
    is      &   xfx &   700 \\
    =..     &   xfx & 700 \\
    +       &   yfx & 500 \\
    -       &   yfx & 500 \\
    $/\setminus$    &   yfx &   500 \\
    $\setminus/$    &   yfx &   500 \\
    $\ast$  &   yfx & 400 \\
    /       &   yfx & 400 \\
    //      &   yfx & 400 \\
    >>      &   yfx & 400 \\
    <<      &   yfx & 400 \\
    >>      &   yfx & 400 \\
    $\ast$$\ast$  &   xfx & 200 \\
    \textasciicircum  &   xfy & 200 \\
    $\setminus$$\setminus$      &   fx & 200 \\
    -       &   fy & 200 \\
    \hline\hline
    \end{tabular}
    }\end{center}
\end{table}

\clearpage

%-----------------------------------------------------------------------
\section{ISOLibrary}
%-----------------------------------------------------------------------

\noindent \emph{Library Dependencies}: BasicLibrary.

This library contains almost\footnote{Currently ISO exceptions, ISO
I/O predicates and some ISO directives are not supported.} all the
built-in predicates and functors that are part of the ISO standard
and that are not part directly of the \tuprolog{} core engine or
other core libraries.
%
Moreover, some features are added, not currently ISO, such as the
support for definite clause grammars (DCGs).
%

%---------------------------------------------------------------------
\subsection{Predicates}
%---------------------------------------------------------------------

\noindent Here follows a list of predicates implemented by this
library, grouped by category.

%---------------------------------------------------------------------
%\subsubsection{Clause Retrieval and Information}
%---------------------------------------------------------------------

%\begin{itemize}
%
%\item \bti{current\_predicate/1}\\
%\noindent\bt{current\_predicate(Functor/Arity)} -- \emph{to be implemented}\\
%\template{\bt{current\_predicate(?term)}}
%
%\end{itemize}

%---------------------------------------------------------------------
\subsubsection{Type Testing}
%---------------------------------------------------------------------

\begin{itemize}
%
\item \bti{bound/1}\\
\noindent\bt{bound(Term)} is a synonym for the \bt{ground/1} predicate
defined in BasicLibrary.\\
\template{bound(+term)}
%
\item \bti{unbound/1}\\
\noindent\bt{unbound(Term)} is true iff \bt{Term} is not a ground
term.\\
\template{unbound(+term)}
%
\end{itemize}

%---------------------------------------------------------------------
\subsubsection{Atoms Processing}
%---------------------------------------------------------------------

\begin{itemize}
%
\item \bti{atom\_length/2}\\
\noindent\bt{atom\_length(Atom, Length)} is true iff the integer
\bt{Length} equals the number of characters in the name of atom
\bt{Atom}.\\
\template{atom\_length(+atom, ?integer)}
%
\item \bti{atom\_concat/3}\\
\noindent\bt{atom\_concat(Start, End, Whole)} is true iff the
\bt{Whole} is the atom obtained by concatenating the characters of
\bt{End} to those of \bt{Start}. If \bt{Whole} is instantiated, then
all decompositions of \bt{Whole} can be obtained by backtracking.\\
\template{atom\_concat(?atom, ?atom, +atom)}\\
\template{atom\_concat(+atom, +atom, -atom)}
%
\item \bti{sub\_atom/5}\\
\noindent\bt{sub\_atom(Atom, Before, Length, After, SubAtom)} is
true iff \bt{SubAtom} is the sub atom of \bt{Atom} of length
\bt{Length} that appears with \bt{Before} characters preceding it
and \bt{After} characters following. It is re-executable.\\
\template{sub\_atom(+atom, ?integer, ?integer, ?integer, ?atom)}
%
\item \bti{atom\_chars/2}\\
\noindent\bt{atom\_chars(Atom,List)} succeeds iff \bt{List} is a
list whose elements are the one character atoms that in order make
up \bt{Atom}.\\
\template{atom\_chars(+atom, ?character\_list)}\\
\template{atom\_chars(-atom, ?character\_list)}
%
\item \bti{atom\_codes/2}\\
\noindent\bt{atom\_codes(Atom, List)} succeeds iff \bt{List} is a
list whose elements are the character codes that in order correspond
to the characters that make up \bt{Atom}.\\
\template{atom\_codes(+atom, ?character\_code\_list)}\\
\template{atom\_chars(-atom, ?character\_code\_list)}
%
\item \bti{char\_code/2}\\
\noindent\bt{char\_code(Char, Code)} succeeds iff \bt{Code} is a
the character code that corresponds to the character \bt{Char}.\\
\template{char\_code(+character, ?character\_code)}\\
\template{char\_code(-character, +character\_code)}
%
\item \bti{number\_chars/2}\\
\noindent\bt{number\_chars(Number, List)} succeeds iff \bt{List}
is a list whose elements are the one character atoms that in
order make up \bt{Number}.\\
\template{number\_chars(+number, ?character\_list)}\\
\template{number\_chars(-number, ?character\_list)}
%
\item \bti{number\_codes/2}\\
\noindent\bt{number\_codes(Number, List)} succeeds iff \bt{List}
is a list whose elements are the codes for the one character atoms
that in order make up \bt{Number}.\\
\template{number\_codes(+number,?character\_code\_list)}\\
\template{number\_codes(-number,?character\_code\_list)}
%
\end{itemize}

%---------------------------------------------------------------------
\subsection{Functors}
%---------------------------------------------------------------------

\begin{itemize}
    \item Trigonometric functions: \bt{sin(+expr)}, \bt{cos(+expr)}, \bt{atan(+expr)}.
    %
    \item Logarithmic functions: \bt{exp(+expr)}, \bt{log(+expr)}, \bt{sqrt(+expr)}.
    %
    \item Absolute value functions: \bt{abs(+expr)}, \bt{sign(+Expr)}.
    %
    \item Rounding functions: \bt{floor(+expr)},
    \bt{ceiling(+expr)}, \bt{round(+expr)}, \bt{truncate(+expr)},
    \bt{float(+expr)}, \bt{float\_integer\_part(+expr)},\\\bt{float\_fractional\_part(+expr)}.
    %
    \item Integer division functions:
    \bt{div(+expr, +expr)}, \bt{mod(+expr, +expr)}, \bt{rem(+expr, +expr)}.
\end{itemize}

%---------------------------------------------------------------------
\subsection{Operators}
%---------------------------------------------------------------------

\begin{table}[h]
    %
    \begin{center}{\small\tt
    \begin{tabular}{p{2cm}|p{1cm}|p{1cm}}\hline\hline
    Name & Assoc. & Prio. \\ \hline
    mod   & yfx & 400\\
    div   & yfx & 300\\
    rem   & yfx & 300\\
    sin   & fx & 200\\
    cos   & fx & 200\\
    sqrt  & fx & 200\\
    atan  & fx & 200\\
    exp   & fx & 200\\
    log   & fx & 200\\
    \hline\hline
    \end{tabular}
    }\end{center}
\end{table}

%---------------------------------------------------------------------
\subsection{Flags}
%---------------------------------------------------------------------

\begin{table}[h]
    %
    \begin{center}{\small\tt
    \begin{tabular}{p{6cm}|p{3cm}|p{3cm}}\hline\hline
        Flag Name   & Possible Values & Default Value\\ \hline\hline
        bounded         & {true}           &  true \\
        max\_integer     & {2147483647}     &  2147483647 \\
        min\_integer     & {-2147483648}    &  -2147483648 \\
        integer\_rounding\_function & {down} & down \\
        char\_conversion & {off}           & off \\
        debug           & {off}           & off \\
        max\_arity       & {2147483647}    & 2147483647 \\
        undefined\_predicates & {fail}         & fail \\
        double\_quotes & {atom}         & atom \\
    \hline\hline
    \end{tabular}
    }\end{center}
\end{table}

\clearpage

%---------------------------------------------------------------------
\section{DCGLibrary}
%---------------------------------------------------------------------

\noindent \emph{Library Dependencies}: BasicLibrary.

This library provides support for Definite Clause Grammar
\cite{bra00}, also known as DCG,\footnote{The DCG formalism is not
defined as an ISO standard at the time of writing this document.} an
extension of context free grammars that have proven useful for
describing natural and formal languages, and that may be
conveniently expressed and executed in Prolog.
%
Note that this library is not loaded by default when a \tuprolog{}
engine is created.

A Definite Clause Grammar rule has the general form:\\\\
%
\begin{verbatim}
Head --> Body
\end{verbatim}
%
with the declarative interpretation that a possible form for \texttt{Head}
is \texttt{Body}.
%
A non-terminal symbol may be any term other than a variable or a
number.
%
A terminal symbol may be any term. In order to distinguish
terminals from nonterminals, a sequence of one or more terminal
symbols  is written within a grammar rule as a Prolog list, with the
empty sequence written as the empty list \verb|[]|.
%
The body can contain also executable blocks -- interpreted
according to normal Prolog rule -- enclosed by the \verb|{| and
\verb|}| parenthesis.
%
A simple example of DCG follows:
%
\begin{verbatim}
sentence --> noun_phrase, verb_phrase.
verb_phrase --> verb, noun_phrase.
noun_phrase --> [charles].
noun_phrase --> [linda].
verb --> [loves].
\end{verbatim}
%
So, you can verify that a phrase is correct according to
the grammar simply by the query:
%
\begin{verbatim}
?- phrase(sentence, [charles, loves, linda]).
\end{verbatim}
%
But also:
%
\begin{verbatim}
?- phrase(sentence, [Who, loves, linda]).
\end{verbatim}
%
which would give, according to the grammar, two solutions,
\texttt{Who} bound to \texttt{charles}, and \texttt{Who} bound to
\texttt{linda}.

%---------------------------------------------------------------------
\subsection{Predicates}
%---------------------------------------------------------------------

\noindent The classic built-in predicates provided for parsing DCG
sentences are:

\begin{itemize}
%
\item \bti{phrase/2}\\
\noindent\bt{phrase(Category, List)} is true iff the list \bt{List}
can be parsed as a phrase (i.e. sequence of terminals) of type
\bt{Category}.
%
\bt{Category} can be any term which would be accepted as a
nonterminal of the grammar (or in general, it can be any grammar
rule body), and must be instantiated to a non-variable term at the
time of the call.
%
This predicate is the usual way to commence execution of grammar
rules.
%
If \bt{List} is bound to a list of terminals by the time of the
call, then the goal corresponds to parsing \bt{List} as a phrase
of type \bt{Category}; otherwise if \bt{List} is unbound, then the
grammar is being used for generation.\\
%
\template{phrase(+term, ?list)}
%
%
\item \bti{phrase/3}\\
\noindent\bt{phrase(Category, List, Rest)} is true iff the segment
between the start of list \bt{List} and the start of list \bt{Rest}
can be parsed as a phrase (i.e. sequence of terminals) of type
\bt{Category}.
%
In other words, if the search for phrase Phrase is started at the
beginning of list \bt{List}, then \bt{Rest} is what remains
unparsed after \bt{Category} has been found.
%
Again, \bt{Category} can be any term which would be accepted as a
nonterminal of the grammar (or in general, any grammar rule body),
and must be instantiated to a non variable term at the time
of the call.\\
%
\template{phrase(+term, ?list, ?rest)}

\end{itemize}

%---------------------------------------------------------------------
\subsection{Operators}
%---------------------------------------------------------------------
\mbox{} % Thanks stupid LaTeX for putting the table below where you want
        % instead of where I say.
\begin{table}[!h]
    \begin{center}{\small\tt
    \begin{tabular}{p{2cm}|p{1cm}|p{1cm}}\hline\hline
    Name & Assoc. & Prio. \\ \hline
    --> & xfx & 1200\\
    \hline\hline
    \end{tabular}
    }\end{center}
\end{table}

%-----------------------------------------------------------------------
\section{IOLibrary}
%-----------------------------------------------------------------------

\noindent \emph{Library Dependencies}: BasicLibrary.

The IOLibrary defines classic Prolog built-ins predicates to enable
interaction between Prolog programs and external resources, typically
files and I/O channels.

%---------------------------------------------------------------------
\subsection{Predicates}
%---------------------------------------------------------------------

\noindent Here follows a list of predicates implemented by this
library, grouped by category.

%---------------------------------------------------------------------
\subsubsection{General I/O}
%---------------------------------------------------------------------

\begin{itemize}

\item \bti{see/1}\\
\noindent\bt{see(StreamName)} is used to create/open an input
stream; the predicate is true iff \bt{StreamName} is a string
representing the name of a file to be created or accessed as input
stream, or the string \texttt{stdin} selecting current standard
input as input stream.\\
\template{see(@atom)}
%
\item \bti{seen/0}\\
\noindent\bt{seen} is used to close the input stream previously
opened; the predicate is true iff the closing action is possible.\\
\template{seen}
%
\item \bti{seeing/1}\\
\noindent\bt{seeing(StreamName)} is true iff \texttt{StreamName}
is the name of the stream currently used as input stream.\\
\template{seeing(?term)}
%
\item \bti{tell/1}\\
\noindent\bt{tell(StreamName)} is used to create/open an output
stream; the predicate is true iff \bt{StreamName} is a string
representing the name of a file to be created or accessed as
output stream, or the string \texttt{stdout} selecting current
standard output as output stream.\\
\template{tell(@atom)}
%
\item \bti{told/0}\\
\noindent\bt{told} is used to close the output stream previously
opened; the predicate is true iff the closing action is possible.\\
\template{told}
%
\item \bti{telling/1}\\
\noindent\bt{telling(StreamName)} is true iff \texttt{StreamName}
is the name of the stream currently used as input stream.\\
\template{telling(?term)}
%
\item \bti{put/1}\\
\noindent\bt{put(Char)} puts the character \bt{Char} on
current output stream; it is true iff the operation is possible.\\
\template{put(@char)}
%
\item \bti{get0/1}\\
\noindent\bt{get0(Value)} is true iff \bt{Value} is the next
character (whose code can span on the entire ASCII codes)
available from the input stream, or -1 if no characters are
available;
%
as a side effect the character is removed from the input stream.\\
%
\template{get0(?charOrMinusOne)}
%
\item \bti{get/1}\\
\noindent\bt{get(Value)} is true iff \bt{Value} is the next
character (whose code can span on the range 32..255 as ASCII
codes) available from the input stream, or -1 if no characters are
available;
%
as a side effect the character (with all the characters that
precede this one not in the range 32..255) is removed from the
input stream.\\
%
\template{get(?charOrMinusOne)}
%
\item \bti{tab/1}\\
\noindent\bt{tab(NumSpaces)} inserts \bt{NumSpaces} space
characters (ASCII code 32) on output stream; the predicate is true
iff the operation is possible.\\
%
\template{tab(+integer)}
%
%
\item \bti{read/1}\\
\noindent\bt{read(Term)} is true iff \bt{Term} is Prolog term
available from the input stream.
%
The term must ends with the \emph{.} character; if no valid terms
are available, the predicate fails.
%
As a side effect, the term is removed from the input stream.\\
%
\template{read(?term)}
%
%
\item \bti{write/1}\\
\noindent\bt{write(Term)} writes the term \bt{Term} on current
output stream.
%
The predicate fails if the operation is not possible.\\
%
\template{write(@term)}
%
%
\item \bti{print/1}\\
\noindent\bt{print(Term)} writes the term \bt{Term} on current
output stream, removing apices if the term is an atom representing
a string.
%
The predicate fails if the operation is not possible.\\
%
\template{print(@term)}
%
\item \bti{nl/0}\\
\noindent\bt{nl} writes a new line control character on current
output stream.
%
The predicate fails if the operation is not possible.\\
\template{nl}
%
\end{itemize}

%---------------------------------------------------------------------
\subsubsection{I/O and Theories Helpers}
%---------------------------------------------------------------------
%
\begin{itemize}
%
\item \bti{text\_from\_file/2}\\
\noindent\bt{text\_from\_file(File, Text)} is true iff \bt{Text} is
the text contained in the file whose name is \texttt{File}.\\
\template{text\_from\_file(+string, -string)}
%
%
\item \bti{agent\_file/1}\\
\noindent\bt{agent\_file(TheoryFileName)} is true iff
\texttt{TheoryFileName} is an accessible file containing a Prolog
knowledge base, and as a side effect it spawns a \tuprolog{} agent
provided with that knowledge base.\\
\template{agent\_file(+string)}
%
%
\item \bti{solve\_file/2}\\
\noindent\bt{solve\_file(TheoryFileName, Goal)} is true iff
\texttt{TheoryFileName} is an accessible file containing a Prolog
knowledge base, and as a side effect it solves the query \texttt{Goal}
according to that knowledge base.\\
\template{solve\_file(+string, +goal)}
%
%
\item \bti{consult/1}\\
\noindent\bt{consult(TheoryFileName)} is true iff
\texttt{TheoryFileName} is an accessible file containing a Prolog
knowledge base, and as a side effect it consult that knowledge base,
by adding it to current knowledge base.\\
\template{consult(+string)}
%
\end{itemize}

%---------------------------------------------------------------------
\subsubsection{Random Generation of Numbers}
%---------------------------------------------------------------------

\noindent The random generation of number can be regarded as a form of
I/O.

\begin{itemize}
%
\item \bti{rand\_float/1}\\
\noindent\bt{rand\_float(RandomFloat)} is true iff
\texttt{RandomFloat} is a float random number generated by the
engine between 0 and 1.\\
\template{rand\_float(?float)}
%
\item \bti{rand\_int/2}\\
\noindent\bt{rand\_int(Seed, RandomInteger)} is true iff
\texttt{RandomInteger} is an integer random number generated by
the engine between 0 and \texttt{Seed}.\\
\template{rand\_int(?integer, @integer)}
%
\end{itemize}

%******************************************************************************%
%=======================================================================
\chapter{Accessing Java from \tuprolog{}}
\label{java-library}
%=======================================================================
One of the main advantages of \tuprolog{} open architecture is
that any Java component can be directly accessed and used from
Prolog, in a simple and effective way, by means of the
\texttt{JavaLibrary} library: this delivers all the power of
existing Java components and packages to \tuprolog{} sources.
%
In this way, all Java packages involving interaction (such as Swing,
JDBC, the socket package, RMI) are immediately available to increase
the interaction abilities of \tuprolog:
%
{``one library for all libraries''} is the basic motto.
%
%
\section{Mapping data structures}

Complete bi-directional mapping is provided between Java primitive
types and \tuprolog{} data types.
%
By default, \tuprolog{} integers are mapped into Java \texttt{int}
or \texttt{long} as appropriate, while \texttt{byte} and
\texttt{short} types are mapped into \tuprolog{}'s \texttt{Int}
instances. Only Java \texttt{double} numbers are used to map
\tuprolog{} reals, but \texttt{float} values returned as result of
method invocations or field accesses are handled properly anyway,
without any loss of information.
%
Boolean Java values are mapped into specific \tuprolog{}
\texttt{Term} constants.
%
Java \texttt{char}s are mapped into Prolog atoms, but atoms are
mapped into Java \texttt{String}s by default.
%
The \emph{any} variable (\_) can be used to specify the Java
\texttt{null} value.

%---------------------------------------------------------------
\section{General predicates description}
%---------------------------------------------------------------

\begin{figure}
\caption{A sample Java class (a counter) used to explain JavaLibrary predicates behaviour.
\labelfig{jreflect-example}}
\begin{verbatim}
public class Counter {
    public String name;
    private long value = 0;

    public Counter() {}
    public Counter(String aName) { name = aName; }

    public void setValue(long val) { value=val; }
    public long getValue() { return value; }
    public void inc() { value++; }

    static public String getVersion() { return "1.0"; }
}
\end{verbatim}
\end{figure}

The library offers the following predicates:
%
\begin{enumerate}
  \renewcommand\labelenumi{\it(\roman{enumi})}
  %
  \item the \texttt{java\_object/3} predicate is used to create a new Java
        object of the specified class, according to the syntax:
        %
        \begin{center}
        \texttt{java\_object(\textit{ClassName},
                             \textit{ArgumentList},
                             \textit{ObjectRef})}
        \end{center}
        %
        \texttt{\textit{ClassName}} is a Prolog atom bound to the name of the
        proper Java class (e.g. \verb|'Counter'|, \verb|'java.io.FileInputStream'|),
        while the parameter \texttt{\textit{ArgumentList}} is a Prolog list used to supply
        the required arguments to the class
        constructor: the empty list matches the default constructor.
        %
        %%%%%% RICCI 020202
        Also Java arrays can be instantiated, by appending
        \texttt{[]} at the end of the \texttt{\textit{ClassName}}
        string.
        %%%%%%
        %
        The reference to the newly-created object is bound to \texttt{\textit{ObjectRef}},
        which is typically a ground Prolog term; alternatively, an unbound term
        may be used, in which case the term is bound to an automatically-generated
        %%%%%% RICCI 020202
        Prolog atom \verb|'$obj_N'|, where \texttt{N} is a progressive integer.
        %%%%%%
        %
        In both cases, these atoms are interpreted as object references --
        and therefore used to operate on the Java object from Prolog -- \textit{only}
        in the context of \texttt{JavaLibrary}'s predicates.
        %
        %%%%%% RICCI 020202
        %
        The predicate fails whenever \textit{ClassName} does not identify a valid Java class,
        or the constructor does not exists, or arguments in
        \texttt{\textit{ArgumentList}} are not ground, or \textit{ObjectRef}
        already identifies an object in the system.
        %
        %%%%%%

        According to the default behaviour of \texttt{java\_object},
        when a ground term is bound to a Java object by means of the predicate,
        the binding is kept for the full time of the demonstration
        (even in the case of backtracking).
        %
        This behaviour can be changed, getting the bindings
        created by the \texttt{java\_object} undone by
        backtracking, by changing the value of the flag \texttt{java\_object\_backtrackable}
        to \texttt{true} (the default is \texttt{false}).



  \item the \texttt{<-/2} predicate is used to invoke a method on a Java
        object according to a send-message pattern:
        %
        \begin{center}
        \texttt{\textit{ObjectRef} <- \textit{MethodName}(\textit{Arguments})}

        \texttt{\textit{ObjectRef} <- \textit{MethodName}(\textit{Arguments})
                returns \textit{Term}}
        \end{center}
        %
        \texttt{\textit{ObjectRef}} is an atom interpreted as a Java object
        reference as explained above, while \texttt{\textit{MethodName}}
        is the Java name of the method to be invoked, along with its
        \texttt{\textit{Arguments}}.
        %
        The \texttt{returns} keyword is used to retrieve the value returned
        from non-void Java methods and bind it to a Prolog term: if
        the type of the returned value can be mapped onto a primitive Prolog
        data type (a number or a string), \texttt{\textit{Term}} is unified
        with the corresponding Prolog value; if, instead, it is a Java object
        other than the ones above, \texttt{\textit{Term}} is handled
        as \texttt{\textit{ObjectRef}} in the case of \texttt{java\_object/3}.
        %
        %%%%%% RICCI 020202
        %
        % - static methods access
        %
        Static methods can be invoked using the compound
        term \texttt{class(\textit{ClassName})} in the place
        of \texttt{\textit{ObjectRef}}.
        %
        % - accennare al most specific method?
        %
        %%%%%%
        %%%%%% RICCI 020202
        %
        % - the predicates fails if...
        %
        If \textit{MethodName} does not identify a valid method for the object (class),
        or arguments in \texttt{\textit{ArgumentList}} are not
        valid (because of a wrong signature or not ground values) the predicate fails.
        %%%%%%

  \item the \texttt{.} infix operator is used, in conjunction with the \texttt{set}
        / \texttt{get} pseudo-method pair, to access the public fields of a Java
        object.
        %
        The syntax is based on the following constructs:
        %
        \begin{center}
        \tt
        \textit{ObjectRef} . \textit{Field} <- set(\textit{GroundTerm})\\
        \textit{ObjectRef} . \textit{Field} <- get(\textit{Term})\\
        \end{center}
        %
        As usual, \texttt{\textit{ObjectRef}} is the Prolog identifier for
        a Java object.
        %
        The first construct set the public field \texttt{\textit{Field}}
        to the specified \texttt{\textit{GroundTerm}}, which may be either
        a value of a primitive data type, or a reference to an existing
        object: if \texttt{\textit{GroundTerm}} is not ground, the infix
        predicate fails.
        %
        The second construct retrieves the value of the public field
        \texttt{\textit{Field}}, where \texttt{\textit{Term}} is handled
        once again as \texttt{\textit{ObjectRef}} in the case of
        \texttt{java\_object/3}.
        %
        %%%%%% RICCI 020202
        %
        % - static class access
        %
        As for methods, static fields of classes can be accessed using the compound
        term \texttt{class(\textit{ClassName})} in the place
        of \texttt{\textit{ObjectRef}}.
        %
        % - accesso ad array
        Some helper predicates are provided to access Java array
        elements:\\
        \texttt{java\_array\_set(\textit{ArrayRef}, \textit{Index}, \textit{Object})}\\
        \texttt{java\_array\_set\_\textit{\emph{Basic Type}}(\textit{ArrayRef}, \textit{Index}, \textit{Value})}\\
        to set elements,\\
        \texttt{java\_array\_get(\textit{ArrayRef}, \textit{Index}, \textit{Object})}\\
        \texttt{java\_array\_get\_\textit{\emph{Basic Type}}(\textit{ArrayRef}, \textit{Index}, \textit{Value})}\\
        to get elements,\\
        \texttt{java\_array\_length(\textit{ArrayObject}, \textit{Size})}
        to get the array length.\\
        %
        %%%%%%
        It is worth to point out that the \texttt{set} and \texttt{get} formal
        pseudo-methods above are \textit{not} methods of some class, but just
        part of the construct of the \texttt{.} infix operator, according to
        a JavaBeans-like approach.

  \item the \texttt{as} infix operator is used to explicitly specify (i.e., cast)
        method argument types:
        %
        \begin{center}
        \texttt{\textit{ObjectRef} as \textit{ClassName}}
        \end{center}
        %
        By writing so, the object represented by \texttt{\textit{ObjectRef}} is
        considered to belong to class \texttt{\textit{Classname}}: both
        \texttt{\textit{ObjectRef}} and \texttt{\textit{Classname}} have
        the usual meaning explained above.
        %
        %%%%%% RICCI 020202
        The operator works also with primitive Java types, specified
        as \texttt{\textit{Classname}} (for instance, \texttt{myNumber \textit{as int}}).
        %%%%%%
        %
        The purpose of this predicate is both to provide methods with the
        exact Java types required, and to solve possible overloading conflicts
        a-priori.
        %
        %%%%%% RICCI 020202
        %
        %In particular, \texttt{as} is needed when invoking a method whose a
        %formal argument is of type \texttt{\textit{T}} (e.g., \texttt{JButton})
        %with an actual object argument whose type is a subclass of
        %\texttt{\textit{T}} (say, \texttt{myButton}) -- that is, when upcasting
        %is needed to let Java identify the method signature unambiguously.
        %%%%%%

  %%%%%% RICCI 020202
  %
  \item The \texttt{java\_class/4} predicate makes it possible
        to create and load a new Java class from a source text provided as an
        argument, thus supporting \textit{dynamic compilation} of Java
        classes:
        %
        %
        \begin{center}
        \texttt{java\_class(\textit{SourceText},
                            \textit{FullClassName},
                            \textit{ClassPathList},
                            \textit{ObjectRef})}
        \end{center}
        %
        \texttt{\textit{SourceText}} is a string representing the
        text source of the Java class, \texttt{\textit{FullClassName}}
        is the full Java class name, and \texttt{\textit{ClassPathList}}
        is a (possibly empty) Prolog list of class paths that may
        be required for a successful dynamic compilation
        of this class.
        %
        \texttt{\textit{ObjectRef}} is a reference to an instance of the
        class \texttt{java.lang.Class} that represents the newly-created class.
        %
        The predicate fails whenever \texttt{\textit{SourceText}} contains errors,
        or the class cannot be located in the package hierarchy
        as specified, or \texttt{\textit{ObjectRef}} already identifies an object
        in the system.
  %
  %%%%%%

\end{enumerate}
%
%%%%%% RICCI 020202
%
\noindent Generally, exceptions thrown by method or constructor
calls cannot be explicitly managed and cause the failure of the
related predicate.
%

To taste the flavour of \texttt{JavaLibrary}, let us consider the
example below (refer to \xf{jreflect-example} for \texttt{Counter}
class definition):

%
{\small
\begin{verbatim}
    ?-  java_object('Counter', ['MyCounter'], myCounter),
        myCounter <- setValue(5),
        myCounter <- inc,
        myCounter <- getValue returns Value,
        write(X),

        class('Counter') <- getVersion return Version,

        myCounter.name <- get(Name),
        class('java.lang.System') . out <- get(Out),
        Out <- println(Name),

        myCounter.name <- set('MyCounter2'),

        java_object('Counter[]', [10], ArrayCounters),
        java_array_set(ArrayCounters, 0, myCounter).
\end{verbatim}}
%
\noindent Here, a \texttt{Counter} object is created providing the
\texttt{MyCounter} name as constructor argument: the reference to
the new object is bound to the Prolog atom \texttt{myCounter}.
%
This reference is then used for method invocation via the
\texttt{<-} operator, calling the \texttt{setValue(5)} method
(which is void and therefore returns nothing) first, incrementing
the counter (no arguments are specified) and invoking the
\texttt{getValue} method just after.
%
Since \texttt{getValue} returns an integer value, the
\texttt{returns} operator retrieves the method result (hopefully,
5) and binds it to the \texttt{X} Prolog variable, which is
printed via the Prolog \texttt{write/1} predicate.
%
Of course, if the Prolog variable \texttt{X} is already bound to
5, the predicate succeeds as well, while fails if \texttt{X} is
bound to anything else.
%
%%%%%%
Then, the static method \texttt{getVersion} is called, retrieving
the version of the class \texttt{Counter}, and printed using the
method \texttt{println} provided by the static \texttt{out} field
in the \texttt{java.lang.System} class.
%
The \texttt{name} public field of \texttt{myCounter} object is
then accessed, setting the \texttt{MyCounter2} value.
%
Finally, an array of 10 counters is created, and the
\texttt{myCounter} object assigned to its first element.
%%%%%%
%

The key point here is that the only requirement for this example
to run is the presence of the \texttt{Counter.class} file in the
proper position in the file system, according to Java naming
conventions: no other auxiliary information is needed -- no
headers, no pre-compilations, etc.
%
This enables the seamless reuse and exploitation of the large
amount of available Java libraries and resources, starting from
the standard ones, such as Swing to manage GUI components, JDBC to
access databases, RMI and CORBA for distributed computing, and so
on.
%
\xf{jexamples-swing} shows an example, where Java Swing API is
exploited to graphically choose a file from Prolog: a Swing
\texttt{JFileChooser} dialog is instantiated and bound to the
Prolog variable \texttt{Dialog} (a univocal Prolog atom of the form
\verb|'$obj_N'|, to be
used as the object reference, is automatically generated and bounded
to the variable) which
is then used to invoke methods \texttt{showOpenDialog} and
\texttt{getSelectedFile} of \texttt{JFileChooser}'s interface.
%
Further examples about exploiting standard Java libraries from
\tuprolog{}
%can be found in the Appendix A.
can be found in \cite{tuprolog-padl2001}.

\begin{figure}
\caption{Using a Swing componen from a \tuprolog{} program. Note the \texttt{\_} Prolog value used to represent the Java \texttt{null} value.
\labelfig{jexamples-swing}}
\begin{verbatim}
test_open_file_dialog(FileName) :-
    java_object('javax.swing.JFileChooser', [], Dialog),
    Dialog <- showOpenDialog(_),
    Dialog <- getSelectedFile returns File,
    File <- getName returns FileName.
\end{verbatim}
\end{figure}

Besides the Prolog predicates, \texttt{JavaLibrary} embeds the
\texttt{register} function, which, unlike the previous
functionalities, is to be used on the Java side.
%
Its purpose is to associate an existing Java object
\texttt{\textit{obj}} to a Prolog identifier
\texttt{\textit{ObjectRef}}, according to the syntax:
%
\begin{center}
 \small\tt
 boolean register(Struct \textit{ObjectRef}, Object \textit{obj})
    throws InvalidObjectIdException;\\
\end{center}
%
\texttt{\textit{ObjectRef}} is a ground term (otherwise an
exception is raised) that represents the Java object
\texttt{\textit{obj}} in the context of \texttt{JavaLibrary}'s
predicates: the function returns \texttt{false} if the object
represented by \texttt{\textit{obj}} is already registered under a
different \texttt{\textit{ObjectRef}}.
%
As an example of use, let us consider the following
case:\footnote{An
  explicit cast to \texttt{alice.tuprolog.lib.JavaLibrary} is needed because
  \texttt{loadLibrary} returns a reference to a generic
  \texttt{Library}, while the \texttt{register} primitive is defined in
  \texttt{JavaLibrary} only.}
%
{\small
\begin{verbatim}
Prolog core = new Prolog();
Library lib = core.loadLibrary("alice.tuprolog.lib.JavaLibrary");
((alice.tuprolog.lib.JavaLibrary)lib).register(new Struct("stdout"),
                                               System.out);
\end{verbatim}}
%
\noindent Here, the Java object \texttt{System.out} is registered
for use in \tuprolog{} under the name \texttt{stdout}.
%
So, within the scope of the \texttt{core} engine, a Prolog program
can now contain
\begin{verbatim}
stdout <- println('What a nice message!')
\end{verbatim}
as if \texttt{stdout} was a pre-defined \tuprolog{} identifier.

%---------------------------------------------------------------------
\section{Predicates}
%---------------------------------------------------------------------

\noindent Here follows a list of predicates implemented by this
library, grouped in categories corresponding to the functionalities
they provide.

%---------------------------------------------------------------------
\subsection{Method Invocation, Object and Class Creation}
%---------------------------------------------------------------------

\begin{itemize}
%
\item \bti{java\_object/3}\\
\noindent\bt{java\_object(ClassName, ArgList, ObjId)} is true iff
\bt{ClassName} is the full class name of a Java class available on
the local file system, \bt{ArgList} is a list of arguments that
can be meaningfully used to instantiate an object of the class,
and \bt{ObjId} can be used to reference such an object;
%
as a side effect, the Java object is created and the reference to
it is unified with \bt{ObjId}.
%
It is worth noting that \bt{ObjId} can be a Prolog variable (that
will be bound to a ground term) as well as a ground term (not a
number).\\
\template{java\_object(+full\_class\_name, +list, ?obj\_id)}
%
\item \bti{java\_object\_bt/3}\\
\noindent\bt{java\_object\_bt(ClassName, ArgList, ObjId)} has the same behaviour of \bt{java\_object/3}, but the binding that is established between the \bt{ObjId} term and the Java object is destroyed with backtracking.\\
\template{java\_object\_bt(+full\_class\_name, +list, ?obj\_id)}
%
\item \bti{destroy\_object/1}\\
\noindent\bt{destroy\_object(ObjId)} is true and as a side effect
the binding between \bt{ObjId} and a Java object,
possibly established, by previous predicates is destroyed.\\
\template{destroy\_object(@obj\_id)}
%
\item \bti{java\_class/4}\\
\noindent\bt{java\_class(ClassSourceText, FullClassName, ClassPathList, ObjId)}
is true iff \bt{ClassSouceText} is a source string describing a
valid Java class declaration, a class whose full name is
\bt{FullClassName}, according to the classes found in paths
listed in \bt{ClassPathList}, and \bt{ObjId} can be used as a
meaningful reference for a \texttt{java.lang.Class} object
representing that class;
%
as a side effect the described class is (possibly created and)
loaded and made available to the system.\\
\template{java\_class(@java\_source, @full\_class\_name, @list, ?obj\_id)}
%
\item \bti{java\_call/3}\\
\noindent\bt{java\_call(ObjId, MethodInfo, ObjIdResult)} is true iff
\bt{ObjId} is a ground term currently referencing a Java object,
which provides a method whose name is the functor name of the term
\bt{MethodInfo} and possible arguments the arguments of
\bt{MethodInfo} as a compound, and \bt{ObjIdResult} can be used as
a meaningful reference for the Java object that the method
possibly returns.
%
As a side effect the method is called on the Java object
referenced by the \bt{ObjId} and the object possibly returned by
the method invocation is referenced by the \bt{ObjIdResult} term.
%
The anonymous variable used as argument in the \bt{MethodInfo}
structure is interpreted as the Java \texttt{null} value.\\
\template{java\_call(@obj\_id, @method\_signature, ?obj\_id)}
%
\item \verb|'<-'/2|\\
\noindent\verb|'<-'(ObjId, MethodInfo)| is true iff \bt{ObjId} is
a ground term currently referencing a Java object, which provides a
method whose name is the functor name of the term \bt{MethodInfo}
and possible arguments the arguments of \bt{MethodInfo} as a
compound.
%
As a side effect the method is called on the Java object
referenced by the \bt{ObjId}.
%
The anonymous variable used as argument in the \bt{MethodInfo}
structure is interpreted as the Java \texttt{null} value.\\
\template{'<-'(@obj\_id, @method\_signature)}
%
\item \bti{return/2}\\
\noindent\verb|return('<-'(ObjId, MethodInfo), ObjIdResult)| is true
iff \bt{ObjId} is a ground term currently referencing a Java object,
which provides a method whose name is the functor name of the term
\bt{MethodInfo} and possible arguments the arguments of
\bt{MethodInfo} as a compound, and \bt{ObjIdResult} can be used as
a meaningful reference for the Java object that the method
possibly returns.
%
As a side effect the method is called on the Java object
referenced by the \bt{ObjId} and the object possibly returned by
the method invocation is referenced by the \bt{ObjIdResult} term.
%
The anonymous variable used as argument in the \bt{MethodInfo}
structure is interpreted as the Java \texttt{null} value.\\
%
It is worth noting that this predicate is equivalent to the
\texttt{java\_call} predicate.\\
\template{return('<-'(@obj\_id, @method\_signature), ?obj\_id)}
%
\end{itemize}

%---------------------------------------------------------------------
\subsection{Java Array Management}
%---------------------------------------------------------------------
\begin{itemize}
%
\item \bti{java\_array\_set/3}\\
\noindent\bt{java\_array\_set(ObjArrayId, Index, ObjId)} is true iff
\bt{ObjArrayId} is a ground term currently referencing a Java
array object, \bt{Index} is a valid index for the array and
\bt{ObjId} is a ground term currently referencing a Java object
that could inserted as an element of the array (according to Java
type rules).
%
As side effect, the object referenced by \bt{ObjId} is set in the
array referenced by \bt{ObjArrayId} in the position (starting from
0, following the Java convention) specified by \bt{Index}.
%
The anonymous variable used as \bt{ObjId} is interpreted as the
Java \texttt{null} value.
%
This predicate can be used for arrays of Java objects:
%
for arrays whose elements are Java primitive types (such as
\texttt{int}, \texttt{float}, etc.) the following predicates can
be used, with the same semantics of \bt{java\_array\_set} but
specifying directly the term to be set as a \tuprolog{} term
(according to the mapping described previously):\\
%
\mbox{~~~~}\bt{java\_array\_set\_int(ObjArrayId, Index, Integer)}\\
\mbox{~~~~}\bt{java\_array\_set\_short(ObjArrayId, Index, ShortInteger)}\\
\mbox{~~~~}\bt{java\_array\_set\_long(ObjArrayId, Index, LongInteger)}\\
\mbox{~~~~}\bt{java\_array\_set\_float(ObjArrayId, Index, Float)}\\
\mbox{~~~~}\bt{java\_array\_set\_double(ObjArrayId, Index, Double)}\\
\mbox{~~~~}\bt{java\_array\_set\_char(ObjArrayId, Index, Char)}\\
\mbox{~~~~}\bt{java\_array\_set\_byte(ObjArrayId, Index, Byte)}\\
\mbox{~~~~}\bt{java\_array\_set\_boolean(ObjArrayId, Index, Boolean)}\\
%
\template{java\_array\_set(@obj\_id, @positive\_integer, +obj\_id)}
%
%
%
\item \bti{java\_array\_get/3}\\
\noindent\bt{java\_array\_get(ObjArrayId, Index, ObjIdResult)} is
true iff \bt{ObjArrayId} is a ground term currently referencing a
Java array object, \bt{Index} is a valid index for the array, and
\bt{ObjIdResult} can be used as a meaningful reference for a Java
object contained in the array.
%
As a side effect, \bt{ObjIdResult} is unified with the reference to
the Java object of the array referenced by \bt{ObjArrayId} in the
\bt{Index} position.
%
This predicate can be used for arrays of Java objects:
%
for arrays whose elements are Java primitive types (such as
\texttt{int}, \texttt{float}, etc.) the following predicates can
be used, with the same semantics of \bt{java\_array\_get} but
binding directly the array element to a \tuprolog{} term
(according to the mapping described previously):\\
%
\mbox{~~~~}\bt{java\_array\_get\_int(ObjArrayId, Index, Integer)}\\
\mbox{~~~~}\bt{java\_array\_get\_short(ObjArrayId, Index, ShortInteger)}\\
\mbox{~~~~}\bt{java\_array\_get\_long(ObjArrayId, Index, LongInteger)}\\
\mbox{~~~~}\bt{java\_array\_get\_float(ObjArrayId, Index, Float)}\\
\mbox{~~~~}\bt{java\_array\_get\_double(ObjArrayId, Index, Double)}\\
\mbox{~~~~}\bt{java\_array\_get\_char(ObjArrayId, Index, Char)}\\
\mbox{~~~~}\bt{java\_array\_get\_byte(ObjArrayId, Index, Byte)}\\
\mbox{~~~~}\bt{java\_array\_get\_boolean(ObjArrayId, Index, Boolean)}\\
%
\template{java\_array\_get(@obj\_id, @positive\_integer, ?obj\_id)}
%
%
\item \bti{java\_array\_length/2}\\
\noindent\bt{java\_array\_length(ObjArrayId, ArrayLength)} is true
iff \bt{ArrayLength} is the length of the Java array referenced by
the term \bt{ObjArrayId}.\\
\template{java\_array\_length(@term, ?integer)}
%
\end{itemize}

%---------------------------------------------------------------------
\subsection{Helper Predicates}
%---------------------------------------------------------------------

\begin{itemize}
%
\item \bti{java\_object\_string/2}\\
\noindent\bt{java\_object\_string(ObjId, String)} is true iff
\bt{ObjId} is a term referencing a Java object and
\bt{PrologString} is the string representation of the object
(according to the semantics of the \texttt{toString} method
provided by the Java object).\\
\template{java\_object\_string(@obj\_id, ?string)}
%
\end{itemize}

%---------------------------------------------------------------------
\section{Functors}
%---------------------------------------------------------------------

No functors are provided by the \texttt{JavaLibrary} library.

%---------------------------------------------------------------------
\section{Operators}
%---------------------------------------------------------------------

\begin{table}[h]
    %
    \begin{center}{\small\tt
    \begin{tabular}{p{2cm}|p{1cm}|p{1cm}}\hline\hline
    Name & Assoc. & Prio. \\ \hline\hline
    <-   & xfx & 800\\
    returns     & xfx & 850 \\
    as   & xfx & 200\\
    .   & xfx & 600\\
    \hline\hline
    \end{tabular}
    }\end{center}
\end{table}

%\clearpage


%-----------------------------------------------------------------------
\section{Java Library Examples}
%-----------------------------------------------------------------------

The following examples are designed to show \texttt{JavaLibrary}'s
ease of use and flexibility.

%-------------------------------
\subsection{RMI Connection to a Remote Object}
%-------------------------------

Here we connect via RMI to a remote Java object.
%
In order to allow the reader to try this example with no need of
other objects, we connect to the remote Java object identified by
the name \verb|'prolog'|, which is an RMI server bundled with
the \tuprolog{} package, and can be spawned by typing:

{\small%
\texttt{java -Djava.security.all=policy.all  alice.tuprologx.runtime.rmi.Daemon}
}

\noindent Then, we invoke the object method whose signature is

{\small%
\texttt{SolveInfo solve(String goal);}
}
%
{\small%
\begin{verbatim}
    ?-  java_object('java.rmi.RMISecurityManager', [], Manager),
        class('java.lang.System') <- setSecurityManager(Manager),
        class('java.rmi.Naming') <- lookup('prolog') returns Engine,
        Engine <- solve('append([1],[2],X).') returns SolInfo,
        SolInfo <- success returns Ok,
        SolInfo <- getSubstitution returns Sub,
        Sub <- toString returns SubStr, write(SubStr), nl,
        SolInfo <- getSolution returns Sol,
        Sol <- toString returns SolStr, write(SolStr), nl.
\end{verbatim}
}
%
\noindent The Java version of the same code would be:
%
{\small%
\begin{verbatim}
        System.setSecurityManager(new RMISecurityManager());
        PrologRMI core = (PrologRMI) Naming.lookup("prolog");
        SolveInfo info = core.solve("append([1],[2],X).");
        boolean ok = info.success();
        String sub = info.getSubstiturion();
        System.out.println(sub);
        String sol = info.getSolution();
        System.out.println(sol);
\end{verbatim}
}


%-------------------------------
\subsection{Java Swing GUI from \tuprolog}
%-------------------------------

What about creating Java GUI components from the \tuprolog{}
environment?
%
Here is a little example, where a standard Java Swing open file
dialog windows is popped up:
%
{\small%
\begin{verbatim}
    open_file_dialog(FileName):-
        java_object('javax.swing.JFileChooser', [], Dialog ),
        Dialog <- showOpenDialog(_) returns Result,
        write(Result),
        Dialog <- getSelectedFile returns File,
        File <- getName returns FileName,
        class('java.lang.System') . out <- get(Out),
        Out <- println('you want to open file '),
        Out <- println(FileName).
\end{verbatim}
}

%-------------------------------
\subsection{Database access via JDBC from \tuprolog}
%-------------------------------

This example shows how to access a database via the Java standard
JDBC interface from \tuprolog{}.
%
The program computes the minimum path between two cities, fetching
the required data from the database called `distances'.
%
The entry point of the Prolog program is the \texttt{find\_path}
predicate.
%
{\small%
\begin{verbatim}
    find_path(From, To) :-
        init_dbase('jdbc:odbc:distances', Connection, '', ''),
        exec_query(Connection,
          'SELECT city_from, city_to, distance FROM distances.txt',
          ResultSet),
        assert_result(ResultSet),
        findall(pa(Length,L), paths(From,To,L,Length), PathList),
        current_prolog_flag(max_integer, Max),
        min_path(PathList, pa(Max,_), pa(MinLength,MinList)),
        outputResult(From, To, MinList, MinLength).

    paths(A, B, List, Length) :-
        path(A, B, List, Length, []).

    path(A, A, [], 0, _).
    path(A, B, [City|Cities], Length, VisitedCities) :-
        distance(A, City, Length1),
        not(member(City, VisitedCities)),
        path(City, B, Cities, Length2, [City|VisitedCities]),
        Length is Length1 + Length2.

    min_path([], X, X) :- !.
    min_path([pa(Length, List) | L],  pa(MinLen,MinList), Res) :-
        Length < MinLen, !,
        min_path(L, pa(Length,List), Res).
    min_path([_|MorePaths], CurrentMinPath, Res) :-
        min_path(MorePaths, CurrentMinPath, Res).

    writeList([]) :- !.
    writeList([X|L]) :- write(','), write(X), !, writeList(L).

    outputResult(From, To, [], _) :- !,
        write('no path found from '), write(From),
        write(' to '), write(To), nl.
    outputResult(From, To, MinList, MinLength) :-
        write('min path from '), write(From),
        write(' to '), write(To), write(': '),
        write(From), writeList(MinList),
        write('  - length: '), write(MinLength).

    % Access to Database

    init_dbase(DBase, Username, Password, Connection) :-
        class('java.lang.Class') <- forName('sun.jdbc.odbc.JdbcOdbcDriver'),
        class('java.sql.DriverManager') <- getConnection(DBase, Username, Password)
            returns Connection,
        write('[ Database '), write(DBase), write(' connected ]'), nl.

    exec_query(Connection, Query, ResultSet):-
        Connection <- createStatement returns Statement,
        Statement <- executeQuery(Query) returns ResultSet,
        write('[ query '), write(Query), write(' executed ]'), nl.

    assert_result(ResultSet) :-
        ResultSet <- next returns Valid, Valid == true, !,
        ResultSet <- getString('city_from') returns From,
        ResultSet <- getString('city_to') returns To,
        ResultSet <- getInt('distance') returns Dist,
        assert(distance(From, To, Dist)),
        assert_result(ResultSet).
    assert_result(_).
\end{verbatim}
}

%-------------------------------
\subsection{Dynamic compilation}
%-------------------------------

As already said, the \texttt{java\_class} predicate performs
\textit{dynamic compilation}, creating an instance of a Java
\texttt{Class} class that represents the public class declared in
the source text provided as argument.
%
The created \texttt{Class} instance, referenced by a Prolog term,
can be used to create instances via the \texttt{newInstance}
method, to retrieve specific constructors via the
\texttt{getConstructor} method, to analyze class methods and
fields, and for other above-mentioned meta-services: a sketch is
reported in \xf{dynamic-compilation}.
%
The \texttt{java\_class} arguments in the example specify, besides
the source text and the binding variable, the full class name
(\texttt{Counter}), which is necessary to locate the class in the
package hierarchy, and possibly a list of class paths required
for a successful compilation (if any).

\begin{figure}
\caption{Predicate \texttt{java\_class} performing dynamic compilation of Java code in \tuprolog{}.
\labelfig{dynamic-compilation}}
\begin{verbatim}
    ?- Source = 'public class Counter { ... }',
       java_class(Source, 'Counter', [], counterClass),
       counterClass <- newInstance returns myCounter,
       myCounter <- setValue(5),
       myCounter <- getValue returns X,
       write(X).
\end{verbatim}
\end{figure}

\xf{jcexamples} shows a more complex example, where a Java source
is retrieved via FTP and then exploited first to create a new
(previously unknown) class, and then a new instance of that class.
(The FTP service is provided by a shareware Java library.)
%
\begin{figure}
\caption{A new Java class is compiled and used after being retrieved via FTP.
\labelfig{jcexamples}}
{\scriptsize
\begin{verbatim}
% A user whose name is 'myName' and whose password is 'myPwd' gets the content of the file
% 'Counter.java' from the server whose IP address is 'srvAddr', creates the corresponding
% Java class and exploits it to instantiate and deploy an object

test :-
    get_remote_file('alice/tuprolog/test', 'Counter.java', srvAddr, myName, myPwd, Content),
    % creating the class
    java_class(Content, 'Counter', [], CounterClass),
    % instantiating (and using) an object of such a class
    CounterClass <- newInstance returns MyCounter,
    MyCounter <- setValue(303),
    MyCounter <- inc,
    MyCounter <- inc,
    MyCounter <- getValue returns Value,
    write(Value), nl.

% +DirName: Directory on the server where the file is located
% +FileName: Name of the file to be retrieved
% +FTPHost: IP address of the FTP server
% +FTPUser: User name of the FTP client
% +FTPPwd: Password of the FTP client
% -Content: Content of the retrieved file

get_remote_file(DirName, FileName, FTPHost, FTPUser, FTPPwd, Content) :-
    java_object('com.enterprisedt.net.ftp.FTPClient', [FTPHost], Client),
    % get file
    Client <- login(FTPUser, FTPPwd),
    Client <- chdir(DirName),
    Client <- get(FileName) returns Content,
    Client <- quit.
\end{verbatim}
}
\end{figure}
%
Though a lot remains to explore, \texttt{java\_class} features
seem quite interesting: in perspective one might think, for
instance, of a Prolog intelligent agent that dynamically acquires
information on a Java resource, and then autonomously builds up,
at run-time, the proper Java machinery enabling efficient
interaction with the resource.

%******************************************************************************%
\section{The IDE}

The \tuprolog\ system comes with a simple application providing an user friendly integrated development environment to interact with a \tuprolog\ engine, manipulate its knowledge base, make queries and explore solutions.
%
In addition, means to dynamically manage the loading and unloading of \tuprolog\ libraries are provided.
%
After a proper installation of the \tuprolog\ distribution, the application is spawned by launching the executable class \classname{alice.tuprologx.ide.GUILauncher}.
%
The console user interface version, providing a command-line shell, can be accessed by launching the executable class \classname{alice.tuprologx.ide.CUIConsole}.

\begin{figure}
\centering
\includegraphics[scale=0.60]{images/tuPrologIDE}
\caption{\tuprolog\ IDE.}
\label{tuprolog-ide}
\end{figure}

The main window of the \tuprolog\ IDE is shown in \figref{tuprolog-ide}.
%
It is divided in two sections:
%
\begin{itemize}
\item an editing area on the middle, providing means to edit the engine's current theory;
\item a console on the bottom, providing means to ask queries and display their solutions.
\end{itemize}
%
In the main window there also are:
%
a toolbar at the top, providing facilities to manage theories, such as load, save as well as create a new theory, to load and unload libraries into and from the \tuprolog\ engine, and to view in a separate window the debug informations activated by means of the \predicate{spy/0} predicate;
%
and a status bar at the very bottom, providing status informations for the IDE and the engine.

\subsection{Editing the theory}

The editing area allows multiple theories to be created and modified at the same time, by allocating a tab with a new text area for each theory.
%
The text area provides syntax highlighting for comments, string and list literals, and predefined predicates.
%
Undo and Redo actions are supported through the usual \keycap{Ctrl}+\keycap{Z} and \keycap{Ctrl}+\keycap{Shift}+\keycap{Z} key bindings.

\begin{figure}
\centering
\includegraphics[scale=0.60]{images/syntaxErrorFound}
\caption{A syntax error is found when setting the content of the editor area as the new engine's theory.}
\label{syntax-error-found}
\end{figure}

\begin{figure}
\centering
\includegraphics[scale=0.60]{images/setTheorySucceeded}
\caption{The syntax error is removed and the \guibutton{Set Theory} operation succeeds.}
\label{set-theory-succeeded}
\end{figure}

Above the editing tabs, a control area is found, where two buttons are provided to get the text of the engine's current theory into a new tab and to set the text contained in the editor of the selected tab as a new theory for the engine, and two buttons are provided for mouse-clicking support of Undo and Redo
actions.
%
An apposite action for retrieving the engine's current theory in an editor (shown in \figref{syntax-error-found}) is needed because whenever that theory gets modified by other means, such as calling the \texttt{consult/1} predicate, the changes are not automatically reflected in any text area.
%
On the left side of the control area, there also is an indicator of the line where the caret is currently positioned in the edit area.
%
Informations about the result of the action issued by the control area are provided in the status bar at the very bottom of the IDE's window:
%
for instance, when setting an invalid theory to the engine because of syntax errors, details about the error are provided.

\subsection{Solving goals}

The console at the bottom of the tuProlog IDE's window is subdivided in two logical panes:
%
\begin{itemize}
\item a query pane composed by a textfield where queries can be inserted, and two buttons to trigger the solving process.
%
The leftmost (\guibutton{Solve}) button asks the engine to find the first solution to the query, allowing the user to possibly navigate through further solutions;
%
the rightmost (\guibutton{Solve All}) button forces the engine to find all solutions to the given query.
%
Pressing the \keycap{Enter} key in the textfield has the same effect as pressing the \guibutton{Solve} button.
\item an answer pane, where answers and output informations are visualized.
%
Answers to Prolog queries are composed by both solutions, showed in a free form within a read-only text area, and bindings, displayed in tabular form.
%
The output tab provides a read-only view on the standard output where informations are possibly written by Prolog programs, by means of the I/O predicates supplied by the \classname{IOLibrary}.\footnote{The information written on standard output by methods invoked on Java objects from the \classname{JavaLibrary} -- for instance using the \varname{stdout} object -- are not displayed on this view.}
%
Control buttons are also provided to iterate through possibly multiple solutions, clear the bindings and output panes, and export tabular data in a convenient CSV format.
\end{itemize}
%
Goals are asked through the query input box, and answers (bindings and solutions) are provided in the related text area.
%
Query and answers are traced in a proper chronological history, that can be explored by means of \keycap{Up} and \keycap{Down} arrow keys from the query input textfield.
%
When open alternatives are found solving a goal, the \guibutton{Next} and \guibutton{Accept} buttons are enabled in the answer pane to interact with the engine, in order to let the user specify if the current solution is accepted or if other alternatives need to be explored.

Note that the theory contained in the currently selected edit pane does not have to be explicitly feeded to the Prolog engine before it could be possible to solve queries against that theory's knowledge base.
%
In fact, any time a goal is asked to be solved, the theory contained in the active edit area is automatically feeded to the engine if its knowledge base has been modified since the last solved goal.
%
(This obviously happens also on the first time a query is asked.)
%
However, whenever the engine's theory is modified by other means than the editor, it does need to be explicitly acquired and presented to the programmer in the text area.
%
In fact, if the theory in the engine is augmented by a call to the predicate \predicate{consult/1} issued from a query, for example, the contents of the newly consulted theory will not be automatically inserted in the editor:
%
when the programmer needs an up-to-date view of the knowledge base contained in the underlying \tuprolog\ engine, its display has to be explicitly triggered by means of the \guibutton{GetTheory} button, available
in the editing area.

An example of the user interaction involving multiple solutions is shown in the following sequence of figures:
%
in \figref{query-issued}, the user issued the query \userinput{test\_color(test, X).}, using the knowledge base written in the edit area (a solution to the Map Coloring problem,\footnote{The problem is to color a planar map so that no two adjoining regions have the same color. A famous conjecture was proved in 1976 showing that four colors are sufficient to color any planar map.} with a test map composed of six areas).
%
The first solution is displayed, and multiple open alternatives can be explored:
%
in \figref{next-solution-asked}, the user asked to get the next possible solution by pressing the \guibutton{Next} button, and another solution is provided;
%
finally, in \figref{accept-solution}, the user, after having explored the first two solutions, accepts the third one by pressing the \guibutton{Accept} button.
%
During the resolution of a goal, all the theory-related buttons are disabled, included the \guibutton{Library Manager} button, since each library can have its theory to be feeded into the engine.

\begin{figure}
\centering
\includegraphics[scale=0.605]{images/queryIssued}
\caption{The user issued a query \userinput{test\_color(test, X).} and the first solution is displayed.}
\label{query-issued}
\end{figure}

\begin{figure}
\centering
\includegraphics[scale=0.605]{images/nextSolutionAsked}
\caption{The user issued a \guibutton{Next} command and got another solution.}
\label{next-solution-asked}
\end{figure}

\begin{figure}
\centering
\includegraphics[scale=0.605]{images/acceptSolution}
\caption{The user accepted the third solution by pressing the \guibutton{Accept} button.}
\label{accept-solution}
\end{figure}

Near to the \guibutton{Next} and \guibutton{Accept} buttons, a \guibutton{Stop} button is found, providing the user with a means to halt the engine if a computation takes too long or a bug in the knowledge base feeded to the engine results in an infinite loop.

% Such a bug is contained in the following theory:
% \begin{verbatim}
% p(a).
% p(b) :- p(b).
% p(c).
% \end{verbatim}
% When solving a goal like \code{p(b)} or asking for the second solution to the query \code{p(X)}, the \tuprolog\ engine
% will be trapped in an infinite loop due to the particular recursive nature of the second clause in the feeded theory.
% By pressing the \guibutton{Stop} button, which is enabled only during computations, the user will be able to halt the
% engine and perform the necessary changes to the knowledge base before issuing another query, instead of being forced
% to close and reopen the IDE.

Finally, a \guibutton{Clear} button is provided, with the aim of allowing the user to clear the bindings and output panes when they get overfull with informations.
%
The button is enabled only when the proper tabs are selected.

\subsection{Debug Informations}

By pressing the \guibutton{View Debug Information} button, a new window is opened, providing a view on the warnings, produced by events such as the attempt at redefining a library predicate, and the spy information, concerning basic steps of the engine computation and state, possibly supplied by the engine during a goal demonstration:
%
Warnings are always active;
%
in order to activate the spy information notification, the \predicate{spy/0} built-in predicate (provided by \classname{BasicLibrary}) must be issued;
%
\predicate{nospy/0} can be used to stop this notification.
%
As an example, \figref{view-debug-information} shows the content of the spy information view after the
execution of a goal involving the activation of spy inspection.

It is worth noting that a computation may contain a huge number of traced steps.
%
For this reason, a toolbar at the top of the window allows to collapse and expand all nodes in the spy information pane, or to expand and collapse selected nodes only.
%
Finally, the content of the warnings and spy panes can be cleared using the \guibutton{Clear} button at the leftmost end of the toolbar.

\begin{figure}
\centering
\includegraphics[scale=0.605]{images/viewDebugInformation}
\caption{Debug Information View after the execution of a goal.}
\label{view-debug-information}
\end{figure}

\subsection{Dynamic library management}


A \tuprolog\ engine can be extended by loading any number of libraries, each provinding a specific set of built-in
predicates and functors, and a related theory. The \tuprolog\ IDE allows a dynamic management of libraries through a
GUI dialog, which can be displayed by pressing the \guibutton{Open Library Manager} button in the toolbar. The Library
Manager dialog is shown in \figref{library-manager-dialog}.

\begin{figure}
\centering
\includegraphics[scale=0.605]{images/libraryManagerDialog}
\caption{The Library Manager dialog.}
\label{library-manager-dialog}
\end{figure}

This dialog displays a list of the libraries currently loaded into the \tuprolog\ engine. For a new instance of the
engine, that list will typically contain the four standard libraries coming with the application core, that is
\classname{BasicLibrary}, \classname{IOLibrary}, \classname{ISOLibrary}, \classname{JavaLibrary}, along with their
current status. The user can add a library to the Library Manager simply by providing the fully qualified name of the
library's class in the textfield on the top of the dialog, then pressing the \guibutton{Add} button: the added library
will be displayed with an initial Unload status. The user can further select the status of each library in the list,
and commit changes to the \tuprolog\ engine by pressing the \guibutton{OK} button, or dismiss the dialog by pressing
the \guibutton{Cancel} button.

The library manager is also capable of updating itself accordingly to the events of libraries load and unload fired by
the \tuprolog\ engine.  Such events are triggered by the use of the \verb|load_library/1| and \verb|unload_library/1|
predicates or directives in query issued or theories feeded to the engine. So, if an user asks to solve the goal \verb|load_library('TestLibrary'), test(X).|,
for example, the manager would immediately reflect the change occurred in the engine's libraries pool, adding a new
entry if \verb|TestLibrary| had not been previously loaded or, if necessary, changing the library's entry status to
show the result of the loading action.

Both the action of adding a library to the manager and the action of loading a library into the engine can fail. If,
for example, the classname provided does not identify a \tuprolog\ library (i.e. it identifies a class not extending the
\classname{alice.tuprolog.Library} class) or the identified class does not exist, an appropriate message will appear
in the status bar at the bottom of the dialog. When adding or loading a library, please remember that every class
needed by that library must be in the classpath in order to have the library correctly added to the manager's list or
loaded into the engine. 
%******************************************************************************%
%=======================================================================
\chapter{Using \tuprolog{} from Java}
\label{java-api}
%=======================================================================

\section{Getting started}

Let's begin with your first Java program using \tuprolog{}.
%
{\small{
\begin{verbatim}
    import alice.tuprolog.*;

    public class Test2P {
        public static void main(String[] args) throws Exception {
            Prolog engine = new Prolog();
            SolveInfo info = engine.solve("append([1],[2,3],X).");
            System.out.println(info.getSolution());
        }
    }
\end{verbatim}
}}
\noindent In this first example a \tuprolog{} engine is
created, and asked to solve a query, provided in a textual form.
%
This query in the Java environment is equivalent to the query\\\\
%
{\small{\texttt{?- append([1],[2,3],X).\\\\}}}
%
\noindent in a classic Prolog environment, and accounts for
finding the list that is obtained by appending the list
\texttt{[2,3]} to the list \texttt{[1]} (\texttt{append} is
included in the theory provided by the
\texttt{alice.tuprolog.lib.BasicLibrary}, which is downloaded by
the default when instantiating the engine).
%

By properly compiling and executing this simple program,\footnote{Save the program in a file called \texttt{Test2P.java}, then compile it with
%
\texttt{javac -classpath tuprolog.jar Test2P.java}
%
and then execute it with
%
\texttt{java -cp .;tuprolog.jar Test2P.java}} the string
\texttt{append([1],[2,3],[1,2,3])} -- that is the solution of out
query -- will be displayed on the standard output.
%
%

\noindent Then, let's consider a little bit more involved example:

{\small{
\begin{verbatim}
public class Test2P {
    public static void main(String[] args) throws Exception {
        Prolog engine = new Prolog();
        SolveInfo info = engine.solve("append(X,Y,[1,2]).");
        while (info.isSuccess()) {
            System.out.println("solution: " + info.getSolution() +
                               " - bindings: " + info);
            if (engine.hasOpenAlternatives()) {
                info = engine.solveNext();
            } else {
                break;
            }
        }
    }
}
\end{verbatim}
}}

In this case, all the solutions of a query are retrieved and
displayed, with also the variable bindings:
{\small{\begin{verbatim}

solution: append([],[1,2],[1,2]) - bindings: Y / [1,2]  X / []
solution: append([1],[2],[1,2]) - bindings: Y / [2]  X / [1]
solution: append([1,2],[],[1,2]) - bindings: Y / []  X / [1,2]

\end{verbatim}
}}

\section{Basic Data Structures}

\noindent All Prolog data objects are mapped onto Java objects:
\texttt{Term} is the base class for Prolog untyped terms such as
atoms and compound terms (represented by the \texttt{Struct} class),
Prolog variables (\texttt{Var} class) and Prolog typed terms such
as numbers (\texttt{Int}, \texttt{Long}, \texttt{Float},
\texttt{Double} classes).
%
In particular:

%
\begin{itemize}
    %
    \item \texttt{Term} -- this abstract class represents a generic Prolog term, and it is the
    root base class of \tuprolog{} data structures, defining the basic common services, such
    as term comparison, term unification and so on.
    %
    It is worth noting that it is an abstract class, so no direct \texttt{Term}
    objects can be instantiated;
    %
    %
    %
    \item \texttt{Var} -- this class (derived from \texttt{Term})
    represents \tuprolog{} variables.
    %
    A variable can be anonymous, created by means of the default constructor, with no
    arguments, or identified by a name, that must starts with an upper case letter or an
    underscore;
    %
    \item \texttt{Struct} -- this class (derived from \texttt{Term})
    represents un-typed \tuprolog{} terms, such as atoms, lists and compound terms;
    %
    \texttt{Struct} objects are characterised by a functor name and a
    list of arguments (which are \texttt{Term}s themselves), while
    \texttt{Var} objects are labelled by a string representing the
    Prolog name.
    %
    Atoms are mapped as \texttt{Struct}s with functor with a name and
    no arguments;
    %
    Lists are mapped as \texttt{Struct} objects with functor
    \verb|'.'|, and two \texttt{Term} arguments (head and tail of the list);
    %
    lists can be also built directly by exploiting the 2-arguments constructor, with
    head and tail terms as arguments.
    %
    Empty list is constructed by means of the no-argument constructor of \texttt{Struct}
    (default constructor).
    %
    \item \texttt{Number} -- this abstract class represents typed, numerical Prolog terms,
        and it is the  base class of \tuprolog{} number classes;
    %
    \item \texttt{Int, Long, Double, Float} -- these classes (derived from \texttt{Number})
    represent  typed numerical \tuprolog{} terms, respectively integer, long, double and
    float numbers.
    %
    Following the Java conventions, the default type for integer number is \texttt{Int}
    (integer, not long, number), and for \texttt{Double} (and so double), for floating point
    number.
\end{itemize}


\noindent Some examples of term creation follow:
%
{\small{\begin{verbatim}

// constructs the atom vodka
Struct drink = new Struct("vodka");

// constructs the number 40
Term degree = new alice.tuprolog.Int(40);

// constructs the compound degree(vodka, 40)
Term drinkDegree = new Struct("degree",
                              new Struct("vodka"),
                              new Int(40));
// second way to constructs the compound degree(vodka,40)
Struct drinkDegree2 = new Struct("degree", drink, degree);

// constructs the compound temperature('Rome', 25.5)
Struct temperature = new Struct("temperature",
                                new Struct("Rome"),
                                new alice.tuprolog.Float(25.5));

// constructs the compound equals(X, X)
Struct t1 = new Struct("equals", new Var("X"), new Var("X"));
t1.resolveTerm();

// mother(John,Mary)
Struct father = new Struct(new Struct("John"), new Struct("Mary")));

// father(John, _)
Term  father = new Struct(new Struct("John"), new Var());

// p(1, _, q(Y, 3.03, 'Hotel'))
Term  t2 = new Struct("p",
                      new Int(1),
                      new Var(),
                      new Struct("q",
                                 new Var("Y"),
                                 new Float(3.03f),
                                 new Struct("Hotel")));

// The Long number 130373303303
Term t3 = new alice.tuprolog.Long(130373303303h);

// The double precision number 1.7625465346573
Term t4 = new alice.tuprolog.Double(1.7625465346573);

// an empty list
Struct empty = new Struct();

// the list [303]
Struct l = new Struct(new Int(303), new Struct());

// the list [1,2,apples]
Struct alist = new Struct(
                   new Int(1),
                   new Struct(
                       new Int(2),
                       new Struct("apples")));

// fruits([apple, orange | _ ])
Term list2 = new Struct("fruits", new Struct(
                                      new Struct("apple",
                                          new Struct("orange"),
                                          new Var())));

// complex_compound(1, _, q(Y, 3.03, 'Hotel', k(Y,X)), [303, 13, _, Y])
Term t5 = Term.parse(
     "complex_compound(1, _, q(Y, 3.03, 'Hotel', k(Y,X)), [303, 13, _, Y])"
);

\end{verbatim}}}
%
\noindent The name of the \tuprolog{} number classes
(\texttt{Int}, \texttt{Float}, \texttt{Long}, \texttt{Double})
follows the name of the primitive Java data type they represents.
%
Note that due to class name clashes (for instance between classes
\texttt{java.lang.Long} and \texttt{alice.tuprolog.Long}), it
could be necessary to use the full class name to identify
\tuprolog{} classes.
%

\section{Engine, Theories and Libraries}

\noindent Then, the other main classes that make \tuprolog{} Core
API concern \tuprolog{} engines, theories and libraries. In
particular:
%
\begin{itemize}
    \item \texttt{Prolog} -- this class represent \tuprolog{}
    engines.
    %
    This class provides a minimal interface that enables users
    to: \\
    %
    \indent{-- set/get the theory to be used for
    demonstrations;}\\
        %
    \indent{-- load/unload libraries;} \\
        %
    \indent{-- solve a goal, represented either by a \texttt{Term} object or by a
        textual representation (a \texttt{String} object) of a
        term.}\\
    %
    A \tuprolog{} engine can be instantiated either with some standard default
    libraries loaded, by means of the default constructor, or with
    a starting set of libraries, which can be empty, provided as
    argument to the constructor (see JavaDoc documentation for
    details).
    %
    Accordingly, a raw, very lightweight, \tuprolog{} engine can
    be created by specifying an empty set of library, providing
    natively a very small set of built-in primitives.


    \item \texttt{Theory} -- this class represent \tuprolog{}
    theories.
    %
    A theory is represented by a text, consisting of a series of
    clauses and/or directives, each followed by a dot and a
    whitespace character.
    %
    Instances of this class are built either from a textual representation,
    directly provided as a string or taken by any possible input
    stream, or from a list of terms representing Prolog clauses.
    %
    %
    %
    %
    %
    \item \texttt{Library} -- this class represents \tuprolog{}
    libraries;
    %
    A \texttt{tuprolog} engine can be dynamically extended by loading
    (and unloading) any number of libraries; each library can provide
    a specific set of of built-ins predicates, functors and a related
    theory.
    %
    A library can be loaded by means of the built-in by means of the method
    \texttt{loadLibrary} of the \tuprolog{} engine.
    %
    Some standard libraries are provided in the
    \texttt{alice.tuprolog.lib} package and loaded by the default
    instantiation of a \tuprolog{} engine:
    %
    \texttt{alice.tuprolog.lib.BasicLibrary}, providing basic and
    well-known Prolog built-ins, \texttt{alice.tuprolog.lib.IOLibrary}
    providing \textit{de facto} standard Prolog I/O predicates, \texttt{alice.tuprolog.lib.ISOLibrary}
    providing some ISO predicates/functors not directly provided
    by \texttt{BasicLibrary} and \texttt{IOLibrary}, and
    \texttt{alice.tuprolog.lib.JavaLibrary}, which enables the
    creation and usage of Java objects from \tuprolog{} programs,
    in particular enabling the reuse of any existing Java resources.
    %
    %
    \item \texttt{SolveInfo} -- this class represents the result of a
    demonstration and instances of these class are returned by the
    \texttt{solve} methods the \texttt{Prolog} engines;
    %
    in particular \texttt{SolveInfo} objects provide services to test the
    success of the demonstration (\texttt{isSuccess} method),
    to access to the term solution of the query
    (\texttt{getSolution} method)  and to access the list of the
    variable with their bindings.
\end{itemize}
%

Some notes about \tuprolog{} terms and the services they provide:
\begin{itemize}
%
\item the static \texttt{parse} method provides a quick way to get a
term from its string representation.
%
\item \tuprolog{} terms provides directly methods for unification and matching: \\
%
{\tt{\small{public boolean unify(Term t)}}}\\
%
{\tt{\small{public boolean match(Term t)}}}\\
%
Terms that have been subject to unification outside a
demonstration context (that is invoking directly these methods,
and not passing through the solving service of an engine) should
not be used then in queries to engines.
%
\item some services are provided to compare terms, according to the
Prolog rules, and to check their type;
%
in particular the standard Java method \texttt{equals} has the
same semantics of the method \texttt{isEqual} which follows the
Prolog comparison semantics.
%
\item some services makes it possible to copy a term as it is
or to get a renamed copy of the term (\texttt{copy} and
\texttt{getRenamedCopy});
%
it is worth noting that the design of \tuprolog{} promotes a
stateless usage of terms; in particular, it is good practice not
to reuse the same terms in different demonstration contexts, as
part of different queries.
%
\item the method \texttt{getTerm} is useful in the case of
variables, providing the term linked possibly considering all the
linking chain in the case of variables referring other variables.
%
\item when a term is created by means of the proper constructor,
consider as example: \\\\
%
{\tt{\scriptsize{Struct myTerm = new Struct("p", new Var("X"), new
Int(1), new Var("X"))}}}\\\\
%
it is \emph{not resolved}, in the sense that possible variable
terms with the same name in the term do not refer each other;
%
so in the example the first and the third argument of the compound
\texttt{myTerm} point to different variable objects.
%
A term is resolved the first time it is involved in a matching or
unification context.
%
\end{itemize}

\noindent Some notes about \tuprolog{} engines, theories,
libraries and the services they provide:

\begin{itemize}
%
\item \tuprolog{} engines support natively some
\emph{directives}, that can be defined by means of the :-/1
predicate in theory specification.
%
Directives are used to specify properties of clauses and of the
engine (\emph{solve/1}, \emph{initialization/1},
\emph{set\_prolog\_flag/1}, \emph{load\_library/1},
\emph{consult/1}), format and syntax of read-terms (\emph{op/3},
\emph{char\_conversion/2}).
%
\item \tuprolog{} engines support natively the dynamic definition
and management of \emph{flags} (or property), used to describe
some aspects of libraries and their built-ins.
%
A flag is identified by a name (an alphanumeric atom), a list of
possible values, a default value and a boolean value specifying if
the flag value can be modified.
%
\item \tuprolog{} engines are thread-safe. The methods that could
create problems in being used in a multi-threaded context are now
synchronised.
%
\item \tuprolog{} engines have no (static) dependencies with each
other, multiple engines can be created independently as simple
objects on the same Java virtual machine,  each with its own
configuration (theory and loaded libraries).
%
Moreover, accordingly to the design of \tuprolog{} system in
general, engines are very lightweight, making suitable the use of
multiple engines in the same execution context.
%
\item \tuprolog{} engines can be serialised and stored as a persistent
object or sent through the network.
%
This is true also for engines with pre-loaded standard libraries:
%
in the case that other libraries are loaded, these must be
serializable in order to have the engine serializable.
%
\end{itemize}

%
%

\section{Some more examples of \tuprolog{} usage}

\noindent Creation of an engine (with default libraries
pre-loaded):

{\tt\scriptsize{
\begin{verbatim}
    import alice.tuprolog.*;

    ...
    Prolog engine = new Prolog();
\end{verbatim} }}


\noindent Creation of an engine specifying only the
\texttt{BasicLibrary} as pre-loaded library:

{\tt\scriptsize{
\begin{verbatim}
    import alice.tuprolog.*;

    ...
    Prolog engine = new Prolog(new String[]{"alice.tuprolog.lib.BasicLibrary"});
\end{verbatim} }}

\noindent Creation and loading of a theory from a string:

{\tt\scriptsize{\begin{verbatim}

    String theoryText = "my_append([],X,X).\n" +
                        "my_append([X|L1],L2,[X|L3]) :- my_append(L1,L2,L3).\n";

    Theory theory = new Theory(theoryText);
    try {
        engine.setTheory(theory);
    } catch(InvalidTheoryException e) {
    }
\end{verbatim} }}

\noindent Creation and loading of a theory from an input stream:

{\tt\scriptsize{\begin{verbatim}

    Theory theory = new Theory(new FileInputStream("test.pl");
    try {
        engine.setTheory(theory);
    } catch(InvalidTheoryException e) {
    }
\end{verbatim} }}

\noindent Goal demonstration (provided as a string):

{\tt\scriptsize{\begin{verbatim}

    // ?- append(X,Y,[1,2,3]).
    try {
        SolveInfo info = engine.solve("append(X,Y,[1,2,3]).");
        Term solution = info.getSolution();
    } catch(MalformedGoalException mge) {
        ...
    } catch(NoSolutionException nse) {
        ...
    }
\end{verbatim} }}

\noindent Goal demonstration (provided as a Term):

{\tt\scriptsize{\begin{verbatim}

    try {
        Term goal = new Struct("p", new Int(1), new Var("X"));
        try {
            // ?- p(1,X).
            SolveInfo info = engine.solve(goal);
            Term solution = info.getSolution();
 
        } catch (NoSolutionException nse) {
        }
    } catch (InvalidVarNameException ivne) {
    }
\end{verbatim} }}

\noindent Getting another solution:

{\tt\scriptsize{\begin{verbatim}
    try {
        SolveInfo info = engine.solve(goal);
        info = engine.solveNext();
    } catch(NoMoreSolutionException e)
\end{verbatim} }}

\noindent Loading a library:

{\tt\scriptsize{\begin{verbatim}
    try {
        engine.loadLibrary('alice.tuprologx.lib.TucsonLibrary');
    } catch(InvalidLibraryException e) {
    }
\end{verbatim} }}

\noindent Here, a complete example of interaction with a
\tuprolog{} engine is shown (refer to the JavaDoc documentation for
details about interfaces):

{\tt\scriptsize{\begin{verbatim}

import alice.tuprolog.*; import java.io.*;

public class Test2P {
    public static void main (String args[]) {
        Prolog engine = new Prolog();
        try {

            // solving a goal
            SolveInfo info = engine.solve(new Struct("append",
                                              new Var("X"),
                                              new Var("Y"),
                                              new Struct(new Term[]{new Struct("hotel"),
                                                                    new Int(303),
                                                                    new Var()})));


            // note we could use strings:
            // SolveInfo info = engine.solve("append(X, Y, [hotel, 303, _]).");

            // test for demonsration success
            if (info.isSuccess()) {

                // acquire solution and substitution
                Term sol = info.getSolution();
                System.out.println("Solution: " + sol);

                System.out.println("Bindings: " + info);

                // open choice points?
                if (engine.hasOpenAlternatives()) {

                    // ask for another solution
                    info = engine.solveNext();

                    if (info.isSuccess()) {
                        System.out.println("An other substitution: " + info);
                    }
                }
            }

            // other frequent interactions

            // setting a new theory in the engine
            String theory = "p(X,Y) :- q(X), r(Y).\n" +
                            "q(1).\n" +
                            "r(1).\n" +
                            "r(2).\n";
            engine.setTheory(new Theory(theory));

            SolveInfo info2 = engine.solve("p(1,X).");
            System.out.println(info2);

            // retrieving the theory from a file
            FileOutputStream os=new FileOutputStream("test.pl");
            os.write(theory.getBytes());
            os.close();
            engine.setTheory(new Theory(new FileInputStream("test.pl")));
            info2 = engine.solve("p(X,X).");
            System.out.println(info2.getSolution());

        } catch (Exception ex) {
            ex.printStackTrace();
        }
    }
}
\end{verbatim}}}

With the program execution, the following string are displayed on
the standard output:

{\tt\small{

\begin{verbatim}
Solution: append([],[hotel,303,_],[hotel,303,_])
Bindings: Y /[hotel,303,_] X / []
An other substitution: Y / [303,_]  X / [hotel]
X / 1
p(1,1)
\end{verbatim}}}
%******************************************************************************%
%=======================================================================
\chapter{How to Develop New Libraries}
\label{ch:howto-develop-libraries}
%=======================================================================

Libraries are \tuprolog{}'s way to achieve the desired characteristics
of minimality, dynamic configurability, and straightforward
Prolog-to-Java integration.
%
Libraries are reflection-based, and can be written both in Prolog
and Java: other languages may be used indirectly, via JNI (Java
Native Interface).
%
At the \tuprolog{} side, exploiting a library written in Java
requires no pre-declaration of the new built-ins, nor any other
special mechanism: all is needed is the presence of the
corresponding \texttt{.class} library file in the proper location
in the file system.

\section{Implementation details}

Syntactically, a library developed in Java must extend the base
abstract class \texttt{alice.tuprolog.Library}, provided within
the \tuprolog{} package, and define new \textit{predicates} and/or
\textit{evaluable functors} and/or \textit{directives} in the form
of methods, following a simple signature convention.
%
In particular, new predicates must adhere to the signature:
%
\begin{center}
\small\tt
    public boolean <\textit{pred name}>\_<\textit{N}>(\textit{T1} arg1,
\textit{T2} arg2, ...,\textit{Tn} argN)
\end{center}
%
while evaluable functors must follow the form:
%
\begin{center}
    \small\tt
    public Term <\textit{eval funct name}>\_<\textit{N}>(\textit{T1} arg1,
\textit{T2} arg2, ...,\textit{Tn} argN)
\end{center}
%
and directives must be provided with the signature:
%
\begin{center}
    \small\tt
    public void <\textit{dir name}>\_<\textit{N}>(\textit{T1} arg1,
\textit{T2} arg2, ..., \textit{Tn} argN)
\end{center}
%
where \textit{T1}, \textit{T2}, ... \textit{Tn} are \texttt{Term} or derived
classes, such as \texttt{Struct}, \texttt{Var}, \texttt{Long}, etc., defined in
the \tuprolog{} package, constituting  the Java counterparts of
the corresponding Prolog data types.
%
The parameters represent the arguments actually passed to the built-in
predicate, functor, or directive.

%
A library defines also a new piece of theory, which is collected
by the Prolog engine through a call to the library method \texttt{String
getTheory()}.
%
By default, this method returns an empty theory: libraries which need to
add a Prolog theory must override it accordingly.
%
Note that only the external representation of a library's theory is
constrained to be in \texttt{String} form; the internal implementation
can be freely chosen by the library designer. However, using a Java
\texttt{String} for wrapping a library's Prolog code guarantees
self-containment while loading libraries through remote mechanisms such
as RMI.

\begin{table}[h]
    %
    \caption{Predicate and functor definitions in Java and their use
    in a \tuprolog{} program.\labeltab{java-preds}}
    %
    \begin{center}{\small\tt
    \begin{tabular}{p{10cm}|p{3.25cm}}
     \hline
     & \\
    \textit{// sample library} & \textit{\% tuProlog test program}\\
     import alice.tuprolog.*; & \\
     & \\
    public class TestLibrary extends Library \{                            &
 test :-\\
    ~~\textit{// builtin functor sum(A,B)}                          & ~~N is sum(5,6),\\
    ~~public Term sum\_2(Number arg0, Number arg1)\{    & ~~println(N).\\
    ~~~~float   arg0 = arg0.floatValue();        & \\
    ~~~~float   arg1 = arg1.floatValue();        & \\
    ~~~~return new Float(arg0+arg1);                                & \\
    ~~\}                                                            & \\
    ~~\textit{// builtin predicate println(Message)}                & \\
    ~~public boolean println\_1(Term arg)\{                      & \\
    ~~~~System.out.println(arg);                                   & \\
    ~~~~return true;                                                & \\
    ~~\}                                                            & \\
    \}                                                              & \\
     & \\
     \hline
    \end{tabular}
    }\end{center}
\end{table}

\xt{java-preds} shows a couple of examples about how a predicate
(such as \texttt{println/1}) and an evaluable functor (such as
\texttt{sum/2}) can be defined in Java and exploited from
\tuprolog{}.
%
The Java method \texttt{sum\_2}, which implements the evaluable
functor \texttt{sum/2}, is passed two \texttt{Number} terms (5 and 6)
which are then used (via \texttt{getTerm}) to retrieve the two
(float) arguments to be summed.
%
In the same way, method \texttt{println\_1}, which implements the
predicate \texttt{println/1}, receives \texttt{N} as \texttt{arg},
and retrieves its actual value via \texttt{getTerm}: since this is
a predicate, a boolean value (\texttt{true} in this case) is returned.
%

The developer of a library may face two corner case as far as method
naming is concerned: the first happens when the name of the
predicate, functor or directive she is defining contains a symbol
which cannot legally appear in a Java method's name; the second
occurs when he has to define a predicate and a directive with the
same Prolog signature, which Java would not be able to tell apart
because it cannot distinguish signatures of methods differing for
their return type only.
%
To overcome this kind of issues, a {\em synonym map} can be
constructed under the form of an array of \texttt{String} arrays,
and returned by the appropriate \texttt{getSynonymMap} method,
defined as abstract by the \texttt{Library} class. In both the cases
described above, another name must be chosen for the Prolog
executable element the library's developer want to define: then, by
means of the synonym map, that fake name can be associated with the
real name and the type of the element, be it a predicate, a functor
or a directive.
%
For example, if a definition for an evaluable functor representing
addition is needed, but the symbol \texttt{+} cannot appear in a
Java method's name, a method called \texttt{add} can be defined and
associated to its original Prolog name and its function by inserting
the array \texttt{\{"+", "add", "functor"\}} in the synonym map.

\section{Library Name}

% Lib name
%
By default, the name of the library coincides with the full class name of the
class implementing it.
%
However, it is possible to define explicitly the name of a library by
overriding the \texttt{getName} method, and returning as a string
the real name.
%
For example:
%
\begin{verbatim}
package acme;
import alice.tuprolog.*;
public class MyLib_ver00 extents Library {
    public String getName(){
        return "MyLibrary";
    }
    ...
}
\end{verbatim}

This class defines a library called \texttt{MyLibrary}.
%
It can be loaded into a Prolog engine by using the
\texttt{loadLibrary} method on the Java side, or a
\texttt{load\_library} built-in predicate on the Prolog side,
specifying the full class name (\texttt{acme.MyLib\_{ve00}}).
%
It can be unloaded then dynamically using the \texttt{unloadLibrary}
method (or the corresponding \texttt{unload\_library} built-in),
specifying instead the \textit{library name} (\texttt{MyLibrary}).

%******************************************************************************%
%=======================================================================
\chapter{\tuprolog{} Exceptions}
ss\label{ch:exceptions}
%=======================================================================

%=======================================================================
\section{Exceptions in ISO Prolog}
\label{sec:exceptions in ISO prolog}
%=======================================================================
Exception handling was first introduced in the ISO Prolog standard (ISO/IEC 13211-1) in 1995.

The first distinction has to be made between \textit{errors} and \textit{exceptions}.
%
An \textit{error} is a particular circumstance that interrupts the execution of a Prolog program: when a Prolog engine encounters an error, it raises an \textit{exception}.
%
The exception handling support is supposed to intercept the exception and transfer the execution flow to a suitable exception handler, with any relevant information. Two basic principles are followed during this operation:

\begin{itemize}
  \item \textit{error bounding} -- an error must be bounded and not propagate through the entire program: in particular, an error occurring inside a given component must either be captured at the component's frontier, or remain invisible and be reported nicely.
      According to ISO Prolog, this is done via the \texttt{catch/3} predicate.

  \item \textit{atomic jump} -- the exception handling mechanism must be able to exit atomically from any number of nested execution contexts. According to ISO Prolog, this is done via the \texttt{throw/1} predicate.
\end{itemize}
%
In practice, the \texttt{catch(\textit{Goal}, \textit{Catcher}, \textit{Handler})} predicate enables the controlled execution of a goal, while the \texttt{throw(\textit{Error})} predicates makes it possible to raise an exception---very much like the \texttt{try/catch} construct of many imperative languages.

Semantically, executing the \texttt{catch(\textit{Goal}, \textit{Catcher}, \textit{Handler})} means that \texttt{\textit{Goal}} is first executed: if an error occurs, the subgoal where the error occurred is replaced by the corresponding \texttt{throw(\textit{Error})}, which raises the exception.
%
Then, a matching \texttt{catch/3} clause -- that is, a clause whose second argument
unifies with \texttt{\textit{Error}} -- is searched among the ancestor nodes in the resolution tree: if one is found, the path in the resolution tree is cut, the catcher itself is removed (because it only applies to the protected goal, not to the handler), and the \texttt{\textit{Handler}} predicate is executed. If, instead, no such matching clause is found, the execution simply fails.

So, \texttt{catch(\textit{Goal}, \textit{Catcher}, \textit{Handler})} performs exactly like \texttt{\textit{Goal}} if no exception are raised: otherwise, all the choicepoints generated by \texttt{\textit{Goal}} are cut, a matching \texttt{\textit{Catcher}} is looked for, and if one is found \texttt{\textit{Handler}} is executed, maintaining the substitutions made during the previous unification process.
%
Then, execution continues with the subgoal following \texttt{catch/3}.
%
Any side effects possibly occurred during the execution of a goal are \textit{not} undone in case of exceptions---as it normally happens when a predicate fails.

Summing up, \texttt{catch/3} succeeds if:
\begin{itemize}
  \item \texttt{call(\textit{Goal})} succeeds \textit{(standard behaviour)};\\\\
        --OR--
  \item \texttt{call(\textit{Goal})} is interrupted by a call to
      \texttt{throw(\textit{Error})} whose \texttt{\textit{Error}} unifies with
      \texttt{\textit{Catcher}}, and the subsequent \texttt{call(\textit{Handler})} succeeds.
\end{itemize}

\noindent If \texttt{\textit{Goal}} is non-deterministic, it can be executed again in
backtracking. However, since all the choicepoints of \texttt{\textit{Goal}} are cut in case of exception, \texttt{\textit{Handler}} \textit{is possibly executed just once}.

\smallskip

\noindent As an example, let us consider the following toy program:
\begin{verbatim}
    p(X):- throw(error), write('---').
    p(X):- write('+++').
\end{verbatim}

\noindent with the following query:

\begin{verbatim}
    ?:- catch(p(0), E, write(E)), fail.
\end{verbatim}
which tries to execute \texttt{p(0)}, catching any exception \texttt{E} and handling the error by just printing it on the standard output (\texttt{write(E)}).

Perhaps surprisingly, the program will just print \texttt{'error'}, not \texttt{'error---'} or \texttt{'error+++'}. The reason is that once the exception is raised, the execution of \texttt{p(X)} is aborted, and after the handler terminates the execution proceeds with the subgoal following \texttt{catch/3}, i.e. \texttt{fail}.
So, \texttt{write('---')} is never reached, nor is \texttt{write('+++')} since all the choicepoints are cut upon exception.

%-----------------------------------------------------------------------
\subsection{Error classification}
%-----------------------------------------------------------------------
This classification was already presented in Section \ref{sec:exception-support} above as a hint to predicate and functor readability: however, we report it here too both for completeness and for the reader's convenience.

When an exception is raised, the relevant error information is also transferred by instantiating a suitable \textit{error term}.

The ISO Prolog standard prescribes that such a term follows the pattern
\texttt{error(\textit{Error\_term}, \textit{Implementation\_defined\_term})} where
\texttt{\textit{Error\_term}} is constrained by the standard to a pre-defined set of values (the error categories), and \texttt{\textit{Implementation\_defined\_term}} is an optional term providing implementation-specific details.
%
Ten error categories are defined:
\begin{enumerate}
  \item \texttt{instantiation\_error}: when the argument of a predicate or one of its components is an unbound variable, which should have been instantiated. Example: \texttt{X is Y+1} when \texttt{Y} is not instantiated at the time \texttt{is/2} is evaluated.

  \item \texttt{type\_error(\textit{ValidType}, \textit{Culprit})}: when the type of an argument of a predicate, or one of its components, is instantiated, but is bound to the wrong type of data. \texttt{\textit{ValidType}} represents the expected data type (one of \texttt{atom}, \texttt{atomic}, \texttt{byte}, \texttt{callable}, \texttt{character}, \texttt{evaluable}, \texttt{in\_byte}, \texttt{in\_character}, \texttt{integer}, \texttt{list}, \texttt{number}, \texttt{predicate\_indicator}, \texttt{variable}), and \texttt{\textit{Culprit}} is the actual (wrong) type found.
      Example: a predicate expecting months to be represented as integers in the range 1--12 called with an argument like \texttt{march} instead of \texttt{3}.

  \item \texttt{domain\_error(\textit{ValidDomain}, \textit{Culprit})}: when the argument type is correct, but its value falls outside the expected range.
      \texttt{\textit{ValidDomain}} is one of \texttt{character\_code\_list},
      \texttt{not\_empty\_list}, \texttt{not\_less\_than\_zero}, \texttt{close\_option}, \texttt{io\_mode}, \texttt{operator\_priority}, \texttt{operator\_specifier}, \texttt{flag\_value}, \texttt{prolog\_flag}, \texttt{read\_option}, \texttt{write\_option}, \texttt{source\_sink}, \texttt{stream}, \texttt{stream\_option}, \texttt{stream\_or\_alias}, \texttt{stream\_position},\\
      \texttt{stream\_property}. Example: a predicate expecting months as above, called with an out-of-range argument like \texttt{13}.

  \item \texttt{existence\_error(\textit{ObjectType}, \textit{ObjectName}}): when the referenced object does not exist. \texttt{\textit{ObjectType}} is
      the type of the unexisting object (one of \texttt{procedure}, \texttt{source\_sink}, or \texttt{stream}), and \texttt{\textit{ObjectName}} is the missing object's name. Example: trying to access an unexisting file like \texttt{usr/goofy} leads to an
      \texttt{existence\_error(stream, 'usr/goofy')}.

  \item \texttt{permission\_error(\textit{Operation}, \textit{ObjectType}, \textit{Object})}: whenever\\
       \texttt{\textit{Operation}} (one of \texttt{access}, \texttt{create}, \texttt{input}, \texttt{modify}, \texttt{open}, \texttt{output}, or \texttt{reposition}) is not allowed on \texttt{\textit{Object}}, of type \texttt{\textit{ObjectType}} (one of  \texttt{binary\_stream}, \texttt{past\_end\_of\_stream}, \texttt{operator}, \texttt{private\_procedure}, \texttt{static\_procedure}, \texttt{source\_sink}, \texttt{stream}, \texttt{text\_stream}, \texttt{flag}).

  \item \texttt{representation\_error(\textit{Flag})}: when an implementation-defined limit, whose category is given by \texttt{\textit{Flag}} (one of
      \texttt{character}, \texttt{character\_code}, \texttt{in\_character\_code}, \texttt{max\_arity}, \texttt{max\_integer}, \texttt{min\_integer}), is violated during execution.

  \item \texttt{evaluation\_error(\textit{Error})}: when the evaluation of a function produces an out-of-range value (one of \texttt{float\_overflow}, \texttt{int\_overflow}, \texttt{undefined}, \texttt{underflow}, \texttt{zero\_divisor}).

  \item \texttt{resource\_error(\textit{Resource})}: when the Prolog engine does not have enough resources to complete the execution of the goal. \texttt{Resource} can be any term useful to describe the situation. Examples: maximum number of opened files reached, no further available memory, etc.

  \item \texttt{syntax\_error(\textit{Message})}: when data read from an external source have an incorrect format or cannot be processed for some reason. \texttt{\textit{Message}} can be any term useful to describe the situation.

  \item \texttt{system\_error}: any other unexpected error not falling into the previous categories.
\end{enumerate}

%=======================================================================
\section{Exceptions in \tuprolog}
\label{sec:exceptions in tuprolog}
%=======================================================================

\tuprolog{} aims to fully comply to ISO Prolog exceptions.
%
In the following, a set of mini-examples are presented which highlight each one single aspect of \tuprolog{} compliance to the ISO standard.

%-----------------------------------------------------------------------
\subsection{Examples}
%-----------------------------------------------------------------------

\medskip\noindent
\textit{\textbf{Example 1:} \texttt{Handler} must be executed maintaining the substitutions made during the unification process between \texttt{Error} and \texttt{Catcher}}

Program: \texttt{p(0) :- throw(error).}

Query: \texttt{ ?- catch(p(0), E, atom\_length(E, Length)).}

Answer: \texttt{ yes.}

Substitutions: \texttt{E/error}, \texttt{Length/5}


\medskip\noindent
\textit{\textbf{Example 2:} the selected \texttt{Catcher} must be the nearest in the resolution tree whose second argument unifies with \texttt{Error}}

Program: \texttt{p(0) :- throw(error).}\\
\mbox{\texttt{~~~~~~~~~~~}}\texttt{p(1).}

Query: \texttt{ ?- catch(p(1), E, fail),  catch(p(0), E, true).}

Answer: \texttt{ yes.}

Substitutions: \texttt{E/error}


\medskip\noindent
\textit{\textbf{Example 3:} execution must fail if an error occurs during a goal execution and there is no matching \texttt{catch/3} predicate whose second argument unifies with \texttt{Error}}

Program: \texttt{p(0) :- throw(error).}

Query: \texttt{ ?- catch(p(0), error(X), true).}

Answer: \texttt{ no.}


\medskip\noindent
\textit{\textbf{Example 4:} execution must fail if \texttt{Handler} is false}

Program: \texttt{p(0) :- throw(error).}

Query: \texttt{ ?- catch(p(0), E, false).}

Answer: \texttt{ no.}


\medskip\noindent
\textit{\textbf{Example 5:} if \texttt{Goal} is non-deterministic, it is executed again on backtracking, but in case of exception all the choicepoints must be cut, and \texttt{Handler} must be executed only once}

Program: \texttt{p(0).}\\
\mbox{\texttt{~~~~~~~~~~~}}\texttt{p(1) :- throw(error).}\\
\mbox{\texttt{~~~~~~~~~~~}}\texttt{p(2).}

Query: \texttt{ ?- catch(p(X), E, true).}

Answer: \texttt{ yes.}

Substitutions: \texttt{X/0}, \texttt{E/error}

Choice: \texttt{ Next solution?}

Answer: \texttt{ yes.}

Substitutions: \texttt{X/1}, \texttt{E/error}

Choice: \texttt{ Next solution?}

Answer: \texttt{ no.}


\medskip\noindent
\textit{\textbf{Example 6:} execution must fail if an exception occurs in \texttt{Handler}}

Program: \texttt{p(0) :- throw(error).}

Query: \texttt{ ?- catch(p(0), E, throw(err)).}

Answer: \texttt{ no.}

%-----------------------------------------------------------------------
\subsection{Handling Java/.NET Exceptions from \tuprolog}
\label{ssec:java-exceptions-in-tuprolog}
%-----------------------------------------------------------------------

One peculiar aspect of \tuprolog{} is the ability to support multi-paradigm programming, mixing object-oriented (mainly, but not exclusively, Java) and Prolog in several ways---in particular, by enabling Java objects to be accessed and exploited from Prolog world via OOLibrary (see Section \ref{sec:java-library}) and by enabling .NET objects to be accessed and exploited from Prolog world via OOLibrary (see Section \ref{sec:dotnet-oolibrary})
%
In this context, the problem arises of properly sensing and handling Java/.NET exceptions from the Prolog side.

At a first sight, one might think of re-mapping such exceptions and constructs onto the Prolog ones, but this approach is unsatisfactory for three reasons:
%
\begin{itemize}
  \item the semantics of the Java/.NET mechanism should not be mixed with the Prolog one, and vice-versa;

  \item the Java/.NET construct admits also a \texttt{finally} clause which has no counterpart in ISO Prolog exceptions;

  \item the Java/.NET catching mechanisms operates hierarchically, while the \texttt{catch/3} predicate operates via pattern matching and unification, allowing for a finer-grain, more flexibly exception filtering.
\end{itemize}

%\noindent Accordingly, Java/.NET exceptions in \tuprolog{} programs are handled by means of two further, \textit{ad hoc} predicates: \texttt{java\_throw/1} and \texttt{java\_catch/3} in the Java case, and \texttt{oo\_throw/1} and \texttt{oo\_catch/3} in the .NET case, respectively.
%%
%Since their behavior can be fully understood only in the context of JavaLibrary/OOLibrary, we forward the reader to Sections \ref{sec:java-library} and \ref{sec:dotnet-oolibrary}, respectively, for further information.
%
\noindent Accordingly, Java/.NET exceptions in \tuprolog{} programs are handled by means of an \textit{ad hoc} predicate, called \texttt{java\_catch/3} in the Java case and \texttt{oo\_catch/3} in the .NET case, respectively.
%
Since their behavior can be fully understood only in the context of OOLibrary, we forward the reader to Sections \ref{sec:java-library} and \ref{sec:dotnet-oolibrary}, respectively, for further information.

%%---------------------------------------------------------------------------------------
%\section*{Appendix: Implementation notes}
%%---------------------------------------------------------------------------------------
%
%Implementing exceptions in \tuprolog{} does not mean just to extend the engine to support the above mechanisms: given its library-based design, and its intrinsic support to multi-paradigm programming, adding exceptions in \tuprolog{} has also meant (1) to revise all the existing libraries, modifying any library predicate so that it raises the appropriate type of exception instead of just failing; and (2) to carefully define and implement a model to make Prolog exceptions not only coexist, but also fruitfully operate with the Java or .NET imperative world, which brings its own concept of exception and its own handling mechanism.
%
%As a preliminary step, the finite-state machine which constitutes the core of the \tuprolog{} engine was extended with a new \textit{Exception} state, between the existing \textit{Goal Evaluation} and \textit{Goal Selection} states \cite{iuliani-masterthesis-2009}.
%
%Then, all the \tuprolog{} libraries were revised, according to clearness and efficiency criteria --- that is, the introduction of the new checks required for proper exception raising should not reduce performance unacceptably. This issue was particularly relevant for runtime checks, such as \texttt{existence\_error}s or \texttt{evaluation\_error}s; moreover, since \tuprolog{} libraries could also be implemented partly in Prolog and partly in Java, careful choices had to be made so as to introduce such checks at the most adequate level in order to intercept all errors while maintaining code readability and overall organisation, while guaranteeing efficiency.
%
%This led to intervene with extra Java checks for libraries fully implemented in Java, and with new ''Java guards'' for predicates implemented in Prolog, keeping the use of Prolog meta-predicates (such as \texttt{integer/1}) to a minimum.
%
%\bigskip
%
%Per quel che riguarda il modo in cui \`{e} stato implementato il meccanismo di controllo degli errori, bisogna distinguere i predicati espressi in Java da quelli espressi in Prolog.
%
%Nel primo caso le eccezioni (cio\`{e} le opportune istanze di \texttt{PrologError}) vengono lanciate direttamente dai corrispondenti metodi Java ogniqualvolta si verifica un errore, mentre nel secondo caso sono lanciate da metodi ``guardia" (sempre espressi in Java) invocati per controllare i parametri prima dell'esecuzione del predicato Prolog.
%
%Nell'implementazione si \`{e} cercato di individuare il maggior numero possibile di condizioni di errore, rispettando per\`{o} sempre il requisito fondamentale di correttezza: se una chiamata a un predicato non falliva prima dell'introduzione del meccanismo delle eccezioni, non deve fallire neanche ora---ovvero, il lancio di una eccezione deve avvenire soltanto in circostanze in cui il motore tuProlog originario falliva.
%
%La correttezza del comportamento del motore \`{e} garantita anche se ci si dimentica di identificare qualche condizione di errore inaspettata: in questo caso infatti il motore non lancia un'eccezione, ma comunque fallisce.
%%
%Ci\`{o} permette ad un utente sia di gestire gli errori che si possono verificare durante l'esecuzione, sia di non gestirli, nel qual caso l'esecuzione fallir\`{a} e dunque l'estensione rester\`{a} trasparente.
%
%A parte le inevitabili modifiche ai built-in e alle librerie (\textit{BasicLibrary}, \textit{ISOLibrary}, \textit{IOLibrary}, \textit{DCGLibrary}), sono state necessarie le seguenti semplici modifiche al motore:
%
%\begin{itemize}
%
%\item alla classe \texttt{alice.tuprolog.FlagManager} sono stati aggiunti due metodi per ricavare informazioni su un flag:
%
%\begin{itemize}
%\item \texttt{boolean isModifiable(String name)}\\
%        che restituisce true se esiste nel motore un flag di nome \texttt{name}, e tale flag \`{e} modificabile;
%
%\item \texttt{boolean isValidValue(String name, Term Value)}\\
%        che restituisce true se esiste nel motore un flag di nome \texttt{name}, e \texttt{Value} \`{e} un valore ammissibile per tale flag.
%\end{itemize}
%
%  \item il metodo \texttt{getEngineManager} della classe \texttt{alice.tuprolog.Prolog} \`{e} ora pubblico (in precedenza aveva visibilit\`{a} di package) per permettere alle librerie di ricavare dal motore l'informazione sul goal correntemente in esecuzione e inserirla nell'eccezione lanciata;
%
%  \item il metodo \texttt{evalAsFunctor} della classe \texttt{alice.tuprolog.PrimitiveInfo} lancia ora un'istanza di \texttt{Throwable} in caso di errore durante la valutazione del goal, mentre prima ritornava \texttt{null}, per permettere di discriminare il tipo di errore verificatosi durante la valutazione di un funtore;
%
%  \item analogamente, il metodo \texttt{evalExpression} di \texttt{alice.tuprolog.Library} rilancia ora l'istanza di \texttt{Throwable} ricevuta dal metodo \texttt{evalAsFuntor} di \texttt{alice.tuprolog.PrimitiveInfo}.
%\end{itemize}


%******************************************************************************%
\bibliography{bibliography}
\bibliographystyle{plain}
%******************************************************************************%
\end{document}
%******************************************************************************%
